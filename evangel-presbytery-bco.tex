% Options for packages loaded elsewhere
\PassOptionsToPackage{unicode}{hyperref}
\PassOptionsToPackage{hyphens}{url}
\PassOptionsToPackage{dvipsnames,svgnames,x11names}{xcolor}
%
\documentclass[
]{book}
\usepackage{amsmath,amssymb}
\usepackage{lmodern}
\usepackage{iftex}
\ifPDFTeX
  \usepackage[T1]{fontenc}
  \usepackage[utf8]{inputenc}
  \usepackage{textcomp} % provide euro and other symbols
\else % if luatex or xetex
  \usepackage{unicode-math}
  \defaultfontfeatures{Scale=MatchLowercase}
  \defaultfontfeatures[\rmfamily]{Ligatures=TeX,Scale=1}
\fi
% Use upquote if available, for straight quotes in verbatim environments
\IfFileExists{upquote.sty}{\usepackage{upquote}}{}
\IfFileExists{microtype.sty}{% use microtype if available
  \usepackage[]{microtype}
  \UseMicrotypeSet[protrusion]{basicmath} % disable protrusion for tt fonts
}{}
\makeatletter
\@ifundefined{KOMAClassName}{% if non-KOMA class
  \IfFileExists{parskip.sty}{%
    \usepackage{parskip}
  }{% else
    \setlength{\parindent}{0pt}
    \setlength{\parskip}{6pt plus 2pt minus 1pt}}
}{% if KOMA class
  \KOMAoptions{parskip=half}}
\makeatother
\usepackage{xcolor}
\usepackage{longtable,booktabs,array}
\usepackage{calc} % for calculating minipage widths
% Correct order of tables after \paragraph or \subparagraph
\usepackage{etoolbox}
\makeatletter
\patchcmd\longtable{\par}{\if@noskipsec\mbox{}\fi\par}{}{}
\makeatother
% Allow footnotes in longtable head/foot
\IfFileExists{footnotehyper.sty}{\usepackage{footnotehyper}}{\usepackage{footnote}}
\makesavenoteenv{longtable}
\usepackage{graphicx}
\makeatletter
\def\maxwidth{\ifdim\Gin@nat@width>\linewidth\linewidth\else\Gin@nat@width\fi}
\def\maxheight{\ifdim\Gin@nat@height>\textheight\textheight\else\Gin@nat@height\fi}
\makeatother
% Scale images if necessary, so that they will not overflow the page
% margins by default, and it is still possible to overwrite the defaults
% using explicit options in \includegraphics[width, height, ...]{}
\setkeys{Gin}{width=\maxwidth,height=\maxheight,keepaspectratio}
% Set default figure placement to htbp
\makeatletter
\def\fps@figure{htbp}
\makeatother
\setlength{\emergencystretch}{3em} % prevent overfull lines
\providecommand{\tightlist}{%
  \setlength{\itemsep}{0pt}\setlength{\parskip}{0pt}}
\setcounter{secnumdepth}{5}
\usepackage{booktabs}
\usepackage{amsthm}
\makeatletter
\def\thm@space@setup{%
  \thm@preskip=8pt plus 2pt minus 4pt
  \thm@postskip=\thm@preskip
}
\makeatother

\usepackage[a4paper]{geometry}
\geometry{left=2cm}
\geometry{right=2cm}
\geometry{bottom=2cm}
\geometry{top=2cm}

%\usepackage{hyperref}
%\hypersetup{
%    colorlinks=true,
%    linkcolor=blue,
%    filecolor=magenta,
%    urlcolor=cyan,
%    }

% https://tex.stackexchange.com/a/233271
\usepackage[explicit]{titlesec}
\titleformat{name=\section,numberless}[hang]{}{}{0cm}{%
  \LARGE #1\markboth{#1}{#1}%
}

\frontmatter
\ifLuaTeX
  \usepackage{selnolig}  % disable illegal ligatures
\fi
\usepackage[]{natbib}
\bibliographystyle{plainnat}
\IfFileExists{bookmark.sty}{\usepackage{bookmark}}{\usepackage{hyperref}}
\IfFileExists{xurl.sty}{\usepackage{xurl}}{} % add URL line breaks if available
\urlstyle{same} % disable monospaced font for URLs
\hypersetup{
  pdftitle={Evangel Presbytery Book of Church Order},
  pdfauthor={Evangel Presbytery},
  colorlinks=true,
  linkcolor={Maroon},
  filecolor={Maroon},
  citecolor={Blue},
  urlcolor={Blue},
  pdfcreator={LaTeX via pandoc}}

\title{Evangel Presbytery Book of Church Order}
\author{Evangel Presbytery}
\date{2022-09-13}

\begin{document}
\maketitle



{
\hypersetup{linkcolor=}
\setcounter{tocdepth}{1}
\tableofcontents
}
\hypertarget{welcome}{%
\chapter*{Welcome}\label{welcome}}
\addcontentsline{toc}{chapter}{Welcome}

This is the official Book of Church Order for Evangel Presbytery. You can always find the latest online version at \url{https://bco.evangelpresbytery.com} and download the \href{https://bco.evangelpresbytery.com/evangel-presbytery-bco.pdf}{latest PDF version here}. A record of the changes can be found in the \href{https://bco.evangelpresbytery.com/updates.html}{Updates} section at the end of the book.

\mainmatter

\hypertarget{preface}{%
\chapter*{Preface}\label{preface}}
\addcontentsline{toc}{chapter}{Preface}

\protect\hypertarget{front-matter-preface}{\href{}{}}

\hypertarget{a.-historical-statement-and-declaration-of-intent}{%
\section*{A. Historical Statement and Declaration of Intent}\label{a.-historical-statement-and-declaration-of-intent}}
\addcontentsline{toc}{section}{A. Historical Statement and Declaration of Intent}

The Lord has blessed His Church with many precious gifts for equipping the saints for the work of ministry and for building up the body of Christ. Among these gifts are basic biblical principles for the system of government and discipline of His Church. Applying this wisdom to our time and place, we learn how to conduct ourselves in the Household of Faith as we labor with other congregations and undershepherds in mutual accountability and love.

We believe the Book of Church Order, whose roots stretch back to Calvin's Geneva, is an excellent outworking of these principles. John Calvin wrote the first modern Presbyterian Book of Order for the church at Geneva in 1542.

John Knox learned under John Calvin's teaching and pastoral ministry for several years, and then returned to Scotland. He wrote The First Book of Discipline for the Presbyterian Church of Scotland, in 1560. The whole history of Presbyterian church government in Scotland goes back to this First Book of Discipline.

The Westminster Assembly, which met in London in 1643, wrote not only the Confession of Faith and Catechisms, but also The Form of Presbyterian Church Government. The Presbyterian Churches of England, Scotland, and Ireland adopted this Westminster Form of Government.

When our Presbyterian forefathers came to America they brought with them the Westminster ``Form of Presbyterian Church Government,'' and it became the basis of Church law in the American Presbyterian Church.

The first General Assembly of the Presbyterian Church in America was organized in 1789. The General Synod in preparing for the organization of the General Assembly practically rewrote The Form of Presbyterian Church Government in 1788, in order to adjust it to the conditions in America. This new book was called The Form of Government and Discipline of the Presbyterian Church in the United States of America. It was revised a number of times prior to 1861 and the beginning of the Civil War, when the Northern and Southern Presbyterian Churches were tragically divided. At that time, the Southern Presbyterians withdrew and formed The Presbyterian Church in the United States.

When the General Assembly of the Presbyterian Church in the United States (PCUS) was organized on December 4, 1861, it adopted the Form of Government and Discipline which had been in use since 1788. In 1863 the General Assembly took steps to revise this Form of Government and Discipline with the result that a thoroughgoing revision was adopted in 1879. A great many amendments were added during the next forty years.

In 1921 the PCUS General Assembly took steps to revise the Book of Church Order again. Another thoroughgoing revision was proposed by the Committee on Revision, adopted by the General Assembly, approved by a large majority of the Presbyteries, and enacted into law by the General Assembly of 1925.

The Form of Government and Rules of Discipline of this 2019 Book of Church Order are based in part on a 1933 revision to the PCUS Book of Church Order and portions of the current Presbyterian Church in America Book of Church Order.\footnote{All quotations from The Book of Church Order of the Presbyterian Church in America used by permission from the Office of the Stated Clerk of the General Assembly of the Presbyterian Church in America.} The Directory for Worship is based on the one in use by the Orthodox Presbyterian Church.

While this current version seeks to likewise build on the solid foundation of centuries of Reformed polity, an important difference should be noted. The fellowship of churches governed by this Book of Church Order is committed to the good and pleasant unity\footnote{Psalm 133.} of brothers whose convictions about the time and mode of baptism differ. We grieve over the almost-perpetual division of reformed believers which has been watched by the world and the larger evangelical community during the last century, as it has been demonstrated over a whole host of secondary doctrinal matters. It's our desire to demonstrate unity amidst diversity at points where the system of doctrine taught by reformed standards such as the Belgic Confession, the London Baptist Confession, and the Westminster Standards, etc. is not at stake. We believe the time and mode of baptism to be such a place.

Our work to keep the unity of the Spirit through the bond of peace means placing ourselves in yoke with like-minded churches who have committed themselves to both aspects of the Reformation Motto: The Church Reformed and Always Reforming. Perhaps no work of reform is more desperately needed in the church today than discipline and accountability. An expositor of a prior version of the Book of Church Order observed---in 1898---the dangerous tendency in the church of that time to avoid discipline and the inevitable harmful fruit of such unfaithfulness:

\begin{quote}
Discipline, thorough and scriptural, is possible under our system. It is not necessary to argue that discipline is a duty enjoined in the Scriptures; but there is among us, one is tempted to say, a pervading infidelity of the worth of such teachings. Outside of the discipline of ministers charged with heresies, and of very notorious offenders in morality, there is seldom anything in the nature of judicial prosecution among us; and there is reason to believe that there is even less of that forewarning which looks forward to such prosecution. . . .

It is not easy to exercise discipline, not only on account of the imperfection of those who are to exercise it, but also on account of the strength of corruption that has come for the lack of discipline; and discipline is especially difficult where the revenues of the church come from voluntary contributions. To censure offenders generally endangers revenue. It requires a lofty indifference to financial considerations in comparison with spiritual results, or the inexperience of youth, to embolden to attempt thorough discipline. Many attempts have failed largely because the men who failed when they had less wisdom of experience and less maturity of spiritual growth have not attempted it when they became better qualified. Their former failures, and the new Book, make them afraid. But we must come to it or we perish. The churches of America must learn to exercise discipline, or the experiment of religious liberty, without financial aid from the civil power, will prove a failure. Such a result will not come, for the churches will learn this lesson of discipline. It may be through bitter experience of the fruits of laxity and of the consequent worldly corruption of the church, but to discipline the church must come. And it is here insisted that we have the usable machinery of discipline, and all we need now is the spiritual power to make it efficient.\footnote{F.~P. Ramsay, \emph{An Exposition of the Form of Government and the Rules of Discipline of the Presbyterian Church in the United States} (Presbyterian Committee of Publication, 1898), 7-9.}
\end{quote}

Now, 120 years later, we are in no less need of the Holy Spirit's sanctifying presence and power. And so we commit this work to God's hands, trusting Him to supply all our needs according to His riches in glory in Christ Jesus.

\hypertarget{b.-preliminary-principles}{%
\section*{B. Preliminary Principles}\label{b.-preliminary-principles}}
\addcontentsline{toc}{section}{B. Preliminary Principles}

The members of Evangel Presbytery, in presenting to the Christian public the system of union, and the form of government and discipline which they have adopted, have thought proper to state, by way of introduction, a few of the general principles by which they have been governed in the formation of the plan. This, it is hoped, will, in some measure, prevent those rash misconstructions, and uncandid reflections, which usually proceed from an imperfect view of any subject; as well as make the several parts of the system plain, and the whole perspicuous and fully understood. They are unanimously of opinion:

\begin{enumerate}
\def\labelenumi{\arabic{enumi}.}
\item
  God alone is Lord of the conscience and has left it free from any doctrines or commandments of men (a) which are in any respect contrary to the Word of God, or (b) which, in regard to matters of faith and worship, are not governed by the Word of God. Therefore, the rights of private judgment in all matters that respect religion are universal and inalienable.
\item
  In perfect consistency with the above principle, every Christian Church, or union or association of particular churches, is entitled to declare the terms of admission into its communion and the qualifications of its ministers and members, as well as the whole system of its internal government which Christ has appointed. In the exercise of this right it may, notwithstanding, err in making the terms of communion either too lax or too narrow; yet even in this case, it does not infringe upon the liberty or the rights of others, but only makes an improper use of its own.
\item
  Our blessed Savior, for the edification of the visible Church, which is His body, has appointed officers not only to preach the Gospel and administer the Sacraments, but also to exercise discipline for the preservation both of truth and duty. It is incumbent upon these officers and upon the whole Church in whose name they act, to censure or cast out the erroneous and scandalous, observing in all cases the rules contained in the Word of God.
\item
  Godliness is founded on truth. A test of truth is its power to promote holiness according to our Savior's rule, ``By their fruits ye shall know them.''\footnote{Matthew 7:20.} No opinion can be more pernicious or more absurd than that which brings truth and falsehood upon the same level. On the contrary, there is an inseparable connection between faith and practice, truth and duty. Otherwise it would be of no consequence either to discover truth or to embrace it.
\end{enumerate}

\begin{enumerate}
\def\labelenumi{\arabic{enumi}.}
\setcounter{enumi}{4}
\item
  While, under the conviction of the above principle, it is necessary to make effective provision that all who are admitted as teachers be sound in the faith, there are truths and forms with respect to which men of good character and principles may differ. In all these it is the duty both of private Christians and societies to exercise mutual forbearance towards each other.
\item
  Though the character, qualifications and authority of church officers are laid down in the Holy Scriptures, as well as the proper method of officer investiture, the power to elect persons to the exercise of authority in any particular society resides in that society.
\item
  All church power, whether exercised by the body in general, or by representation, is only ministerial and declarative since the Holy Scriptures are the only rule of faith and practice. No church judicatory may make laws to bind the conscience. All church courts may err through human frailty, yet it rests upon them to uphold the laws of Scripture though this obligation be lodged with fallible men.
\item
  Since ecclesiastical discipline derives its force only from the power and authority of Christ, the great Head of the Church Universal, it must be purely moral and spiritual in its nature.
\end{enumerate}

If the preceding scriptural principles be steadfastly adhered to, the vigor and strictness of disciplines will contribute to the glory and well-being of the Church.

\hypertarget{c.-constitution-defined}{%
\section*{C. Constitution Defined}\label{c.-constitution-defined}}
\addcontentsline{toc}{section}{C. Constitution Defined}

The Constitution of Evangel Presbytery, which is subject to and subordinate to the Scriptures of the Old and New Testaments, the infallible Word of God, consists of its doctrinal standards set forth in the \emph{Westminster Confession of Faith,}\footnote{American revisions as adopted by the Orthodox Presbyterian Church in 1936, \url{https://evangelpresbytery.com/westminster-confession-of-faith}.} together with the \emph{Larger}\footnote{\url{https://evangelpresbytery.com/westminster-larger-catechism}.} and \emph{Shorter Catechisms}\footnote{\url{https://evangelpresbytery.com/westminster-shorter-catechism}.}; the Book of Church Order, which comprises the Form of Government, the Rules of Discipline, and the Directory for the Worship of God; and the Apostles' Creed, Nicene Creed, the Chalcedonian Creed, and Athanasian Creed; all as adopted by the Presbytery (BCO \protect\hyperlink{29.1}{29.1}).

\hypertarget{form-of-government}{%
\chapter*{Form of Government}\label{form-of-government}}
\addcontentsline{toc}{chapter}{Form of Government}

\hypertarget{the-doctrine-of-church-government}{%
\section*{1. The Doctrine of Church Government}\label{the-doctrine-of-church-government}}
\addcontentsline{toc}{section}{1. The Doctrine of Church Government}

\protect\hypertarget{part-main-body-2}{\href{}{}}
\protect\hypertarget{chapter-slug-1-the-doctrine-of-church-government}{\href{}{}}

\begin{enumerate}
\def\labelenumi{\arabic{enumi}.}
\tightlist
\item
  \protect\hypertarget{1}{\href{}{}}The scriptural form of Church government, which is that of Presbytery, is comprehended under five heads, namely: 1. The Church; 2. Its Members; 3. Its Officers; 4. Its Courts; and 5. Its Orders.
\item
  The Church which the Lord Jesus Christ has erected in this world for the gathering and perfecting of the saints, is His visible kingdom of grace, and is one and the same in all ages.
\item
  \protect\hypertarget{1.3}{\href{}{}}The members of this visible Church catholic are all those persons in every nation who make profession of their faith in the Lord Jesus Christ and promise submission to His laws. Communing members are those who have made a profession of faith in Christ, have been baptized, and have been admitted by the Session to the Lord's Table. The children of communing members are, through the covenant and by right of birth, non-communing members of the church. The children of communing members holding credo-baptistic convictions are non-communing members in the sense that they are entitled to the church's care, love, discipline, and training, with a view to their embracing Christ and thus possessing personally all the benefits of the covenant. The children of communing members holding paedo-baptistic convictions are non-communing members in the sense that they are entitled to baptism and to the church's care, love, discipline, and training, with a view to their embracing Christ and thus possessing personally all the benefits of the covenant.
\item
  The officers of the Church, by whom all its powers are administered, are, according to the Scriptures, Ministers of the Word, Ruling Elders, and Deacons.
\item
  Ecclesiastical jurisdiction is not a several, but a joint power, to be exercised by Presbyters in courts. These courts may have jurisdiction over one or many churches, but they sustain such mutual relations as to realize the idea of the unity of the Church.
\item
  The ordination of officers is ordinarily by a court.
\item
  This scriptural doctrine of Presbytery is necessary to the perfection of the order of the visible Church, but is not essential to its existence.
\end{enumerate}

\hypertarget{the-king-and-head-of-the-church}{%
\section*{2. The King and Head of the Church}\label{the-king-and-head-of-the-church}}
\addcontentsline{toc}{section}{2. The King and Head of the Church}

\protect\hypertarget{chapter-slug-2-the-king-and-head-of-the-church}{\href{}{}}

\begin{enumerate}
\def\labelenumi{\arabic{enumi}.}
\tightlist
\item
  \protect\hypertarget{2}{\href{}{}}Jesus Christ, upon whose shoulders the government is, whose name is called Wonderful, Counsellor, the Mighty God, the Everlasting Father, the Prince of Peace; of the increase of whose government and peace there shall be no end; who sits upon the throne of David, and upon His kingdom to order it and to establish it with judgment and with justice from henceforth, even for ever; having all power given unto him in heaven and in earth by the Father, who raised him from the dead, and set him on His own right hand, far above all principality and power, and might, and dominion, and every name that is named, not only in this world, but also in that which is to come, and hath put all things under His feet, and gave him to be the Head over all things to the Church, which is His body, the fullness of him that filleth all in all; He, being ascended up far above all heavens, that He might fill all things, received gifts for His Church, and gave all officers necessary for the edification of His Church and the perfecting of His saints.
\item
  Jesus, the Mediator, the sole Priest, Prophet, King, Savior, and Head of the Church, contains in himself, by way of eminency, all the offices in His Church, and has many of their names attributed to him in the Scriptures. He is Apostle, Teacher, Pastor, Minister, and Bishop, and the only Lawgiver in Zion. It belongs to His Majesty from His throne of glory, to rule and teach the Church, through His Word and Spirit, by the ministry of men; thus mediately exercising His own authority, and enforcing His own laws, unto the edification and establishment of His kingdom.
\item
  Christ, as King, has given to His Church, officers, oracles and ordinances; and especially has He ordained therein His system of doctrine, government, discipline, and worship; all which are either expressly set down in Scripture, or by good and necessary consequence may be deduced therefrom; and to which things He commands that nothing be added, and that from them naught be taken away.
\item
  Since the ascension of Jesus Christ to heaven, He is present with the Church by His Word and Spirit, and the benefits of all His offices are effectually applied by the Holy Spirit.
\end{enumerate}

\hypertarget{the-visible-church-defined}{%
\section*{3. The Visible Church Defined}\label{the-visible-church-defined}}
\addcontentsline{toc}{section}{3. The Visible Church Defined}

\protect\hypertarget{chapter-slug-3-the-visible-church-defined}{\href{}{}}

\begin{enumerate}
\def\labelenumi{\arabic{enumi}.}
\tightlist
\item
  \protect\hypertarget{3}{\href{}{}}The Visible Church before the law, under the law, and now under the Gospel, is one and the same, and consists of all those who make profession of the true religion, together with their children.
\item
  \protect\hypertarget{3.2}{\href{}{}}This visible unity of the body of Christ, though obscured, is not destroyed by its division into different denominations of professing Christians; but all of these which maintain the Word and Sacraments in their fundamental integrity are to be recognized as true branches of the Church of Jesus Christ.
\item
  It is according to scriptural example that the Church should be divided into many individual churches.
\end{enumerate}

\hypertarget{the-nature-and-extent-of-church-power}{%
\section*{4. The Nature and Extent of Church Power}\label{the-nature-and-extent-of-church-power}}
\addcontentsline{toc}{section}{4. The Nature and Extent of Church Power}

\protect\hypertarget{chapter-slug-4-the-nature-and-extent-of-church-power}{\href{}{}}

\begin{enumerate}
\def\labelenumi{\arabic{enumi}.}
\tightlist
\item
  \protect\hypertarget{4}{\href{}{}}The power which Christ has committed to His Church vests in the whole body, the rulers and the ruled, constituting it a spiritual commonwealth. This power, as exercised by the people, extends to the choice of those officers whom He has appointed in His Church.
\item
  Ecclesiastical power, which is wholly spiritual, is twofold: the officers exercise it sometimes severally, as in preaching the Gospel, administering the sacraments, reproving the erring, visiting the sick, and comforting the afflicted, which is the power of order; and they exercise it sometimes jointly in Church courts, after the form of judgment, which is the power of jurisdiction.
\item
  The sole functions of the Church as a kingdom and government distinct from the civil commonwealth, are to proclaim, to administer, and to enforce the law of Christ revealed in the Scriptures.
\item
  The Church, with its ordinances, officers, and courts, is the agency which Christ has ordained for the edification and government of His people, for the propagation of the faith, and for the evangelization of the world. The power of the Church is exclusively spiritual; that of the State includes the exercise of force. The constitution of the Church derives from divine revelation; the constitution of the State will also be determined by human reason and the course of providential events. The Church has an obligation to speak prophetically to the civil magistrates, reminding them their authority is from God and they are to rule according to God's Law; the Church has no authority to construct or modify a government for the State, and the State has no authority to frame a creed or polity for the Church. ``Render unto Caesar the things that are Caesar's and to God the things that are God's.''\footnote{Matthew 22:21.}
\item
  The exercise of ecclesiastical power, whether joint or several, has the divine sanction, when in conformity with the statutes enacted by Christ, the Lawgiver, and when put forth by courts or by officers appointed thereunto in His Word.
\end{enumerate}

\hypertarget{the-particular-church}{%
\section*{5. The Particular Church}\label{the-particular-church}}
\addcontentsline{toc}{section}{5. The Particular Church}

\protect\hypertarget{chapter-slug-5-the-particular-church}{\href{}{}}

\begin{enumerate}
\def\labelenumi{\arabic{enumi}.}
\tightlist
\item
  \protect\hypertarget{5.1}{\href{}{}}A particular church consists of a number of professing Christians, with their children, associated together for divine worship and godly living, agreeably to the Scriptures, and submitting to the lawful government of Christ's Kingdom.
\item
  Its officers are the Pastor(s), the Ruling Elders, and the Deacons.
\item
  Its jurisdiction being a joint power, is lodged in the hands of the church Session, consisting of the Pastor(s) and Ruling Elders.
\item
  To the Deacons belong the collection and administration of the offerings of mercy and benevolence of the people for the relief of those in need under the supervision of the Session.
\item
  The ordinances established by Christ, the Head, in His Church, are prayer; singing praises; reading, expounding and preaching the Word of God; administering the sacraments of Baptism and the Lord's Supper; public solemn fasting and thanksgiving; catechising; making offerings for the relief of the poor and for other pious uses; exercising discipline; the taking of solemn vows; and the ordination to sacred office.
\item
  Churches destitute of the official ministration of the Word ought not to forsake the assembling of themselves together, but should be convened by the Session on the Lord's Day, and at other suitable times, for prayer, praise, the reading and expounding of the Holy Scriptures, and exhortation, or the reading of a sermon of some approved minister. In like manner, Christians whose lot is cast in destitute regions ought to meet regularly for the worship of God.
\end{enumerate}

\hypertarget{the-organization-of-a-particular-church}{%
\section*{6. The Organization of a Particular Church}\label{the-organization-of-a-particular-church}}
\addcontentsline{toc}{section}{6. The Organization of a Particular Church}

\protect\hypertarget{chapter-slug-6-the-organization-of-a-particular-church}{\href{}{}}

\begin{enumerate}
\def\labelenumi{\arabic{enumi}.}
\tightlist
\item
  \protect\hypertarget{6}{\href{}{}}A mission church may be properly described in the same manner as the particular church is described in BCO \protect\hyperlink{5.1}{5.1}. It is distinguished from a particular church in that it has no permanent governing body, and thus must be governed or supervised by others. However, its goal is to mature and be organized as a particular church as soon as this can be done decently and in good order.

  \begin{enumerate}
  \def\labelenumii{\alph{enumii}.}
  \tightlist
  \item
    Ordinarily, mission churches are established by Presbyteries within their boundaries.

    \begin{enumerate}
    \def\labelenumiii{\roman{enumiii}.}
    \tightlist
    \item
      Initiatives to which the Presbytery may respond in establishing a mission church include, but are not limited to, the following:

      \begin{enumerate}
      \def\labelenumiv{\arabic{enumiv}.}
      \tightlist
      \item
        The Presbytery establishes a mission church at its own initiative.
      \item
        The Presbytery responds to the initiative of a Session of a particular church.
      \item
        The Presbytery responds to the petition of an independent gathering of believers who have expressed their desire to become a congregation by submitting to the Presbytery a written request.
      \end{enumerate}
    \item
      In the event an existing non-Evangel Presbytery church is interested in coming into Evangel Presbytery, the Presbytery shall work with the church leadership to determine whether the church should come into Evangel Presbytery as a mission church or seek Presbytery approval to be received under the provisions of BCO \protect\hyperlink{15.7}{15.7}.
    \item
      Should it become necessary, the Presbytery may dissolve the mission church. Church members enrolled should be cared for according to the procedures of BCO \protect\hyperlink{15.9}{15.9}.
    \end{enumerate}
  \item
    The mission church, because of its transitional condition, requires a temporary system of government. Depending on the circumstances and at its own discretion, the Presbytery may provide for such government in one of several ways:

    \begin{enumerate}
    \def\labelenumiii{\roman{enumiii}.}
    \tightlist
    \item
      Appoint an Evangelist as prescribed in BCO \protect\hyperlink{9.6}{9.6}.
    \item
      Cooperate with the Session of a particular church in arranging a mother-daughter relationship with a mission church. The Session may then serve as the temporary governing body of the mission church.
    \item
      Appoint a BCO \protect\hyperlink{17.1}{17.1} commission to serve as a temporary Session of the mission church. When a minister of the Presbytery has been approved to serve as pastor of the mission church, he shall be included as a member of the commission and serve as its moderator. The temporary system of government shall record and submit its records to Presbytery for annual review in the same manner as Sessions of particular churches.
    \end{enumerate}
  \item
    \protect\hypertarget{6.1.c}{\href{}{}}Pastoral ministry for the mission church may be provided:

    \begin{enumerate}
    \def\labelenumiii{\roman{enumiii}.}
    \tightlist
    \item
      by a minister of the Presbytery called by Presbytery to serve as pastor, or
    \item
      by stated, student, or ruling elder supply (BCO \protect\hyperlink{24.5}{24.5-6}), or
    \item
      by a series of qualified preachers approved by the temporary government
    \end{enumerate}
  \item
    \protect\hypertarget{6.1.d}{\href{}{}}The temporary government shall receive members (BCO \protect\hyperlink{14.5}{14.5}) into the mission church according to the provisions of BCO \protect\hyperlink{53}{53} so far as they may be applicable. As members of the mission church those received are communing or non-communing members of Evangel Presbytery.

    \begin{enumerate}
    \def\labelenumiii{\roman{enumiii}.}
    \tightlist
    \item
      If there is a minister approved by Presbytery to serve the mission church as its pastor (BCO \protect\hyperlink{6.1.c}{6.1.c}), each member so received shall be understood to assent to the call of that minister and to affirm the promises made to the pastor in BCO \protect\hyperlink{23.10}{23.10}.
    \item
      Meetings of the members of the mission church shall be governed according to the provisions of BCO \protect\hyperlink{27}{27} so far as they may be applicable.
    \end{enumerate}
  \item
    Mission churches and their members shall have the right of judicial process to the court having oversight of their temporary governing body.
  \item
    Mission churches shall maintain a roll of communicant and non-communicant members, in the same manner as, but separate from, other particular churches.
  \item
    It is the intention of Evangel Presbytery that mission churches enjoy the same status as particular churches in relation to civil government.
  \end{enumerate}
\item
  A new particular church can be organized only by the authority of Presbytery.

  \begin{enumerate}
  \def\labelenumii{\alph{enumii}.}
  \tightlist
  \item
    The Presbytery should establish standing rules setting forth the prerequisites that qualify a mission church to begin the organization process, e.g., the minimum number of petitioners and the level of financial support to be provided by the congregation. The number of officers sufficient to constitute the quorum for a session shall be necessary to complete the organization process.
  \item
    The temporary government of the mission church shall oversee the steps necessary for organization.
  \item
    When among the members of the mission congregation to be organized there are men who appear qualified as officers, the temporary government shall examine these men and present a slate of candidates for election by the congregation.
  \item
    The election of officers shall normally take place at least two weeks prior to the date of the organization service. However, the effective date of service for the newly elected officers shall be upon the completion of the organization service.
  \item
    If deacons are not elected, the duties of the office shall devolve upon the session, until deacons can be secured.
  \item
    If there is a minister approved by Presbytery to serve the mission church as its pastor, and members of the mission church have been received according to BCO \protect\hyperlink{6.1.d}{6.1.d}, the temporary session shall call a congregational meeting at which the congregation may, by majority vote, call the organizing pastor to be their pastor without the steps of BCO \protect\hyperlink{22}{22}. If no such minister has been appointed, or the minister or congregation choose not to continue the pastoral relationship of the newly organized church, a pastor shall be called as follows:

    \begin{enumerate}
    \def\labelenumiii{\roman{enumiii}.}
    \tightlist
    \item
      The temporary government shall oversee the election of a pastor according the provisions of BCO \protect\hyperlink{22}{22} so far as they are applicable. If a candidate is to be proposed before the organization, the congregational meeting to elect a pastor shall take place early enough for Presbytery to consider and approve the pastor's call prior to the service of organization. This may be the same meeting called for the election of other officers.
    \item
      The ordination and/or installation shall be according to the provisions of BCO \protect\hyperlink{23}{23} so far as they are applicable. The service may take place at the service of organization.
    \end{enumerate}
  \item
    In order to proceed to organization as a particular church the members of the mission church shall sign a petition to Presbytery requesting the same.
  \item
    Upon Presbytery's approval of the petition, Presbytery shall appoint an organizing commission and shall set the date and time of the organization service.
  \item
    At the service of organization the following elements shall be included in the order deemed by the organizing commission to be appropriate:

    \begin{enumerate}
    \def\labelenumiii{\roman{enumiii}.}
    \tightlist
    \item
      The organizing commission shall ordain and/or install ruling elders and/or deacons according to the provisions of BCO \protect\hyperlink{26.5}{26.5} so far as they may be applicable.
    \item
      If a pastor is being ordained and/or installed at the service, the organizing commission shall act according to the provisions of BCO \protect\hyperlink{23}{23} so far as they may be applicable.
    \item
      A member of the organizing commission shall require communicant members of the mission church present to enter into covenant, by answering the following question affirmatively:
    \end{enumerate}

    \begin{quote}
    Do you, in reliance on God for strength, solemnly promise and covenant that you will walk together as a particular church, on the principles of the faith and order of Evangel Presbytery, and that you will be zealous and faithful in maintaining the purity and peace of the whole body?
    \end{quote}

    \begin{enumerate}
    \def\labelenumiii{\roman{enumiii}.}
    \setcounter{enumiii}{3}
    \tightlist
    \item
      A member of the organizing commission shall then say:
    \end{enumerate}

    \begin{quote}
    I now pronounce and declare that you are constituted a church according to the Word of God and the faith and order of Evangel Presbytery. In the name of the Father and of the Son and of the Holy Spirit. Amen.
    \end{quote}
  \end{enumerate}
\item
  Upon organization, the newly elected session should meet as soon as is practicable to elect a stated clerk and formulate a budget. If there is no pastor, the session may elect as moderator one of their own number or any teaching elder of the Presbytery with Presbytery's approval. Alternatively, the administrative committee has the power of committee to appoint a moderator for the new session, giving approval on behalf of Presbytery. Further, if there is no pastor, action shall be taken to secure, as soon as practicable, the regular administration of Word and Sacraments.
\end{enumerate}

\hypertarget{church-members}{%
\section*{7. Church Members}\label{church-members}}
\addcontentsline{toc}{section}{7. Church Members}

\protect\hypertarget{chapter-slug-7-church-members}{\href{}{}}

\protect\hypertarget{7}{\href{}{}}The members of this visible Church catholic are all those persons in every nation who make profession of their faith in the Lord Jesus Christ and promise submission to His laws. Communing members are those who have made a profession of faith in Christ, have been baptized, and have been admitted by the Session to the Lord's Table. The children of communing members are, through the covenant and by right of birth, non-communing members of the church. The children of communing members holding credo-baptistic convictions are non-communing members in the sense that they are entitled to the church's care, love, discipline, and training, with a view to their embracing Christ and thus possessing personally all the benefits of the covenant. The children of communing members holding paedo-baptistic convictions are non-communing members in the sense that they are entitled to baptism and to the church's care, love, discipline, and training, with a view to their embracing Christ and thus possessing personally all the benefits of the covenant.

\hypertarget{church-officersgeneral-classification}{%
\section*{8. Church Officers---General Classification}\label{church-officersgeneral-classification}}
\addcontentsline{toc}{section}{8. Church Officers---General Classification}

\protect\hypertarget{chapter-slug-8-church-officers-general-classification}{\href{}{}}

\begin{enumerate}
\def\labelenumi{\arabic{enumi}.}
\tightlist
\item
  \protect\hypertarget{8}{\href{}{}}Under the New Testament, our Lord at first collected His people out of different nations, and united them to the household of faith by the mission of the apostles, endued with miraculous gifts. This office has ceased.
\item
  The whole polity of the Church consists in doctrine, government, and distribution. And the ordinary and perpetual officers in the Church are Pastors, or Ministers of the Word, who are commissioned to preach the Gospel and administer the sacraments; Ruling Elders, whose office is to have the government and spiritual oversight of the church; and Deacons, whose office is to collect and administer the people's offerings of mercy and benevolence and to minister to their physical and material needs. In accordance with the Scripture, these offices are open to men only.
\item
  No one who holds office in the Church ought to usurp authority therein, or receive any official titles of spiritual preeminence, except such as are employed in the Scriptures.
\end{enumerate}

\hypertarget{the-minister-of-the-word}{%
\section*{9. The Minister of the Word}\label{the-minister-of-the-word}}
\addcontentsline{toc}{section}{9. The Minister of the Word}

\protect\hypertarget{chapter-slug-9-the-minister-of-the-word}{\href{}{}}

\begin{enumerate}
\def\labelenumi{\arabic{enumi}.}
\tightlist
\item
  \protect\hypertarget{9}{\href{}{}}This office is the first in the Church, both for dignity and usefulness. The person who fills it has in Scripture different titles expressive of his various duties. As he has the oversight of the flock of Christ, he is termed Bishop. As he feeds them with spiritual food, he is termed Pastor. As he serves Christ in the Church, he is termed Minister. As it is his duty to be grave and prudent, and an example to the flock, and to govern well in the house and kingdom of Christ, he is termed Presbyter or Elder. As he is sent to declare the will of God to sinners, and to beseech them to be reconciled to God through Christ, he is termed Ambassador. As he bears the glad tidings of salvation to the ignorant and perishing, he is termed Evangelist. As he stands to proclaim the Gospel, he is termed Preacher. As he expounds the Word, and by sound doctrine both exhorts and convinces the gainsayer, he is termed Teacher. As he dispenses the manifold grace of God, and the ordinances instituted by Christ, he is termed Steward of the mysteries of God. These titles do not indicate different grades of office, but all describe one and the same officer.
\item
  He that fills this office should possess a competency of human learning, and be blameless in life, sound in the faith, and apt to teach; he should exhibit a sobriety and holiness of life becoming the Gospel; he should rule his own house well; and should have a good report of them that are without.
\item
  As the Lord has given different gifts to the Ministers of the Word, and has committed to them various works to execute, the Church is authorized to call and appoint them to labor as Pastors, Teachers, and Evangelists, and in such other works as may be needful to the Church, according to the gifts in which they excel.
\item
  When a Minister is called to labor as a Pastor, it belongs to his office to pray for and with his flock, as the mouth of the people unto God; to feed the flock, by reading, expounding, and preaching the Word; to direct the congregation in singing the praises of God; to administer the sacraments; to catechize the children and youth; to visit officially the people, devoting especial attention to the poor, the sick, the afflicted, and the dying; and with the other Elders, to exercise the joint power of government---and to that end, he shall be an ex officio member of every board and committee of the church.
\item
  When a Minister is appointed to be a teacher in a school of divinity, or to give instruction in the doctrines and duties of religion to youth assembled in a college or university, it appertains to his office to take a pastoral oversight of those committed to his charge, and to be diligent in sowing the seed of the Word and gathering the fruit thereof, as one who watches for souls.
\item
  \protect\hypertarget{9.6}{\href{}{}}When a Minister is appointed to the work of the Evangelist, he is commissioned to preach the Word and administer the sacraments in foreign countries, frontier settlements, or the destitute parts of the Church; and to him may be entrusted power to organize churches, and ordain Ruling Elders and Deacons therein. He is commissioned for a renewable term of twelve months to preach the Word, to administer the Sacraments, to receive and dismiss members of mission churches, and to train potential officers.
\item
  When a Minister is called to labor through the press, in an educational setting, or in any other similar necessary work, it shall be incumbent on him to make full proof of his ministry by disseminating the Gospel for the edification of the Church. He shall give a report of his work to the Presbytery annually.
\item
  The Presbytery may, at its discretion, approve the call of a Minister to work with an organization outside the jurisdiction of Evangel Presbytery, provided that he be engaged in preaching and teaching the Word, that the Presbytery be assured he will have full freedom to maintain and teach the doctrine of our Church, and that he submit a written report on his work to the Presbytery at least annually.
\end{enumerate}

\hypertarget{the-ruling-elder}{%
\section*{10. The Ruling Elder}\label{the-ruling-elder}}
\addcontentsline{toc}{section}{10. The Ruling Elder}

\protect\hypertarget{chapter-slug-10-the-ruling-elder}{\href{}{}}

\begin{enumerate}
\def\labelenumi{\arabic{enumi}.}
\tightlist
\item
  \protect\hypertarget{10}{\href{}{}}As there were in the Church, under the law, Elders of the people for the government thereof, so in the Gospel Church, Christ has furnished others besides the Ministers of the Word with gifts and commission to govern when called thereunto, which officers are entitled Ruling Elders.
\item
  These Ruling Elders possess the same authority and eligibility to office in the courts of the Church as the Ministers of the Word. They should, moreover, cultivate zealously their aptness to teach the Bible and should improve every opportunity of doing so, to the end that destitute places, mission points, and churches without Pastors may be supplied with religious services.
\item
  Those who fill this office ought to be blameless in life and sound in the faith; they should be men of wisdom and discretion; and by the holiness of their walk and conversation should be examples to the flock.
\item
  Ruling Elders, the immediate representatives of the people, are chosen by them, that, in conjunction with the Pastors or Ministers, they may exercise government and discipline, including determining those qualified to receive the sacraments and assisting in the administration thereof, and take the oversight of the spiritual interests of the particular church, and also of the Church generally, when called thereunto. It appertains to their office, both severally and jointly, to watch diligently over the flock committed to their charge, that no corruption of doctrine or of morals enter therein. Evils which they cannot correct by private admonition they should bring to the notice of the Session. They should visit the people at their homes, especially the sick; they should instruct the ignorant, comfort the mourner, nourish and guard the children of the Church; and all those duties which private Christians are bound to discharge by the law of love are especially incumbent upon them by divine vocation, and are to be discharged as official duties; they should pray with and for the people; they should be careful and diligent in seeking the fruit of the preached Word among the flock; and should inform the Pastor of cases of sickness, affliction, and awakening, and of all others which may need his special attention. As overseers, Ruling Elders shall also appoint a Treasurer responsible for financial matters of the church and safekeeping of church funds, who shall report to the Session when requested.
\end{enumerate}

\hypertarget{the-deacon}{%
\section*{11. The Deacon}\label{the-deacon}}
\addcontentsline{toc}{section}{11. The Deacon}

\protect\hypertarget{chapter-slug-11-the-deacon}{\href{}{}}

\begin{enumerate}
\def\labelenumi{\arabic{enumi}.}
\tightlist
\item
  \protect\hypertarget{11}{\href{}{}}The office of Deacon is set forth in the Scriptures as ordinary and perpetual in the Church. The office is one of sympathy and service, after the example of the Lord Jesus; it expresses also the communion of saints, especially in their helping one another in time of need.
\item
  It is the duty of the Deacons to minister to those who are in need, to the sick, to the friendless, and to any who may be in distress. It is their duty also to develop the grace of liberality in the members of the church, to devise effective methods of collecting the gifts of the people, and to distribute these gifts among the objects to which they are contributed. They shall keep in proper repair the church edifice and other buildings and all physical property belonging to the congregation. In the discharge of their duties the Deacons are under the supervision and authority of the Session. In a church in which it is impossible for any reason to secure Deacons, the duties of the office shall devolve upon the Ruling Elders.
\item
  To the office of Deacon, which is spiritual in nature, should be chosen men of spiritual character, honest repute, exemplary lives, brotherly spirit, warm sympathies, and sound judgment.
\item
  The Deacons of a particular church shall be organized as a Board, of which the Pastor, or an Elder or Pastor of his choosing, shall be an ex officio member. The Board of Deacons shall elect a Chairman to be presented for the approval of the Session. The Board of Deacons shall elect their Secretary. The Board shall meet at least once a quarter, and whenever requested by the Session. The Board of each church shall determine the number necessary for a quorum, subject to the Session's approval. The Board shall keep a record of its proceedings, and of all funds and their distribution, and shall submit its minutes to the Session once a year, and at other times upon request of the Session. It is desirable that the Session and the Board of Deacons meet in joint session once a year to confer on matters of common interest.
\item
  Deacons may properly be appointed by the higher courts to serve on committees, especially as treasurers. It may also be helpful for the Church courts, when devising plans of church finance, to invite wise and consecrated Deacons to their counsels.
\end{enumerate}

\hypertarget{church-courtsin-general}{%
\section*{12. Church Courts---In General}\label{church-courtsin-general}}
\addcontentsline{toc}{section}{12. Church Courts---In General}

\protect\hypertarget{chapter-slug-12-church-courts-in-general}{\href{}{}}

\begin{enumerate}
\def\labelenumi{\arabic{enumi}.}
\item
  \protect\hypertarget{12}{\href{}{}}The Church is governed by various courts, in regular gradation; which are all, nevertheless, Presbyteries, as being composed exclusively of Presbyters.
\item
  These courts are church Sessions and Presbyteries.
\item
  The Pastor is, for prudential reasons, Moderator of the Session. The Moderator of the Presbytery shall be chosen at each stated meeting or for a period of time up to one year; and the Moderator, or, in case of his absence, the last Moderator present, or the oldest Minister in attendance, shall open the next meeting with a sermon, unless it be highly inconvenient, and shall hold the chair until a new Moderator be chosen. Provided, however, that when the Moderator of one of the higher courts is a Ruling Elder, the preaching of the opening sermon, or any other official duty, the performance of which requires the exercise of functions pertaining only to the Teaching Elder, shall be remitted by him for execution to such Minister of the Word, being a member of the court, as he may select.

  The Moderator has all authority necessary for the preservation of order and for the proper and expeditious conduct of all business before the court, and for convening and adjourning the court according to its own ruling. In any extraordinary emergency, he may, by an appropriate form of communication, change the time or place, or both, of meetings to which the court stands adjourned, giving reasonable notice thereof.
\item
  It is the duty of the Clerk, besides recording the transactions, to preserve the records carefully, and to grant extracts from them whenever properly required. Such extracts, under the hand of the Clerk, shall be evidence to any ecclesiastical court, and to every part of the Church.
\item
  Every meeting of the Presbytery and Session shall be opened and closed with prayer, and in closing the final session a psalm or hymn may be sung and the benediction pronounced.
\item
  The expenses of Ministers and Ruling Elders in their attendance on the courts shall be defrayed by the bodies which they respectively represent.
\end{enumerate}

\hypertarget{jurisdiction-of-church-courts}{%
\section*{13. Jurisdiction of Church Courts}\label{jurisdiction-of-church-courts}}
\addcontentsline{toc}{section}{13. Jurisdiction of Church Courts}

\protect\hypertarget{chapter-slug-13-jurisdiction-of-church-courts}{\href{}{}}

\begin{enumerate}
\def\labelenumi{\arabic{enumi}.}
\tightlist
\item
  \protect\hypertarget{13}{\href{}{}}These assemblies are altogether distinct from the civil magistrate, nor have they any jurisdiction in political or civil affairs. They have no power to inflict temporal pains and penalties, but their authority is in all respects moral or spiritual.
\item
  The jurisdiction of Church courts is only ministerial and declarative, and relates to the doctrines and precepts of Christ, to the order of the Church, and to the exercise of discipline. \emph{First}, They can make no laws binding the conscience; but may frame doctrinal standards, bear testimony against error in doctrine and immorality in practice, within or without the pale of the Church, and decide cases of conscience. \emph{Secondly}, They have power to establish rules for the government, discipline, worship, and extension of the Church, which must be agreeable to the doctrines relating thereto contained in the Scriptures, the circumstantial details only of these matters being left to the Christian prudence and wisdom of church officers and courts. \emph{Thirdly}, They possess the right of requiring obedience to the laws of Christ. Hence, they admit those qualified to sealing ordinances and to their respective offices, and they exclude the disobedient and disorderly from their offices or from sacramental privileges; but the highest censure to which their authority extends is to cut off the contumacious and impenitent from the congregation of believers. \emph{Moreover}, they possess all the administrative authority necessary to give effect to the powers.
\item
  All Church courts are one in nature, constituted of the same elements, possessed inherently of the same kinds of rights and powers, and differing only as the Constitution may provide. Yet it is according to scriptural example, and needful to the purity and harmony of the whole Church, that disputed matters of doctrine and order, arising in the lower courts, should be referred to the higher courts for decision.
\item
  For the orderly and efficient dispatch of ecclesiastical business, it is necessary that the sphere of action of each court should be distinctly defined. The Session exercises jurisdiction over a single church; the Presbytery over what is common to the Ministers, Sessions, and churches within a prescribed district. The jurisdiction of these courts is limited by the express provisions of the Constitution. Every court has the right to resolve questions of doctrine and discipline seriously and reasonably proposed, and in general to maintain truth and righteousness, condemning erroneous opinions and practices which tend to the injury of the peace, purity, or progress of the Church; and although each court exercises exclusive original jurisdiction over all matters specially belonging to it, the lower courts are subject to the review and control of the higher courts, in regular gradation. Hence, these courts are not separate and independent tribunals; but they have a mutual relation, and every act of jurisdiction is the act of the whole Church performed by it through the appropriate organ.
\end{enumerate}

\hypertarget{the-church-session}{%
\section*{14. The Church Session}\label{the-church-session}}
\addcontentsline{toc}{section}{14. The Church Session}

\protect\hypertarget{chapter-slug-14-the-church-session}{\href{}{}}

\begin{enumerate}
\def\labelenumi{\arabic{enumi}.}
\tightlist
\item
  \protect\hypertarget{14}{\href{}{}}The church Session consists of the Pastor and Associate Pastors, if there be any, and the Ruling Elders of a church. If there are three or more Ruling Elders, the Pastor and two Ruling Elders shall constitute a quorum. If there are fewer than three Ruling Elders, the Pastor and one Ruling Elder shall constitute a quorum. When a church has no Pastor, the session shall ask the Presbytery to appoint a pastor to serve as moderator of session. If there are five or more Ruling Elders, three, along with the appointed moderator, shall constitute a quorum; if there are fewer than five Ruling Elders, two, along with the appointed moderator, shall constitute a quorum; if there is only one Ruling Elder, he does not constitute a Session, but he should take spiritual oversight of the church, should represent it at Presbytery, should grant letters of dismissal, and should report to the Presbytery any matter needing the action of a Church court. However, any Session, by a majority vote of its members, may fix its own quorum, provided that it is not smaller than the quorum stated in this paragraph.
\item
  The Pastor is, by virtue of his office, the Moderator of the Session. In his absence, if an emergency should arise requiring immediate action, the Session may elect one of its members to preside. In any case where such meeting is required, the pastor must be informed before the meeting is held. Should prudential reasons at any time make it advisable for a Minister other than the Pastor to preside, the Pastor may, with the concurrence of the Session, invite a Minister of the same Presbytery to perform this service.
\item
  When a church is without a Pastor, the Moderator of the Session shall be either a Minister appointed for that purpose by the Presbytery, or one invited by the Session to preside on a particular occasion. When it is inconvenient to procure such a Moderator, the Session may elect one of its own members to preside. In judicial cases, the Moderator shall be a Minister of the Presbytery to which the church belongs.
\item
  \protect\hypertarget{14.4}{\href{}{}}In churches where there are two or more Pastors, the Senior Pastor shall preside. An Associate Pastor or Assistant Pastor may substitute for the Senior Pastor as moderator of the Session at the discretion of the Senior Pastor and Session.
\item
  \protect\hypertarget{14.5}{\href{}{}}The church Session is charged with maintaining the spiritual government of the church, for which purpose it has power to inquire into the knowledge, principles, and Christian conduct of the church members under its care; to censure those found delinquent; to see that parents do not neglect to bring up their children in the nurture and admonition of the Lord; to receive members into the communion of the church; to grant letters of dismissal to other churches, which, when given to parents, shall always include the names of their minor children and whether they have been baptized and whether they have been admitted to the Lord's Table; to examine, ordain, and install Ruling Elders and Deacons on their election by the church, and to require these officers to devote themselves to their work; to examine the records of the proceedings of the Deacons; to oversee, if established, Sunday Schools and all other ministries and discipleship; to order collections for pious uses; to take the oversight of the singing in the public worship of God; to assemble the people for worship when there is no Minister; to admit persons qualified to receive the sacraments and exercise authority concerning the time, place, frequency, and elements of the sacraments; to concert the best measures for promoting the spiritual interests of the church and congregation; to observe and carry out the lawful injunctions of the higher courts; and to appoint representatives to the Presbytery, who shall, on their return, make report of their diligence.
\item
  The Session shall hold stated meetings at least quarterly. Moreover, the Pastor has power to convene the Session when he may judge it requisite; and he shall always convene it when requested to do so by any two of the Ruling Elders; and when there is no Pastor, it may be convened by two Ruling Elders. The Session shall also convene when directed so to do by the Presbytery.
\item
  Every Session shall keep a fair record of its proceedings, which record shall be at least once in every year submitted to the inspection of the Presbytery.
\item
  Every Session shall keep a fair record of baptisms, of those admitted to the Lord's table, of non-communing members, and of the births, marriages, deaths, and dismissions of church members and of any baptisms and marriage ceremonies presided over by the pastors of the church outside its bounds.
\item
  Meetings of the Session shall be opened and closed with prayer.
\end{enumerate}

\hypertarget{the-presbytery}{%
\section*{15. The Presbytery}\label{the-presbytery}}
\addcontentsline{toc}{section}{15. The Presbytery}

\protect\hypertarget{chapter-slug-15-the-presbytery}{\href{}{}}

\begin{enumerate}
\def\labelenumi{\arabic{enumi}.}
\item
  \protect\hypertarget{15}{\href{}{}}\protect\hypertarget{15.1}{\href{}{}}The Presbytery consists of all the Ministers and churches within its bounds that have been accepted by the Presbytery. All Ministers and Ruling Elders in good standing of Evangel Presbytery shall have floor privileges.

  In presbytery votes, in order to promote parity between Ministers and Ruling Elders and to prevent centralization of power in large churches, voting privileges for members of presbytery shall be capped at two Ministers per church. Up to two Ruling Elders from each church may be appointed by their sessions as voting commissioners. A voting Ruling Elder shall be added to each church's allowed number of voting commissioners for each full one hundred communicant members over fifty members, up to four additional voting Ruling Elders.

  The Senior Pastor of each church, if present, shall be a voting member of Presbytery. All other voting members shall be chosen by each church's session. All Ministers and Ruling Elders may serve as voting members of presbytery committees and commissions.
\item
  \protect\hypertarget{15.2}{\href{}{}}A Minister shall be required to hold his membership in the Presbytery within whose bounds he resides, unless there are reasons which are satisfactory to his Presbytery why he should not do so. When a minister labors outside the geographical bounds of, or in a work not under the jurisdiction of his Presbytery, at home or abroad, it shall be only with the full concurrence of and under circumstances agreeable to his Presbytery, and to the Presbytery within whose geographical bounds he labors, if one exists. When a minister shall continue on the rolls of his Presbytery without a call to a particular work for a prolonged period, not exceeding three years, the procedure as set forth in BCO \protect\hyperlink{37.10}{37.10} shall be followed. A minister without call shall make or file a report to his Presbytery at least once each year.
\item
  Every Ruling Elder not known to the Presbytery shall produce a certificate of his regular appointment from the Session of the church which he represents.
\item
  Any three Ministers belonging to the Presbytery, together with at least three Ruling Elders, representing at least three churches, being met at the time and place appointed, shall be a quorum competent to proceed to business. However, any Presbytery, by a majority vote of those present at a stated meeting, may fix its own quorum, provided it is not smaller than the quorum stated in this paragraph.
\item
  Ministers seeking admission to a Presbytery shall be examined on Christian experience, and also concerning their views in theology, sacraments, and church government. If applicants come from other denominations, the Presbytery shall examine them thoroughly in knowledge and views as required by BCO \protect\hyperlink{23.4}{23.4} and require them to answer in the affirmative the questions put to candidates at their ordination. Ordained ministers from other denominations being considered by Presbyteries for reception may come under the extraordinary provisions set forth in BCO \protect\hyperlink{23.4}{23.4}. Presbyteries shall also require ordained ministers coming from other denominations to state the specific instances in which they may differ with the Confession of Faith and Catechisms in any of their statements and/or propositions, which differences the court shall judge in accordance with BCO \protect\hyperlink{23.4.e}{23.4, sections e-f}. All Ministers shall be required to submit their exceptions to the Constitution of this church BCO \protect\hyperlink{29.1}{29.1} in writing to the Presbytery prior to their examination. No exceptions to the Book of Church Order are permitted; however, a Minister may believe a change to the Book of Church Order is necessary and seek that change through the amendment process established therein.
\item
  The Presbytery shall cause to be transcribed, in some convenient part of the book of records, the obligations required of Ministers at their ordination, which shall be subscribed by all admitted to membership, in the following or like form, namely: ``I, {[}Name{]}, do sincerely receive and subscribe to the above obligation as a just and true exhibition of my faith and principles, and do resolve and promise to exercise my ministry in conformity thereunto.''
\item
  \protect\hypertarget{15.7}{\href{}{}}The Presbytery, before receiving into its membership any church, shall designate a commission to meet with the church's ruling elders to make certain that the elders understand and can sincerely adopt the doctrines and polity of Evangel Presbytery as contained in its Constitution. In the presence of the commission, the ruling elders shall be required to answer affirmatively the questions required of officers at their ordination.
\item
  \protect\hypertarget{15.8}{\href{}{}}The Presbytery has power to receive and issue appeals, complaints, and references brought before it in an orderly manner, and in cases in which the Session cannot exercise its authority, shall have power to assume original jurisdiction; to receive under its care candidates for the ministry; to examine and license candidates for the holy ministry; to receive, dismiss, ordain, install, remove, and judge Ministers; to review the record of church Sessions, redress whatever they may have done contrary to order, and take effectual care that they observe the Constitution of the Church; to establish the pastoral relation, and to dissolve it at the request of one or both of the parties, or where the interests of religion imperatively demand it; to set apart Evangelists to their proper work; to require Ministers to devote themselves diligently to their sacred calling and to censure the delinquent; to see that the lawful injunctions of the higher courts are obeyed; to condemn erroneous opinions which injure the purity or peace of the Church; to visit churches for the purpose of inquiring into and redressing the evils that may have arisen in them; to unite or divide churches, at the request of the members thereof; to form and receive new churches; to take special oversight of churches without Pastors; to dissolve churches; to concert measures for the enlargement of the Church within its bounds; in general, to order whatever pertains to the spiritual welfare of the churches under its care. The Presbytery's power in no way diminishes the absolute right of a church to withdraw or secede from the Presbytery. In doing so, the church retains all right, title, and interest in the church's real and personal property.
\item
  \protect\hypertarget{15.9}{\href{}{}}The Presbytery shall keep a full and fair record of its proceedings; and all the important changes which may have taken place, such as the licensures, the ordinations, the receiving or dismissing of members, the removal of members by death, the union and the division of churches, and the formation of new ones.
\item
  The Presbytery shall meet at least twice a year on its own adjournment. When any emergency shall require a meeting earlier than the time to which the Presbytery stands adjourned, the Moderator shall, at the request, or with the concurrence, of two Ministers and two Ruling Elders of different churches, call a special meeting. Should the Moderator be for any reason unable to act, the Stated Clerk shall, under the same requirements, issue the call. If both Moderator and Stated Clerk are unable to act, any three Ministers and two Ruling Elders of different churches shall have power to call a meeting. Notice of the special meeting shall be sent not less than ten days in advance to each Minister and to the Session of every church without a Pastor. In the notice the purpose of the meeting shall be stated, and no business other than that named in the notice is to be transacted.
\item
  Ministers in good standing in other Presbyteries, or in any evangelical Church, being present at any meeting of Presbytery, may be invited to sit as visiting brethren. It is proper for the Moderator to introduce these brethren to the Presbytery. They shall be granted the privilege of the floor, but no voting privileges.
\end{enumerate}

\hypertarget{provisions-concerning-baptism}{%
\section*{16. Provisions Concerning Baptism}\label{provisions-concerning-baptism}}
\addcontentsline{toc}{section}{16. Provisions Concerning Baptism}

\protect\hypertarget{chapter-slug-16-provisions-concerning-baptism}{\href{}{}}

\begin{enumerate}
\def\labelenumi{\arabic{enumi}.}
\tightlist
\item
  \protect\hypertarget{16}{\href{}{}}Being diligent to preserve the unity of the Spirit in the bond of peace'' (Eph. 4:3), the members and churches of the Presbytery are committed to the formal communion of Christians who hold differing convictions concerning time and mode of baptism. The members and churches of the Presbytery agree to the following:

  \begin{enumerate}
  \def\labelenumii{\alph{enumii}.}
  \tightlist
  \item
    Both paedo-baptists and credo-baptists recognize the Westminster Standards (Confession of Faith and Larger and Shorter Catechisms) as an excellent summary of the system of doctrine taught in Scripture and believe submission to a single standard best promotes the peaceable government and unity of the Presbytery.
  \item
    Evangel Presbytery will permit no exceptions to the Westminster Standards' teaching that all who profess faith in Jesus Christ along with their children are members of the visible Church of Christ and all members of that visible Church are under obligation to love, honor and obey their God and proper subjects of the discipline of the Church.
  \item
    Exceptions to the Westminster Standards concerning baptism will be granted in the following areas, and such exceptions will be deemed ``not out of accord with any fundamental of our system of doctrine because the difference is neither hostile to the system nor strikes at the vitals of religion'' (BCO \protect\hyperlink{23.4.f}{23.4.f}):

    \begin{enumerate}
    \def\labelenumiii{\arabic{enumiii}.}
    \tightlist
    \item
      Time of baptism: Many God-fearing men and women are of the conviction that though children of believers are members of the visible Church, rightly treated as such by the church, the application of baptism is so tied to the credible profession of faith by the individual that it is properly applied subsequent to that profession.
    \item
      Mode of baptism: Many God-fearing men and women are also of the conviction that baptism is rightly administered by immersion only rather than by pouring or sprinkling.
    \end{enumerate}
  \end{enumerate}
\item
  Exceptions to the Westminster Standards' teaching on baptism may be granted both to churches and to pastors ordained by presbytery. Churches granted an exception may ordain officers whose beliefs are in accord with their exception.
\item
  A church may declare an exception to the Westminster Standards to permit both credo-baptist and paedo-baptist views and practice by officers and pastors. Ordinarily, a church's bylaws will indicate whether officers must be exclusively credo-baptist, exclusively paedo-baptist, or may be comprised of both.
\item
  Exceptions granted to the Westminster Standards' teaching on baptism are effective in perpetuity and may only be revoked by a process effecting a change to Evangel Presbytery's constitution.
\item
  Churches granted an exception to the Westminster Standards' teaching on baptism may reverse that exception through a duly-effected process of bylaw change. Churches which enter Evangel Presbytery without claiming an exception in the area of baptism may claim an exception after a duly-effected bylaw change and a majority vote approving the new exception by the presbytery.
\item
  Charity will be shown to those holding differing views on the time and mode of baptism in the following ways:

  \begin{enumerate}
  \def\labelenumii{\alph{enumii}.}
  \tightlist
  \item
    Churches holding exceptions on baptism will honor the baptisms of infants and children who were baptized in accordance with the Westminster Standards.
  \item
    Churches embracing the Westminster Standards' on time and mode of baptism will:

    \begin{enumerate}
    \def\labelenumiii{\arabic{enumiii}.}
    \tightlist
    \item
      deny that the ``great sin'' of ``contemning or neglecting'' baptism spoken of in WCF 28.5 refers to a principled credo-baptist's delay of baptism until after a credible profession of faith
    \item
      accept as non-communicant members dedicated and non-dedicated children of members holding credo-baptist views
    \end{enumerate}
  \end{enumerate}
\item
  The children of communing believers are non-communing members, though only in the strict and distinct sense as set forth in BCO \protect\hyperlink{7}{7}. A church's membership roll shall include the children of communing members in a separate list, identifying them as non-communing members. However, such membership shall not be understood to be contrary to the convictions of the church and parents concerning infant baptism. This unity is possible because the church and parents understand the child is a proper recipient of the church's care, love, discipline, and training, with a view to his embracing Christ and thus possessing personally all the benefits of the covenant.
\end{enumerate}

\hypertarget{ecclesiastical-commissions}{%
\section*{17. Ecclesiastical Commissions}\label{ecclesiastical-commissions}}
\addcontentsline{toc}{section}{17. Ecclesiastical Commissions}

\protect\hypertarget{chapter-slug-17-ecclesiastical-commissions}{\href{}{}}

\begin{enumerate}
\def\labelenumi{\arabic{enumi}.}
\tightlist
\item
  \protect\hypertarget{17.1}{\href{}{}}A Commission differs from an ordinary committee in this, that while a committee is appointed to examine, consider and report, a Commission is authorized to deliberate upon and conclude the business referred to it. It shall keep a full record of its proceedings, which shall be submitted to the court appointing it, entered on its minutes, and regarded and treated as the action of the court.
\item
  Among the matters that may be properly executed by Commissions are the taking of testimony in judicial cases, the ordination of Ministers, the installation of Ministers, the visitation of portions of the Church affected with disorder, and the organization of new churches.Every Commission appointed by Presbytery shall consist of at least two Ministers and two Ruling Elders, and the Presbytery at the time of the appointment of the Commission shall determine what the quorum shall be. However, should a Presbytery clothe a Commission with judicial powers and authority to conduct judicial process, or with power to ordain a Minister of the Gospel, the quorum of such Commission shall not be less than two Ministers and two Ruling Elders. When the ordination of a Minister is committed to a Commission, the Presbytery itself shall conduct the previous examination.
\item
  The Presbytery may, of its own motion, commit any judicial case coming before it by appeal or complaint to a Commission, and should ordinarily follow this procedure, especially when requested by one or both parties to the case. Such a Commission shall be appointed by the court from its members other than members of the court from which the case comes up. Two-thirds of the Commissioners shall be a quorum to attend to business. The Commission shall try the case in the manner prescribed by the Rules of Discipline; shall submit to the court a full statement of the case and the judgment rendered, all of which shall be entered on the minutes of the court and accepted as its action and judgment in the case.
\item
  The Presbytery shall have power to commit the various interests pertaining to the general work of evangelization to one or more Commissions.
\end{enumerate}

\hypertarget{church-ordersthe-doctrine-of-vocation}{%
\section*{18. Church Orders---The Doctrine of Vocation}\label{church-ordersthe-doctrine-of-vocation}}
\addcontentsline{toc}{section}{18. Church Orders---The Doctrine of Vocation}

\protect\hypertarget{chapter-slug-18-church-orders-the-doctrine-of-vocation}{\href{}{}}

\begin{enumerate}
\def\labelenumi{\arabic{enumi}.}
\tightlist
\item
  \protect\hypertarget{18}{\href{}{}}Ordinary vocation to office in the Church is the calling of God by the Spirit, through the inward testimony of a good conscience, the manifest approbation of God's people, and the concurring judgment of a lawful court of the Church.
\item
  The government of the Church is representative, and the right of God's people to elect their officers is inalienable. Therefore no man can be placed over a church in any office without the election, or at least the consent of that church.
\item
  Upon those whom God calls to bear office in His Church He bestows suitable gifts for the discharge of their various duties. And it is indispensable that, besides possessing the necessary gifts and abilities, natural and acquired, every one admitted to an office should be sound in the faith, and his life be according to godliness. Wherefore every candidate for office is to be approved by the court by which he is to be ordained.
\end{enumerate}

\hypertarget{the-doctrine-of-ordination}{%
\section*{19. The Doctrine of Ordination}\label{the-doctrine-of-ordination}}
\addcontentsline{toc}{section}{19. The Doctrine of Ordination}

\protect\hypertarget{chapter-slug-19-the-doctrine-of-ordination}{\href{}{}}

\begin{enumerate}
\def\labelenumi{\arabic{enumi}.}
\tightlist
\item
  \protect\hypertarget{19}{\href{}{}}Those who have been called to office in the Church are to be inducted by the ordination of a court.
\item
  Ordination is the authoritative admission of one duly called to an office in the Church of God, accompanied with fasting, prayer, and the laying on of hands, to which it is proper to add the giving of the right hand of fellowship.
\item
  As every ecclesiastical office, according to the Scriptures, is a special charge, no man shall be ordained unless it be to the performance of a definite work. In other words, a man cannot be ordained without a call.
\end{enumerate}

\hypertarget{candidates-for-the-gospel-ministry}{%
\section*{20. Candidates for the Gospel Ministry}\label{candidates-for-the-gospel-ministry}}
\addcontentsline{toc}{section}{20. Candidates for the Gospel Ministry}

\protect\hypertarget{chapter-slug-20-candidates-for-the-gospel-ministry}{\href{}{}}

\begin{enumerate}
\def\labelenumi{\arabic{enumi}.}
\item
  \protect\hypertarget{20}{\href{}{}}A candidate for the ministry is a member of the Church in full communion who, believing himself to be called to preach the Gospel, submits himself to the care and guidance of the Presbytery in his course of study and of practical training to prepare himself for this office.
\item
  \protect\hypertarget{20.2}{\href{}{}}It is recommended that every candidate for the ministry should put himself under the care of a Presbytery, which should ordinarily be the Presbytery that has jurisdiction of the church of which he is a member. He should be encouraged by the Session to do this; and upon his request, the Session should furnish him with a certificate of his membership, and with testimonials of its judgment regarding his Christian character and promise of usefulness in the ministry, to be laid before the Presbytery. Every applicant for care shall be a member of the congregation whose session provides an endorsement for at least six months before filing his application, except in those cases deemed extraordinary by the Presbytery.
\item
  In making application to be taken under the care of the Presbytery, the candidate for the ministry, in addition to presenting testimonials from his church Session, shall be examined by the Presbytery in person on his Christian experience and on his motives for seeking the ministry.

  If the testimonials and the examination prove satisfactory, the Presbytery shall receive him under its care after the following manner:

  The Moderator shall propose to the candidate these questions:

  \begin{enumerate}
  \def\labelenumii{\alph{enumii}.}
  \item
    Do you promise in reliance upon the grace of God to maintain a becoming Christian character, and to be diligent and faithful in making full preparation for the sacred ministry?
  \item
    Do you promise to submit yourself to the proper supervision of the Presbytery in matters that concern your preparation for the ministry?
  \end{enumerate}

  If these questions be answered in the affirmative, the Moderator, or some one appointed by him, shall give the candidate a brief charge; and the proceedings shall close with prayer. The name of the candidate is then to be recorded on the Presbytery's Roll of Candidates for the Ministry.
\item
  The candidate continues to be a private member of the church and subject to the jurisdiction of the Session, but as respects his preparatory training for the ministry, he is under the oversight of the Presbytery. It shall be the duty of the Presbytery to show a kindly and sympathetic interest in him, and to give him counsel and guidance in regard to his studies, his practical training, and the institutions of learning he should attend. In no case may a candidate omit from his course of study any of the subjects prescribed in the Form of Government as tests for licensure and ordination without obtaining the consent of Presbytery; and where such consent is given the Presbytery shall record the fact and the reasons therefor.
\item
  For the development of his Christian character, for the service he can render, and for his more effective training, the candidate, when entering on his theological studies, should be authorized and encouraged by the Presbytery to conduct public worship, to expound the Scriptures to the people, and to engage in other forms of Christian work. These forms of service should be rendered under the direction of Presbytery, and also with the sanction and under the guidance of the candidate's instructors during the time of his being under their instruction. A candidate should not undertake to serve statedly a church which is without a pastor unless he has the approval of the Presbytery having jurisdiction of the church.
\item
  The Presbytery shall require every candidate for the ministry under its care to make a report to it at least once a year; and it shall secure from his instructors an annual report upon his deportment, diligence, and progress in study.
\item
  The Presbytery may, upon application of the candidate, give him a certificate of dismissal to another Presbytery; and a candidate may, at his request, be allowed to withdraw from the care of the Presbytery. The Presbytery may also, for sufficient reasons, remove the name of a candidate from its roll of candidates; but in such case it shall report its action and the reasons therefor to the candidate and to the Session of his church.
\item
  An applicant coming as a candidate from another denomination must present testimonials of his standing in that body and must become a member of a congregation in Evangel Presbytery. He shall then fulfill the requirements of applicants listed under BCO \protect\hyperlink{20.2}{20.2}, as well as requirements placed upon those desiring to be licensed or to become an intern as set forth in BCO \protect\hyperlink{21}{21}.
\end{enumerate}

\hypertarget{the-licensure-of-candidates-for-the-gospel-ministry}{%
\section*{21. The Licensure of Candidates for the Gospel Ministry}\label{the-licensure-of-candidates-for-the-gospel-ministry}}
\addcontentsline{toc}{section}{21. The Licensure of Candidates for the Gospel Ministry}

\protect\hypertarget{chapter-slug-21-the-licensure-of-candidates-for-the-gospel-ministry}{\href{}{}}

\begin{enumerate}
\def\labelenumi{\arabic{enumi}.}
\item
  \protect\hypertarget{21}{\href{}{}}\protect\hypertarget{21.1}{\href{}{}}To preserve the purity of the preaching of the Gospel, no man is permitted to preach in the pulpits of Evangel Presbytery on a regular basis without proper licensure from the Presbytery having jurisdiction where he will preach. This license shall immediately become void if the Presbytery administers against him a censure of suspension from office or the sacraments, or deposition from office, or of excommunication. A ruling elder, a candidate for the ministry, a minister from some other denomination, or some other man may be licensed for the purpose of regularly providing the preaching of the Word upon his giving satisfaction to the Presbytery of his gifts and passing the licensure examination. (See also BCO \protect\hyperlink{24.5}{24.5} and BCO \protect\hyperlink{24.6}{24.6}.)
\item
  \protect\hypertarget{21.2}{\href{}{}}The examination for licensure shall be as follows:

  \begin{enumerate}
  \def\labelenumii{\alph{enumii}.}
  \tightlist
  \item
    The candidate shall give a statement of his Christian experience and inward call to preach the Gospel in written form and/or orally before the Presbytery (at the discretion of the Presbytery).
  \item
    The candidate shall be tested with a written and/or oral examination by the Presbytery (at the discretion of the Presbytery) for his:

    \begin{enumerate}
    \def\labelenumiii{\roman{enumiii}.}
    \tightlist
    \item
      basic knowledge of biblical doctrine as outlined in the Confession of Faith and Larger and Shorter Catechisms.
    \item
      practical knowledge of Bible content.
    \item
      basic knowledge of the government of the church as defined in the Book of Church Order.
    \end{enumerate}
  \item
    The candidate shall be examined orally before Presbytery for his views in the areas outlined in part b above.
  \item
    The candidate shall provide his written sermon on an assigned passage of Scripture embodying both explanation and application, and present orally his sermon or exhortation before Presbytery or before a committee of Presbytery.
  \item
    While our Constitution does not require the applicant's affirmation of every statement and/or proposition of doctrine in our Confession of Faith and Catechisms, it is the right and responsibility of the Presbytery to determine if the applicant is out of accord with any of the fundamentals of these doctrinal standards and, as a consequence, may not be able in good faith sincerely to receive and adopt the Confession of Faith and Catechisms of this church as the system of doctrine taught in the Holy Scriptures.
  \item
    Therefore, in examining an applicant for licensure, the Presbytery shall inquire not only into the candidate's knowledge and views in the areas specified above, but also shall require the candidate to state the specific instances in which he may differ with the Confession of Faith and Catechisms in any of their statements and/or propositions. The court may grant an exception to any difference of doctrine only if in the court's judgment the applicant's declared difference is not out of accord with any fundamental of our system of doctrine because the difference is neither hostile to the system nor strikes at the vitals of religion.
  \end{enumerate}

  No Presbytery shall omit any of these parts of examination except in extraordinary cases; and whenever a Presbytery shall omit any of these parts, it shall always make a record of the reasons therefor, and of the trial parts omitted.
\item
  If the Presbytery be satisfied with the trials of the applicant, it shall then proceed to license him in the following manner. The moderator shall propose to him the following questions:

  \begin{enumerate}
  \def\labelenumii{\alph{enumii}.}
  \tightlist
  \item
    Do you believe the Scriptures of the Old and New Testaments, as originally given, to be the infallible Word of God, which is the only infallible rule of faith and practice?
  \item
    Do you sincerely receive and adopt the Confession of Faith and the Catechisms of this Church as the system of doctrine taught in the Holy Scripture?
  \item
    Do you promise to strive for the purity, peace, unity and edification of the Church?
  \item
    Do you promise to submit yourself, in the Lord, to the government of this Presbytery?
  \end{enumerate}
\item
  The applicant having answered these questions in the affirmative, the moderator shall offer a prayer suitable for the occasion, and shall address the applicant as follows:

  \begin{quote}
  In the name of the Lord Jesus Christ, and by that authority which He has given to the Church for its edification, we do license you to preach the Gospel in this Presbytery wherever God in His providence may call you; and for this purpose may the blessing of God rest upon you, and the Spirit of Christ fill your heart. Amen.
  \end{quote}

  Record shall be made of the licensure in the following or like form:

  \begin{quote}
  At \_\_\_\_\_\_\_\_\_\_\_\_\_, the \_\_\_\_\_\_\_\_ day of\_\_\_\_\_\_\_\_, the \_\_\_\_\_\_\_\_\_\_\_\_\_\_\_\_\_\_\_\_ Presbytery, having received testimonials commending \_\_\_\_\_\_\_\_\_\_\_\_\_\_\_\_\_\_\_\_, proceeded to submit him to the prescribed examination for licensure, which was met to the approval of the Presbytery. Having satisfactorily answered the questions for licensure, \_\_\_\_\_\_\_\_\_\_\_\_\_\_\_\_\_\_\_\_ was licensed by the Presbytery to preach the Gospel within the bounds of this Presbytery.
  \end{quote}
\item
  The license to preach the Gospel shall expire at the end of four years. The Presbytery may, if it thinks proper, renew it without further examination. The licentiate must apply for renewal prior to expiration. If the license expires, the stated clerk shall report the expiration to the Presbytery and to the individual's Session, and such action shall be recorded in the minutes. The procedures of BCO \protect\hyperlink{21.2}{21.2} must be followed for re-licensure and such fact shall be recorded in the minutes. The license may be terminated at any time by a simple majority vote of the Presbytery. The Presbytery shall always record its reasons for this action in its minutes.
\end{enumerate}

\hypertarget{the-election-of-pastors}{%
\section*{22. The Election of Pastors}\label{the-election-of-pastors}}
\addcontentsline{toc}{section}{22. The Election of Pastors}

\protect\hypertarget{chapter-slug-22-the-election-of-pastors}{\href{}{}}

\begin{enumerate}
\def\labelenumi{\arabic{enumi}.}
\item
  \protect\hypertarget{22}{\href{}{}}\protect\hypertarget{22.1}{\href{}{}}Before a candidate, or licentiate, can be ordained to the office of the ministry, he must receive a call to a definite work. Ordinarily the call must come from a church or the Presbytery. If the call comes from another source, the Presbytery shall always make a record of the reasons why it considers the work to be a valid Christian ministry.

  A proper call must be written and in the hands of the Presbytery prior to being acted upon by the Presbytery. It must include financial arrangements (such as salary, vacation, insurance, retirement, etc.) between those calling and the one called, and assurance that the definite work will afford the liberty to proclaim and practice fully and freely the whole counsel of God, as set forth in the Scriptures and understood in the Westminster Confession of Faith.
\item
  Every church should be under the pastoral oversight of a minister, and when a church has no Senior Pastor it should seek to secure one without delay.

  A church shall proceed to elect a senior pastor in the following manner: The Session shall appoint a pulpit committee which may be composed of male members from the congregation at large or the Session, or a mixture of Session members and male members at large. The pulpit committee shall, after consultation and deliberation, recommend to the session a pastoral candidate who, in its judgment, fulfills the Constitutional requirements of that office and is most suited to be profitable to the spiritual interests of the congregation.~The Session, once it receives the candidate recommended by the pulpit committee, shall order a congregational meeting to convene to vote on the presented candidate.

  The Session shall order a congregational meeting to convene to vote on the presented candidate according to the rules and procedures set forth in BCO \protect\hyperlink{27}{27}.
\item
  When a congregation is convened for the election of a pastor it is important that they should elect a minister or ruling elder of Evangel Presbytery to preside, but if this be not feasible, they may elect any male member of that church.
\item
  Method of voting: The voters being convened, and prayer for divine guidance having been offered, the moderator shall put the question:

  \begin{quote}
  ``Are you ready to proceed to the election of a pastor?''
  \end{quote}

  If they declare themselves ready, the moderator shall call for nominations, or the election may proceed by ballot without nominations. In every case a majority of all the votes cast shall be required to elect.
\item
  On the election of a pastor, if it appears that a large minority of the voters are averse to the candidate who has received a majority of votes, and cannot be induced to concur in the call, the moderator shall endeavor to dissuade the majority from prosecuting it further; but if the electors be nearly or quite unanimous, or if the majority shall insist upon their right to call a pastor, the moderator shall proceed to draft a call in due form, and to have it subscribed by them, certifying at the same time in writing the number of those who do not concur in the call, and any facts of importance, all of which proceedings shall be laid before the Presbytery, together with the call.
\item
  \protect\hypertarget{22.6}{\href{}{}}Form of call: The terms of the call shall be approved by the congregation in the following or like form:

  \begin{quote}
  The \_\_\_\_\_\_\_\_\_\_\_\_\_\_\_\_\_\_\_\_ Church being on sufficient grounds well satisfied of the ministerial qualifications of you, \_\_\_\_\_\_\_\_\_\_\_\_, and having good hopes from our knowledge of your labors that your ministrations in the Gospel will be profitable to our spiritual interests, do earnestly call you to undertake the pastoral office in said congregation, promising you, in the discharge of your duty, all proper support, encouragement and obedience in the Lord. That you may be free from worldly cares and avocations, we hereby promise and oblige ourselves to pay you the sum of \$\_\_\_\_\_\_\_\_\_\_\_ a year in regular payments, and other benefits, such as, housing allowance, retirement, insurance, vacations, moving expenses etc., during the time of your being and continuing the regular pastor of this church.

  In testimony whereof we have respectively subscribed our names this \_\_\_\_\_\_\_\_\_\_\_day of\_\_\_\_\_\_\_\_\_\_\_\_\_\_\_\_\_\_\_\_, A.D.\_\_\_\_\_\_\_\_.

  Attest: I, having moderated the congregational meeting which extended a call to \_\_\_\_\_\_\_\_\_\_\_\_\_\_ for his ministerial services, do certify that the call has been made in all respects according to the rules laid down in the Book of Church Order, and that the persons who signed the foregoing call were authorized to do so by vote of the congregation.

  \_\_\_\_\_\_\_\_\_\_\_\_\_\_\_\_\_\_\_\_\_\_\_\_\_\_\_\_\_\_\_\_\_\\
  Moderator of the Meeting
  \end{quote}
\item
  If any church shall choose to designate its ruling elders and deacons, or a committee to sign its call, it shall be at liberty to do so. But it shall, in such case, be fully certified to the Presbytery by the minister or other person who presided, that the persons signing have been appointed for that purpose by a public vote of the church, and that the call has been in all other respects prepared as above directed.
\item
  Prosecution of call: One or more commissioners shall be appointed by the church to present and prosecute the call before their Presbytery.
\item
  A congregation desiring to call a pastor from his charge, shall, by its commissioners to the Presbytery, prosecute the call before its Presbytery. The Presbytery, having heard all the parties, may, upon viewing the whole case, either recommend them to desist from prosecuting the call; or may order it to be delivered to the minister to whom it is addressed, with or without advice; or may decline to place the call in his hands; as it shall appear most beneficial for the peace, unity, and edification of the Church at large. No pastor shall be transferred without his own consent; if the parties are not ready to have the matter decided at the meeting then in progress, a written citation shall be given the minister and his church to appear before the Presbytery at its next meeting, which citation shall be read from the pulpit during a regular service, at least two weeks before the intended meeting.
\item
  A candidate or licentiate found fit and called for missionary service by a missionary agency or Presbytery shall be examined by Presbytery for ordination. If approved the Presbytery shall proceed to his ordination.
\item
  A missionary who is an ordained teaching elder in another denomination found fit and called for missionary service by a missionary agency or Presbytery shall be examined by Presbytery for admission to Presbytery. If approved he shall be enrolled as a member of Presbytery.
\end{enumerate}

\hypertarget{the-ordination-and-installation-of-ministers}{%
\section*{23. The Ordination and Installation of Ministers}\label{the-ordination-and-installation-of-ministers}}
\addcontentsline{toc}{section}{23. The Ordination and Installation of Ministers}

\protect\hypertarget{chapter-slug-23-the-ordination-and-installation-of-ministers}{\href{}{}}

\begin{enumerate}
\def\labelenumi{\arabic{enumi}.}
\item
  \protect\hypertarget{23}{\href{}{}}No Minister or licentiate or candidate shall receive a call from a church but by the permission of his Presbytery. When a call has been presented to the Presbytery, if found in order and the Presbytery deem it for the good of the Church, they shall place it in the hands of the person to whom it is addressed. Ordinarily a candidate or licentiate may not be granted permission by the Presbytery to move on to the field to which he has been called, prior to his examination for licensure or ordination. Likewise an ordained minister from another denomination, ordinarily shall not move on to the field to which he has been called until examined and received by Presbytery.
\item
  When a call for the pastoral services of a licentiate has been accepted by him, the Presbytery shall take immediate steps for his ordination.
\item
  No Presbytery shall ordain any licentiate or candidate to the office of the Gospel ministry, with reference to his laboring within the bounds of another Presbytery, but shall furnish him with the necessary testimonials, and require him to repair to the Presbytery within whose bounds he expects to labor, that he may submit himself to its authority, according to the Constitution of the Church.
\item
  \protect\hypertarget{23.4}{\href{}{}}\textbf{Ordination Requirements and Procedures}

  \begin{enumerate}
  \def\labelenumii{\alph{enumii}.}
  \item
    A candidate applying for ordination shall be required to present at least a diploma of Bachelor of Divinity from some approved pastoral training institution or authentic testimonials of having completed a regular course of theological studies. No Presbytery shall omit any of these educational requirements except in extraordinary cases, and then only with a three-fourths (¾) approval of the Presbytery. Whenever a Presbytery shall omit any of these educational requirements, it shall always make a record of the reasons for such omission and the parts omitted. The candidate shall also present satisfactory testimonials as to the completion and approval of an internship in the practice of the ministry.
  \item
    Every candidate for ordination shall ordinarily have met the requirements of the Presbytery's approved curriculum. Ordinarily, the candidate shall have been examined in most of the following trials when he was licensed. If the Presbytery previously approved all parts of the licensure examination, it need not re-examine the candidate in those areas at this time. If there were areas of weakness, which the Presbytery noted, or if any member of the Presbytery desires to do so, the candidate may be examined on particular points again. Additionally, the candidate shall be examined on any parts required for ordination which were not covered in his examination for licensure. In all cases, he should be asked to indicate whether he has changed his previous views concerning any points in the Confession of Faith, Catechisms, and Book of Church Order.
  \item
    A candidate shall undergo the following ordination trials:

    \begin{enumerate}
    \def\labelenumiii{\roman{enumiii}.}
    \tightlist
    \item
      He shall be given a careful examination in the following areas:

      \begin{enumerate}
      \def\labelenumiv{\arabic{enumiv}.}
      \item
        his Christian experience, especially his personal character and family management (based on the qualifications set out in 1 Timothy 3:1-7, and Titus 1:6-9),
      \item
        his knowledge of the biblical Greek and Hebrew languages,
      \item
        Bible content,
      \item
        theology,
      \item
        the Sacraments,
      \item
        Church history,
      \item
        the principles and rules of the government and discipline of the church.

        A Presbytery may accept a seminary degree which includes study in the original languages in lieu of an examination in the original languages.
      \end{enumerate}
    \item
      He shall prepare a thesis on some theological topic assigned by Presbytery.
    \item
      The candidate shall prepare an exegesis on an assigned portion of Scripture, requiring the use of the original language or languages.
    \item
      Upon a three-fourths (¾) vote of the Presbytery, he shall further be required to preach a sermon before the Presbytery or committee thereof.
    \end{enumerate}

    No Presbytery shall omit any of these parts of trial for ordination except in extraordinary cases, and then only with three-fourths (¾) approval of Presbytery.
  \item
    Whenever a Presbytery shall omit any of these parts, it shall always make a record of the reasons for such omissions and of the trial parts omitted.
  \item
    \protect\hypertarget{23.4.e}{\href{}{}}While our Constitution does not require the candidate's affirmation of every statement and/or proposition of doctrine in our Confession of Faith and Catechisms, it is the right and responsibility of the Presbytery to determine if the candidate is out of accord with any of the fundamentals of these doctrinal standards and, as a consequence, may not be able in good faith sincerely to receive and adopt the Confession of Faith and Catechisms of this Church as containing the system of doctrine taught in the Holy Scriptures.
  \item
    \protect\hypertarget{23.4.f}{\href{}{}}Therefore, in examining a candidate for ordination, the Presbytery shall inquire not only into the candidate's knowledge and views in the areas specified above, but also shall require the candidate to state the specific instances in which he may differ with the Confession of Faith and Catechisms in any of their statements and/or propositions. The court may grant an exception to any difference of doctrine only if in the court's judgment the candidate's declared difference is not out of accord with any fundamental of our system of doctrine because the difference is neither hostile to the system nor strikes at the vitals of religion.
  \item
    The Presbytery, being fully satisfied of his qualifications for the sacred office, shall appoint a day for his ordination, which ought, if practicable, to be in that church of which he is to be the pastor.
  \item
    The extraordinary clauses should be limited to extraordinary circumstances of the church or proven extraordinary gifts of the man. Presbyteries should exercise diligence and care in the use of these provisions in order that they not prevent the ordination of a candidate for whom there are truly exceptional circumstances, nor ordain (nor receive from other denominations) a person who is inadequately prepared for the ministry.
  \end{enumerate}
\item
  The day appointed for the ordination having come, and the Presbytery, or a commission thereof, being convened, a member of the Presbytery, previously appointed to that duty, shall preach a sermon adapted to the occasion. The same, or another member appointed to preside, shall afterwards briefly recite from the pulpit the proceedings of the Presbytery preparatory to the ordination; he shall point out the nature and importance of the ordinance, and endeavor to impress the audience with a proper sense of the solemnity of the transaction.
\item
  \textbf{Questions for Ordination}

  Then addressing himself to the candidate, he shall propose to him the following questions:

  \begin{enumerate}
  \def\labelenumii{\alph{enumii}.}
  \tightlist
  \item
    Do you believe the Scriptures of the Old and New Testaments, as originally given, to be the infallible Word of God, which is the only infallible rule of faith and practice?
  \item
    Do you sincerely receive and adopt the Confession of Faith and the Catechisms of this Church, as the system of doctrine taught in the Holy Scriptures; and do you further promise that if at any time you find yourself out of accord with any of the fundamentals of this system of doctrine, you will on your own initiative, make known to your Presbytery the change which has taken place in your views since the assumption of this ordination vow?
  \item
    \protect\hypertarget{23.6.c}{\href{}{}}Do you approve of the government and discipline of Evangel Presbytery, as being in conformity with the general principles of biblical polity?
  \item
    Do you promise subjection to your brethren in the Lord?
  \item
    Have you been induced, as far as you know your own heart, to seek the office of the holy ministry from love to God and a sincere desire to promote His glory in the Gospel of His Son?
  \item
    Do you promise to be zealous and faithful in maintaining the truths of the Gospel and the purity and peace and unity of the Church, whatever persecution or opposition may arise unto you on that account?
  \item
    Do you engage to be faithful and diligent in the exercise of all your duties as a Christian and a Minister of the Gospel, whether personal or relational, private or public; and to endeavor by the grace of God to adorn the profession of the Gospel in your manner of life, and to walk with exemplary piety before the flock of which God shall make you overseer?
  \item
    Are you now willing to take the charge of this church, agreeably to your declaration when accepting their call? And do you, relying upon God for strength, promise to discharge to it the duties of a Pastor?\footnote{For an assistant pastor, the following vow shall be used: 'Are you now willing to serve this church, agreeably to your declaration when accepting the call of the session? And do you, relying upon God for strength, promise to discharge to it the duties of an Assistant Pastor?}
  \end{enumerate}
\end{enumerate}

\begin{enumerate}
\def\labelenumi{\arabic{enumi}.}
\setcounter{enumi}{6}
\item
  \textbf{Questions to Congregation}

  The candidate having answered these questions in the affirmative, the presiding Minister shall propose to the church the following questions:

  \begin{enumerate}
  \def\labelenumii{\alph{enumii}.}
  \tightlist
  \item
    Do you, the people of this congregation, continue to profess your readiness to receive \_\_\_\_\_\_\_\_\_\_\_\_\_\_\_\_, whom you have called to be your Pastor?\footnote{For an assistant pastor, the following vow shall be used: 'Do you, the people of this congregation, profess your readiness to receive \_\_\_\_\_\_, whom the session has called to be your assistant pastor?}
  \item
    Do you promise to receive the word of truth from his mouth with meekness and love, and to submit to him in the due exercise of discipline?
  \item
    Do you promise to encourage him in his labors, and to assist his endeavors for your instruction and spiritual edification?
  \item
    And do you engage to continue to him while he is your Pastor that competent worldly maintenance which you have promised, and to furnish him with whatever you may see needful for the honor of religion and for his comfort among you?
  \end{enumerate}
\end{enumerate}

\begin{enumerate}
\def\labelenumi{\arabic{enumi}.}
\setcounter{enumi}{7}
\tightlist
\item
  The people having answered these questions in the affirmative, the candidate shall kneel, and the presiding Minister shall, with prayer and the laying on of the hands of the Presbytery, according to the apostolic example, solemnly set him apart to the holy office of the Gospel ministry. Prayer being ended, he shall rise from his knees; and the Minister who presides shall first, and afterwards all the members of the Presbytery in their order, take him by the right hand, saying, in words to this effect: ``We give you the right hand of fellowship, to take part in this ministry with us.'' The presiding Minister shall then say: ``I now pronounce and declare that {[}Name{]} has been regularly elected, ordained, and installed Pastor of this congregation, agreeably to the Word of God, and according to the Constitution of Evangel Presbytery; and that as such he is entitled to all support, encouragement, honor, and obedience in the Lord: In the name of the Father, and of the Son, and of the Holy Spirit. Amen.'' After which the Minister presiding, or some other Minister or Ruling Elder appointed for the purpose, shall give a solemn charge to the Pastor and to the congregation, to persevere in the discharge of their reciprocal duties, and then after prayer and the singing of a psalm, or hymn, the congregation shall be dismissed with the benediction. And the Presbytery shall duly record its proceedings.\footnote{For assistant pastor, add the word ``assistant'' before ``pastor.''}
\end{enumerate}

\begin{enumerate}
\def\labelenumi{\arabic{enumi}.}
\setcounter{enumi}{8}
\item
  After the installation, the heads of families of the congregation then present, or at least the Ruling Elders and Deacons, should come forward to their Pastor, and give him their right hand and the holy kiss of Christian fellowship, in token of cordial reception and affectionate regard.
\item
  \protect\hypertarget{23.10}{\href{}{}} \textbf{Questions for Installation}

  In the installation of an ordained Minister, the following questions are to be substituted for those addressed to a candidate for ordination, namely:

  \begin{enumerate}
  \def\labelenumii{\alph{enumii}.}
  \tightlist
  \item
    Are you now willing to take charge of this congregation as their Pastor, agreeably to your declaration in accepting its call?\footnote{For an assistant pastor, the following vow shall be used: `Are you now willing to serve this congregation as their Assistant Pastor, agreeably to your declaration in accepting the call of its session?'}
  \item
    Do you conscientiously believe and declare, as far as you know your own heart, that, in taking upon you this charge, you are influenced by a sincere desire to promote the glory of God and the good of the Church?
  \item
    Do you solemnly promise that, by the assistance of the grace of God, you will endeavor faithfully to discharge all the duties of a Pastor to this congregation, and will be careful to maintain a deportment in all respects becoming a Minister of the Gospel of Christ, agreeably to your ordination engagements?\footnote{For assistant pastor, add the word `assistant' before 'pastor.}
  \end{enumerate}
\end{enumerate}

\begin{enumerate}
\def\labelenumi{\arabic{enumi}.}
\setcounter{enumi}{10}
\item
  \textbf{Questions to the Congregation}

  The candidate having answered these questions in the affirmative, the presiding minister shall propose to the church the following questions:

  \begin{enumerate}
  \def\labelenumii{\alph{enumii}.}
  \tightlist
  \item
    Do you, the people of this congregation, continue to profess your readiness to receive\_\_\_\_\_\_, whom you have called to be your pastor?\footnote{For an assistant pastor, the following vow shall be used: 'Do you, the people of this congregation, profess your readiness to receive \_\_\_\_\_\_, whom the session has called to be your assistant pastor?}
  \item
    Do you promise to receive the word of truth from his mouth with meekness and love, and to submit to him in the due exercise of discipline?
  \item
    Do you promise to encourage him in his labors, and to assist his endeavors for your instruction and spiritual edification?
  \item
    Do you engage to continue to him while he is your pastor that competent worldly maintenance which you have promised, and to furnish him with whatever you may see needful for the honor of religion and for his comfort among you?
  \end{enumerate}
\end{enumerate}

\begin{enumerate}
\def\labelenumi{\arabic{enumi}.}
\setcounter{enumi}{11}
\item
  In the ordination of candidates or licentiates as Evangelists the same questions are to be propounded as in the ordination of Pastors, with the exception of the eighth, for which the following shall be substituted:

  \begin{quote}
  ``Do you now undertake the work of an Evangelist, and do you promise, in reliance on God for strength, to be faithful in the discharge of all the duties incumbent on you as a Minister of the Gospel of the Lord Jesus Christ?''
  \end{quote}
\end{enumerate}

\hypertarget{the-pastoral-relations}{%
\section*{24. The Pastoral Relations}\label{the-pastoral-relations}}
\addcontentsline{toc}{section}{24. The Pastoral Relations}

\protect\hypertarget{chapter-slug-24-the-pastoral-relations}{\href{}{}}

\begin{enumerate}
\def\labelenumi{\arabic{enumi}.}
\tightlist
\item
  \protect\hypertarget{24}{\href{}{}}The various pastoral relations are pastor, associate pastor, and assistant pastor.
\item
  The pastor and associate pastor are elected by the congregation using the form of call in BCO \protect\hyperlink{22.6}{22.6}. Being elected by the congregation, they become members of the Session.
\item
  \protect\hypertarget{24.3}{\href{}{}}An assistant pastor is called by the Session, by the permission and approval of Presbytery, under the provisions of BCO \protect\hyperlink{22.1}{22.1} and BCO \protect\hyperlink{15.2}{15.2}, with Presbytery membership being governed by the same provisions that apply to pastors. He is not a member of the Session, but may be appointed on special occasions to moderate the Session under the provisions of BCO \protect\hyperlink{14.4}{14.4}. A man may serve as an assistant pastor for a term not to exceed two years after which his term may be renewed by the session to continue as an assistant pastor or the congregation may vote to call him as an associate pastor.
\item
  The relationship of the associate pastor to the church is determined by the congregation. The relationship of the assistant pastor to the church is determined by the Session. The dissolution of an associate pastor is governed by the provisions of BCO \protect\hyperlink{25}{25}. The dissolution of the relationship of an assistant pastor is governed by the provisions of BCO \protect\hyperlink{24.3}{24.3} and BCO \protect\hyperlink{25}{25}.
\item
  \protect\hypertarget{24.5}{\href{}{}}In order to provide necessary changes in pastorates, a temporary relation may be established between a church and a minister called Stated Supply. If a church is unable to secure a regular pastor or a Stated Supply, then the Session with approval of Presbytery may establish a temporary relation between the church and a licentiate called Student Supply or Ruling Elder Supply.
\item
  \protect\hypertarget{24.6}{\href{}{}}Such temporary relationships can take place at the invitation of the church Session to the minister of the Word, the licentiate, or the ruling elder. The length of the relationship will be determined by the Session and the minister, the licentiate, or the ruling elder, with the approval of the Presbytery. Stated supply, student supply, or ruling elder supply relationships will be for no longer than one year, renewable at the request of the Session and at the review of the Presbytery. (See also BCO \protect\hyperlink{21.1}{21.1}).
\end{enumerate}

\hypertarget{the-dissolution-of-the-pastoral-relation}{%
\section*{25. The Dissolution of the Pastoral Relation}\label{the-dissolution-of-the-pastoral-relation}}
\addcontentsline{toc}{section}{25. The Dissolution of the Pastoral Relation}

\protect\hypertarget{chapter-slug-25-the-dissolution-of-the-pastoral-relation}{\href{}{}}

\begin{enumerate}
\def\labelenumi{\arabic{enumi}.}
\item
  \protect\hypertarget{25}{\href{}{}}When any Minister shall tender the resignation of his pastoral charge to his Presbytery, the Presbytery shall cite the church to appear by its commissioners, or the church may so appear upon its own motion, to show cause, if it has any, why the Presbytery should not accept the resignation. If the church fail to appear, or if its reasons for retaining its Pastor be deemed insufficient, his resignation shall be accepted, and the pastoral relation dissolved. If any church desires to be relieved of its Pastor, a similar procedure shall be observed. But whether the Minister or the church initiate proceedings for a dissolution of the relation, there shall always be a meeting of the congregation called and conducted precisely in the same manner as when the call of a Pastor is to be made out. The pastoral relation of the assistant pastor may be dissolved by the Session; the presbytery shall be informed of their action.

  Upon dissolution of the pastoral relationship of the senior pastor, the Presbytery must determine if the dissolution of the pastoral relationship of the senior pastor was brought about in Christian love and good order on the part of the parties concerned. The Presbytery must review the call of any new senior pastor and the continuation of existing pastoral relationships, which calls are subject to the Presbytery's approval by majority vote.
\item
  The Presbytery may designate a minister as honorably retired when the minister by reason of age wishes to be retired, or as medically disabled when by reason of infirmity is no longer able to serve the church in the active ministry of the Gospel. A minister medically disabled or honorably retired shall continue to hold membership in his Presbytery. He may serve on committees or commissions if so elected or appointed.
\item
  In order to provide necessary changes in pastorates, a temporary relation may be established between a church and a minister called Stated Supply. If a church is unable to secure a regular pastor or a Stated Supply, then the Session with approval of Presbytery may establish a temporary relation between the church and a licentiate called Student Supply or Ruling Elder Supply.
\item
  Such temporary relationships can take place at the invitation of the church Session to the minister of the Word, the licentiate, or the ruling elder. The length of the relationship will be determined by the Session and the minister, the licentiate, or the ruling elder, with the approval of the Presbytery. Stated supply, student supply, or ruling elder supply relationships will be for no longer than one year, renewable at the request of the Session and at the review of the Presbytery (cf.~BCO \protect\hyperlink{21.1}{21.1}).
\end{enumerate}

\hypertarget{election-ordination-and-installation-of-ruling-elders-and-deacons}{%
\section*{26. Election, Ordination, and Installation of Ruling Elders and Deacons}\label{election-ordination-and-installation-of-ruling-elders-and-deacons}}
\addcontentsline{toc}{section}{26. Election, Ordination, and Installation of Ruling Elders and Deacons}

\protect\hypertarget{chapter-slug-26-election-ordination-and-installation-of-ruling-elders-and-deacons}{\href{}{}}

\hypertarget{election}{%
\subsection*{\texorpdfstring{\protect\hypertarget{26}{\href{}{}} Election}{ Election}}\label{election}}
\addcontentsline{toc}{subsection}{\protect\hypertarget{26}{\href{}{}} Election}

\begin{enumerate}
\def\labelenumi{\arabic{enumi}.}
\item
  Every church shall elect men to the offices of ruling elder and deacon in the following manner: At such times as determined by the Session, voting members of the congregation may submit names to the Session, keeping in mind that each prospective officer should be an active male member who meets the qualifications set forth in 1 Timothy 3 and Titus 1. After the close of the nomination period those nominees for the office of ruling elder and/or deacon approved by the Session shall receive instruction in the qualifications and work of the office. Each nominee shall then be examined in:

  \begin{enumerate}
  \def\labelenumii{\alph{enumii}.}
  \tightlist
  \item
    his Christian experience, especially his personal character and family management (based on the qualifications set out in 1 Timothy 3:1-7 and Titus 1:6-9),
  \item
    his knowledge of Bible content,
  \item
    his knowledge of the system of doctrine, government, discipline contained in the Constitution of Evangel Presbytery BCO \protect\hyperlink{29.1}{29.1},
  \item
    the duties of the office to which he has been nominated, and
  \item
    his willingness to give assent to the questions required for ordination. BCO \protect\hyperlink{26.4}{26.4}
  \end{enumerate}

  If the nominee is married, his wife shall be interviewed regarding the above criteria.

  If there are candidates eligible for the election, the Session shall report to the congregation those eligible, giving at least thirty (30) days prior notice of the time and place of a congregational meeting for elections.

  If one-fourth (1/4) of the persons entitled to vote shall at any time request the Session to call a congregational meeting for the purpose of electing additional officers, it shall be the duty of the Session to call such a meeting on the above procedure. The number of officers to be elected shall be determined by the congregation after hearing the Session's recommendation.
\item
  A two-thirds majority of votes cast is required for election.
\item
  The voters being convened, the moderator shall explain the purpose of the meeting and then put the question:Are you now ready to proceed to the election of additional ruling elders (or deacons) from the slate presented?If they declare themselves ready, the election may proceed by private ballot without nomination. In every case a two-thirds majority of all the votes cast shall be required to elect.
\end{enumerate}

\hypertarget{ordination-and-installation}{%
\subsection*{Ordination and Installation}\label{ordination-and-installation}}
\addcontentsline{toc}{subsection}{Ordination and Installation}

\begin{enumerate}
\def\labelenumi{\arabic{enumi}.}
\setcounter{enumi}{3}
\item
  \protect\hypertarget{26.4}{\href{}{}}The day having arrived, and the Session being convened in the presence of the congregation, a sermon shall be preached after which the presiding minister shall state in a concise manner the warrant and nature of the office of ruling elder, or deacon, together with the character proper to be sustained and the duties to be fulfilled. Having done this, he shall propose to the candidate, in the presence of the church, the following questions, namely:

  \begin{enumerate}
  \def\labelenumii{\alph{enumii}.}
  \tightlist
  \item
    Do you believe the Scriptures of the Old and New Testaments, as originally given, to be the infallible Word of God, which is the only infallible rule of faith and practice?
  \item
    Do you sincerely receive and adopt the Confession of Faith and the Catechisms of this Church, as the system of doctrine taught in the Holy Scriptures; and do you further promise that if at any time you find yourself out of accord with any of the fundamentals of this system of doctrine, you will on your own initiative, make known to your Presbytery the change which has taken place in your views since the assumption of this ordination vow?
  \item
    Do you approve of the government and discipline of Evangel Presbytery, as being in conformity with the general principles of biblical polity?
  \item
    Do you accept the office of ruling elder (or deacon, as the case may be) in this church, and promise faithfully to perform all the duties thereof, and to endeavor by the grace of God to adorn the profession of the Gospel in your life, and to set a worthy example before the Church of which God has made you an officer?
  \item
    Do you promise subjection to your brethren in the Lord?
  \item
    Do you promise to strive for the purity, peace, unity and edification of the Church?
  \end{enumerate}

  The ruling elder or deacon elect having answered in the affirmative, the minister shall address to the members of the church the following question:

  \begin{quote}
  Do you, the members of this church, acknowledge and receive this brother as a ruling elder (or deacon), and do you promise to yield him all that honor, encouragement and obedience in the Lord to which his office, according to the Word of God and the Constitution of this Church, entitles him?
  \end{quote}

  The members of the church having answered this question in the affirmative, the candidate shall then be set apart, with prayer by the minister or any other Session member and the laying on of the hands of the Session, to the office of ruling elder (or deacon). Prayer being ended, the members of the Session (and the deacons, if the case be that of a deacon) shall greet the newly ordained officer with a holy kiss and take him by the hand, saying in words to this effect:

  \begin{quote}
  We give you the right hand of fellowship, to take part in this office with us.
  \end{quote}

  The minister shall then say:

  \begin{quote}
  I now pronounce and declare that \_\_\_\_\_\_\_\_\_\_\_\_\_\_\_\_\_\_\_\_ has been regularly elected, ordained and installed a ruling elder (or deacon) in this church, agreeable to the Word of God, and according to the Constitution of Evangel Presbytery; and that as such he is entitled to all encouragement, honor and obedience in the Lord: In the name of the Father, and of the Son, and of the Holy Spirit. Amen.
  \end{quote}

  After which the minister or any other member of the Session shall give to the ruling elder (or deacon) and to the church an exhortation suited to the occasion.
\item
  \protect\hypertarget{26.5}{\href{}{}}Ordination to the offices of ruling elder or deacon is perpetual; nor can such offices be laid aside at pleasure; nor can any person be degraded from either office but by deposition after regular trial; yet a ruling elder or deacon may have reasons which he deems valid for being released from the active duties of his office. In such a case the Session, after conference with him and careful consideration of the matter, may, if it thinks proper, accept his resignation and dissolve the official relationship which exists between him and the church.The ruling elder or deacon, though chargeable with neither heresy nor immorality, may become unacceptable in his official capacity to a majority of the church which he serves. In such a case the church may take the initiative by a majority vote at a regularly called congregational meeting, and request the Session to dissolve the official relationship between the church and the officer without censure. The Session, after conference with the ruling elder or deacon, and after careful consideration, may use its discretion as to dissolving the official relationship. In either case the Session shall report its action to the congregation. If the Session fails or refuses to report to the congregation within sixty (60) days from the date of the congregational meeting or if the Session reports to the congregation that it declined to dissolve such relationship, then any member or members in good standing may file a complaint against the Session in accordance with the provisions of BCO \protect\hyperlink{46}{46}.
\item
  When a ruling elder or deacon who has been released from his official relation is again elected to his office in the same or another church, he shall be installed after the above form with the omission of ordination.
\item
  When a ruling elder or deacon cannot or does not for a period of one year perform the duties of his office, his official relationship shall be dissolved by the Session and the action reported to the congregation.
\item
  When a deacon or ruling elder by reason of age or infirmity desires to be released from the active duties of the office, he may at his request and with the approval of the Session be designated deacon or elder emeritus. When so designated, he is no longer required to perform the regular duties of his office, but may continue to perform certain of these duties on a voluntary basis, if requested by the Session or a higher court. He may attend Diaconate or Session meetings, if he so desires, and may participate fully in the discussion of any issues, but may not vote.
\end{enumerate}

\hypertarget{congregational-meetings}{%
\section*{27. Congregational Meetings}\label{congregational-meetings}}
\addcontentsline{toc}{section}{27. Congregational Meetings}

\protect\hypertarget{chapter-slug-27-congregational-meetings}{\href{}{}}

\begin{enumerate}
\def\labelenumi{\arabic{enumi}.}
\item
  \protect\hypertarget{27}{\href{}{}} The congregation shall consist of both communing and non-communing members of a particular church. Voting privileges shall be limited to communing members in good standing at that particular church. Any additional voting requirements and procedures are subject to the Bylaws of the particular church which must always be in accord with the Evangel Presbytery Book of Church Order.
\item
  Whenever it may seem for the best interests of the church that a congregational meeting should be held, the Session shall call such meeting and give public notice two weeks in advance, and no business shall be transacted at such meeting except what is stated in the notice. The Session shall always call a congregational meeting when requested in writing to do so by one-fourth of the voting members of the church.

  Upon such a proper request, if the Session cannot act, fails to act or refuses to act, to call such a congregational meeting within thirty (30) days from the receipt of such a request, then any member or members in good standing may file a complaint in accordance with the provisions of BCO \protect\hyperlink{46}{46}.
\item
  A quorum of the congregational meeting shall consist of one-fourth of the voting members of the church.
\item
  The Pastor shall be the Moderator of congregational meetings by virtue of his office. If it should be not feasible or inexpedient for him to preside, or if there is no Pastor, the Session shall appoint one of their number to call the meeting to order and to preside until the congregation shall elect their presiding officer, who may be a Minister of Evangel Presbytery or any male member of that particular church.
\item
  A Clerk shall be elected by the congregation to serve at that meeting or for a definite period, whose duty shall be to keep correct minutes of the proceedings and of all business transacted and to preserve these minutes in a permanent form, after they have been attested by the Moderator and the Clerk of the meeting. He shall also send a copy of these minutes to the Session of the church.
\item
  When a particular church is incorporated, its meetings for the transaction of the business of the corporation will be provided for in its charter and bylaws, which must always be in accord with Evangel Presbytery, and must not infringe upon the powers or duties of the Session or of the Board of Deacons.
\item
  The corporation of a particular church, through its duly elected trustees or corporation officers, (or, if unincorporated, through those who are entitled to represent the particular church in matters related to real property) shall have sole title to its property, real, personal, or mixed, tangible or intangible, and shall be sole owner of any equity in any real estate, or any fund or property of any kind held by or belonging to any particular church, or any board, society, committee, Sunday school class or branch thereof. The superior courts of the Church may receive monies or properties from a local church only by free and voluntary action of the latter.
\item
  All particular churches shall be entitled to hold, own and enjoy their own local properties, without any right of reversion whatsoever to any Presbytery, General Assembly or any other courts hereafter created, trustees or other officers of such courts.
\item
  The provisions of this BCO 27 are to be construed as a solemn covenant whereby the Church as a whole promises never to attempt to secure possession of the property of any congregation against its will, whether or not such congregation remains within or chooses to withdraw from this body. All officers and courts of the Church are hereby prohibited from making any such attempt.
\item
  While a congregation consists of all the communing members of a particular church, and in matters ecclesiastical the actions of such local congregation or church shall be in conformity with the provisions of this Book of Church Order, nevertheless, in matters pertaining to the subject matters referred to in this BCO 27, including specifically the right to affiliate with or become a member of this Presbytery and the right to withdraw from or to sever any affiliation of connection with this Presbytery hereof, action may be taken by such local congregation or local church in accordance with the civil laws applicable to such local congregation or local church; and as long as such action is taken in compliance with such applicable civil laws, then such shall be the action of the local congregation or local church.

  It is expressly recognized that each local congregation or local church shall be competent to function and to take actions covering the matters set forth herein as long as such action is in compliance with the civil laws with which said local congregation or local church must comply, and this right shall never be taken from said local congregation or local church without the express consent of and affirmative action of such local church or congregation.

  Particular churches need remain in association with any court of this body only so long as they themselves so desire. The relationship is voluntary, based upon mutual love and confidence, and is in no sense to be maintained by the exercise of any force or coercion whatsoever. A particular church may withdraw from any court of this body at any time for reasons which seem to it sufficient.
\item
  If a church shall be dissolved by the Presbytery, or otherwise cease to exist, and no disposition has been made of its property by those who hold the title to the property within six months after such dissolution, then those who held the title to the property at the time of such dissolution shall, if requested by the Presbytery, deliver, convey, and transfer to the Presbytery of which the church was a member, or to the authorized agents of the Presbytery, all property of the church; and the receipt and acquittance of the Presbytery, or its proper representatives, shall be a full and complete discharge of all liabilities of such persons holding the property of the church. The Presbytery receiving such property shall apply the same or the proceeds thereof at its discretion.
\end{enumerate}

\hypertarget{declaration-of-doctrine-and-policies-concerning-sexuality}{%
\section*{28. Declaration of Doctrine and Policies Concerning Sexuality}\label{declaration-of-doctrine-and-policies-concerning-sexuality}}
\addcontentsline{toc}{section}{28. Declaration of Doctrine and Policies Concerning Sexuality}

\protect\hypertarget{chapter-slug-28-declaration-of-doctrine-and-policies-concerning-sexuality}{\href{}{}}

\protect\hypertarget{28}{\href{}{}}\textbf{Introduction.} Since the mid-twentieth century, rebellion against God's divine pattern of sex within the loving union of lifelong, monogamous, heterosexual marriage has become widespread, and those attacks are increasingly perpetrated in concert with the civil magistrate. Thus for the protection of the Christian church's conscience and the purity of the Faith once for all delivered to the saints, it has become necessary for Christians to declare our Scriptural convictions and commitments concerning sexuality and marriage. These convictions and commitments are testified to by God's natural revelation and they are explicitly commanded by God's special revelation of Holy Scripture. Prior generations of Christians and non-Christians alike lived under the beneficial constraints of laws written to protect civil society from sexual relations outside these commands of Scripture. Western law reinforced God's law concerning sexual relations and His Order of Creation of Adam first, then Eve. For this reason, the church had no need to adopt doctrinal creeds or statements concerning sexuality. Now though, with the heathens' attack growing ever more intense and becoming institutionalized by the power of civil authority, the time has come for the church to declare her allegiance to God's law of male and female, and to do so specifically, forthrightly, and with confidence in the wisdom and kindness of God. He is the One who made us. We did not make ourselves. He is the One who created us male and female, pronouncing His creation ``very good'' only after He had made man and woman.

\begin{enumerate}
\def\labelenumi{\arabic{enumi}.}
\tightlist
\item
  ``In the day when God created man, He made him in the likeness of God. God created them male and female, and He blessed them and named them Man in the day when they were created.''\footnote{Genesis 5:2. Unless otherwise indicated, all Scripture references are to the New American Standard Bible (1995).} God formed the first male, Adam, from the dust of the ground. He made the first female, Eve, from Adam's rib\footnote{``However, in the Lord, neither is woman independent of man, nor is man independent of woman. For as the woman originates from the man, so also the man \emph{has his birth} through the woman; and all things originate from God.'' 1 Corinthians 11:11--12.} and presented her to Adam to be his helpmeet. Adam called his wife ``Woman, because she was taken out of Man.''\footnote{Genesis 2:23.} God named the race adam after the first man Adam.\footnote{See Hebrew word translated as ``man'' in Genesis 1:27, and throughout the Old Testament.}
\end{enumerate}

\begin{enumerate}
\def\labelenumi{\arabic{enumi}.}
\setcounter{enumi}{1}
\tightlist
\item
  From the beginning, God gave Adam authority over Eve and responsibility for her. This authority and responsibility are inseparably joined together. Eve was created to be a ``help meet'' for Adam,\footnote{Genesis 2:18 (KJV).} that is, a help fitting for Adam. God's decree of father-rule is the necessary outworking of the authority over Eve and responsibility for her that He placed on Adam. Man is to love and take responsibility for woman by leading her and laying down his life for her, providing a living illustration of Christ's sacrificial leadership of His Bride, the Church. This estate of male responsibility and authority is not a consequence of the Fall, but was ordained by God in the beginning: ``But I do not allow a woman to teach or exercise authority over a man, but to remain quiet. For it was Adam who was first created, and then Eve. And it was not Adam who was deceived, but the woman being deceived, fell into transgression.''\footnote{1 Timothy 2:12--14.} God's subordination of the woman to man in no way diminishes the woman's perfect equality with man in essence, worth, and honor.
\end{enumerate}

\begin{enumerate}
\def\labelenumi{\arabic{enumi}.}
\setcounter{enumi}{2}
\tightlist
\item
  God's bifurcation of mankind into two and only two sexes, male and female, is an act of His creative will and power and continues through the generations since Adam, our first father.
\item
  God forms each person in his mother's womb and creates the unborn child male or female.\footnote{See Psalm 139:13--16; Genesis 5:2.} From the moment of conception,\footnote{Because all truth is God's truth and God is not a man that He should lie, what God reveals through His Creation never contradicts what He reveals in His Word. Thus, genetics and other scientific disciplines, when not abused or corrupted for sinful purposes, can declare truth. It is proper then to recognize that a man conceived and born a man is genetically male, and a woman conceived and born a woman is genetically female.} males are distinct from females, and females are distinct from males.\footnote{This is not to address the extremely rare case of intersex children born with a variety of conditions, including: Not XX and Not XY, Hypospadias, Androgen Insensitivity Syndrome, Ovotestes, etc. ``Intersex'' is a medical diagnosis of atypical male or female anatomies not to be confused with those born with typical male or female anatomies who claim a ``transgender'' or ``transsexual'' identity. The suffering of those born with physical anomalies must not be used to justify the rebellion of those who repudiate the sex God made them. All such abnormalities and genetic deformities (e.g., deafness, blindness, etc.) along with illness of any kind, pain, suffering, and death itself, are a result of the Fall as described in Genesis 3, and pass through God's sovereign hand as He gives to each man both good and adversity. Exodus 4:11; Job 2:9--10.}
\end{enumerate}

\begin{enumerate}
\def\labelenumi{\arabic{enumi}.}
\setcounter{enumi}{4}
\tightlist
\item
  What God has decreed as each one's sex at the moment of conception, either male or female, is His gift and must be received with gratitude, each man living out his manhood and each woman her womanhood, in humble reliance upon God's grace.
\item
  God's created order of man and woman was established while man was in a state of perfection before the Fall. God's creation of Adam first, then Eve, is the origin of Scripture's condemnation of woman exercising authority over man and is further delineated by Scripture's declaration that man is the glory of God, but woman is the glory of man.\footnote{``For a man ought not to have his head covered, since he is the image and glory of God; but the woman is the glory of man. For man does not originate from woman, but woman from man; for indeed man was not created for the woman's sake, but woman for the man's sake.'' 1 Corinthians 11:7--9.} This order of creation, then, is a universal truth for all mankind.
\end{enumerate}

\begin{enumerate}
\def\labelenumi{\arabic{enumi}.}
\setcounter{enumi}{6}
\tightlist
\item
  \protect\hypertarget{28.7}{\href{}{}}The proper conception of ``sexual identity'' is a matter of being and obeying the genetic sex God made us, either male or female. Genetic sex and sexual identity cannot be separated, and they remain bound together throughout one's life. Sexuality does not admit of gradations. You are either male or female, not part male and part female. Nor are there a great multitude of sexual identities. There are only two, male and female. Any attempt of a man to play the woman or a woman to play the man violates God's decree, attacks His created order, and constitutes sin so serious that God Himself pronounces it an ``abomination.''\footnote{Deuteronomy 22:5. See also 1 Corinthians 6:9--10.}
\end{enumerate}

\begin{enumerate}
\def\labelenumi{\arabic{enumi}.}
\setcounter{enumi}{7}
\tightlist
\item
  This sin includes transvestitism and any efforts, including chemical or surgical, as well as behavioral (e.g., effeminacy), to reject and efface one's sex and to adopt characteristics of the opposite sex.
\item
  This sin also includes the conscription of woman as a military combatant or the placing of woman in harm's way as a law enforcement officer. Following the example of Christ Jesus who gave up His life for His Bride, the Church, man is to lay down his life in defense of woman.\footnote{See Ephesians 5:25.} As life-giver,\footnote{See Genesis 3:20.} woman has always been honored by man's defense of her and her children. A civil magistrate defaces woman's sexuality by placing her in the uniform of a combatant\footnote{See Deuteronomy 22:5. In the text, ``that which pertaineth to a man'' (KJV) refers to the clothing and weapons worn by men for combat.} and commanding her to take up arms. Further, given woman's comparative physical weakness in the face of male enemies, such magistrates place their homeland at unnecessary risk. Finally, female military combatants of childbearing age often (whether known or unknown) place at risk unborn children, which is an act contrary to the just war principle of avoiding needless loss of life. Therefore, Evangel Presbytery condemns the use of women as military combatants, the conscription of women into the armed forces, and any requirement for her to register for conscription into military service.\footnote{See Constitution and Bylaws of Trinity Presbyterian Church (PCA) (Spartanburg, SC), art. XIII (``Selective Service'') (updated May 28, 2017). See also ``Man's Duty to Protect Woman,'' Majority Report of the Presbyterian Church in America's General Assembly Ad Interim Study Committee on Women in the Military, 2001, \url{https://www.pcahistory.org/pca/digest/studies/01-278.html}.}
\end{enumerate}

\begin{enumerate}
\def\labelenumi{\arabic{enumi}.}
\setcounter{enumi}{9}
\tightlist
\item
  All the sins mentioned above are violations of the separate nature and callings of Adam and Eve, man and woman. Through repentance and faith in Jesus Christ, there is full forgiveness for each of these sins.
\item
  It is a particularly necessary and vital ministry of the church today to proclaim this forgiveness and to teach and help those who repent of these sins to live faithfully as the man or woman God made them.
\item
  As the household of faith, the church is privileged to assist parents in training boys to be men and girls to be women. This training is not simply Christian discipleship but is also part of the Church's proclamation of the Gospel for the salvation of mankind.
\item
  Being committed to such evangelism and discipleship, Evangel Presbytery limits participation in any program or activity that is limited to men (e.g., a men's Bible study or a men's retreat) exclusively to males.
\item
  Participation in any program or activity of Evangel Presbytery that is limited to women (e.g., a women's Bible study or a women's retreat) is exclusively limited to females.
\item
  Access to facilities of Evangel Presbytery that are designated for use by men (e.g., a men's restroom) is exclusively limited to males.
\item
  Access to facilities of Evangel Presbytery that are designated for use by women (e.g., a women's restroom or nursing/cry room, etc.) is exclusively limited to females.
\item
  Recognizing the church's own sins and failures in obeying God's commands of sexuality and recognizing that sinners on the road of repentance and faith in Jesus Christ often enter the church bound in deep patterns of sexual sin, it is not our expectation that sinners will respond to the preaching of the Law and the proclamation of the Gospel with immediate and full repentance. For this reason, wise and compassionate pastoral discretion is necessary to apply the rules set forth in this Declaration, especially concerning access to facilities. In certain instances the higher law of love will preclude swift and rigid enforcement of rules. For example, if a female has transitioned to a male in appearance, it may be best that she not use the bathroom of her birth sex until she has been presented with pastoral counsel concerning God's calling of manhood and womanhood and she begins to learn of Jesus' Lordship over her sexuality and the implications it has for her sexual identity and its public expression. In this case, the female would be asked not to use the men's restroom, but instead a single-stall restroom available to either sex.
\item
  Marriage is instituted by God as the union of one male and one female,\footnote{``FOR THIS REASON A MAN SHALL LEAVE HIS FATHER AND MOTHER, AND THE TWO SHALL BECOME ONE FLESH; so they are no longer two, but one flesh. What therefore God has joined together, let no man separate.'' Mark 10:7--9.``Or do you not know that the unrighteous will not inherit the kingdom of God? Do not be deceived; neither fornicators, nor idolaters, nor adulterers, nor effeminate, nor homosexuals, nor thieves, nor the covetous, nor drunkards, nor revilers, nor swindlers, will inherit the kingdom of God.'' 1 Corinthians 6:9--10.} which union is intended to be monogamous\footnote{``You shall not commit adultery.'' Exodus 20:14.} and lifelong.\footnote{``So then, if while her husband is living she is joined to another man, she shall be called an adulteress; but if her husband dies, she is free from the law, so that she is not an adulteress though she is joined to another man.'' Romans 7:3.} This relationship, with these limitations, was established while man was in a state of perfection before the Fall\footnote{See Genesis 2:18--24.} and is established for all members of the human race. Marriage is an honorable estate that God Himself made, and it symbolizes to us the mystical union which is between Christ and His Church.\footnote{``Husbands, love your wives, just as Christ also loved the church and gave Himself up for her, so that He might sanctify her, having cleansed her by the washing of water with the word, that He might present to Himself the church in all her glory, having no spot or wrinkle or any such thing; but that she would be holy and blameless. So husbands ought also to love their own wives as their own bodies. He who loves his own wife loves himself; for no one ever hated his own flesh, but nourishes and cherishes it, just as Christ also does the church, because we are members of His body. FOR THIS REASON A MAN SHALL LEAVE HIS FATHER AND MOTHER AND SHALL BE JOINED TO HIS WIFE, AND THE TWO SHALL BECOME ONE FLESH. This mystery is great; but I am speaking with reference to Christ and the church.'' Ephesians 5:25--32.} For centuries, Christians have recognized the following vital purposes of marriage: ``Marriage was ordained for the mutual help of husband and wife, for the increase of mankind with a legitimate issue, and of the Church with an holy seed; and for preventing of uncleanness.''\footnote{Westminster Confession of Faith, ch.~24 (``Of Marriage and Divorce''), para. 2. Westminster Confession of Faith hereinafter abbreviated as ``WCF.''} This holy estate Christ made beautiful by His presence and first miracle at a wedding in Cana of Galilee.\footnote{See John 2:1--11.}
\end{enumerate}

\begin{enumerate}
\def\labelenumi{\arabic{enumi}.}
\setcounter{enumi}{18}
\tightlist
\item
  When sin entered the race, marriage was sorely affected.\footnote{``When the woman saw that the tree was good for food, and that it was a delight to the eyes, and that the tree was desirable to make \emph{one} wise, she took from its fruit and ate; and she gave also to her husband with her, and he ate. Then the eyes of both of them were opened, and they knew that they were naked; and they sewed fig leaves together and made themselves loin coverings. They heard the sound of the LORD God walking in the garden in the cool of the day, and the man and his wife hid themselves from the presence of the LORD God among the trees of the garden. Then the LORD God called to the man, and said to him, `Where are you?' He said, `I heard the sound of You in the garden, and I was afraid because I was naked; so I hid myself.' And He said, `Who told you that you were naked? Have you eaten from the tree of which I commanded you not to eat?'\,'' Genesis 3:6--11.} God therefore created laws to govern the violation of His established pattern, while not changing the pattern.\footnote{``He said to them, `Because of your hardness of heart Moses permitted you to divorce your wives; but from the beginning it has not been this way.'\,'' Matthew 19:8.``For truly I say to you, until heaven and earth pass away, not the smallest letter or stroke shall pass from the Law until all is accomplished.'' Matthew 5:18.} Since marriage is a lifelong and monogamous union, God forbids divorce except in two circumstances: (1) when one spouse abandons the other,\footnote{Abandonment or ``wilful desertion'' (WCF 24.6) is not limited only to cases of one spouse's unjustified departure and refusal to be reconciled to the abandoned or injured spouse. A state of willful desertion also exists if the offending party's conduct is so egregious that the injured party is forced to leave the marital home and reconciliation is impossible due to the nature or seriousness of the sin and the offending party's persistent impenitence. As in any other case involving the Scriptural permissibility of divorce, the board of elders will judge whether the offending party's sin justifies the injured party in separating and seeking dissolution of the marital bond. Two resources we commend in considering cases of divorce are: ``Marriage, Singleness, Divorce, and Remarriage,'' statement adopted by the session of Trinity Reformed Church of Bloomington, Indiana, June 2, 1999, available at \url{https://evangelpresbytery.com/wp-content/uploads/2019/10/Divorce-Policy-FINAL-2000.pdf}; and ``Divorce and Remarriage,'' Report of the Ad-Interim Study Committee on Marriage and Divorce to the Twentieth General Assembly of the Presbyterian Church in America, 1992, \url{https://pcahistory.org/pca/digest/studies/divorce-remarriage.pdf}.} and (2) when a spouse engages in sexual immorality---what Jesus referred to as the sin of \emph{porneia}.\footnote{See ``Marriage, Singleness, Divorce, and Remarriage,'' pp.~3--4. See also WCF 24.6.}
\end{enumerate}

\begin{enumerate}
\def\labelenumi{\arabic{enumi}.}
\setcounter{enumi}{19}
\tightlist
\item
  More broadly, in His Law God forbids any deviation from the established pattern of marriage and from His gift of sexual union to be enjoyed by a married couple. These forbidden deviations include lust for anyone but one's spouse, pornography,\footnote{``But I say to you that everyone who looks at a woman with lust for her has already committed adultery with her in his heart.'' Matthew 5:28.} masturbation (including simulated copulation with any inanimate object no matter how lifelike),\footnote{``Finally then, brethren, we request and exhort you in the Lord Jesus, that as you received from us \emph{instruction} as to how you ought to walk and please God (just as you actually do walk), that you excel still more. For you know what commandments we gave you by \emph{the authority} of the Lord Jesus. For this is the will of God, your sanctification; \emph{that is}, that you abstain from sexual immorality; that each of you know how to possess his own vessel in sanctification and honor, not in lustful passion, like the Gentiles who do not know God; \emph{and} that no man transgress and defraud his brother in the matter because the Lord is \emph{the} avenger in all these things, just as we also told you before and solemnly warned \emph{you}. For God has not called us for the purpose of impurity, but in sanctification. So, he who rejects \emph{this} is not rejecting man but the God who gives His Holy Spirit to you.'' 1 Thessalonians 4:1--8.``But I say, walk by the Spirit, and you will not carry out the desire of the flesh.'' Galatians 5:16.} fornication,\footnote{``Marriage is to be held in honor among all, and the marriage bed is to be undefiled; for fornicators and adulterers God will judge.'' Hebrews 13:4.} adultery,\footnote{``You shall not commit adultery.'' Exodus 20:14.} polygamy,\footnote{``And He answered and said, `Have you not read that He who created them from the beginning MADE THEM MALE AND FEMALE, and said, ``FOR THIS REASON A MAN SHALL LEAVE HIS FATHER AND MOTHER AND BE JOINED TO HIS WIFE, AND THE TWO SHALL BECOME ONE FLESH''? So they are no longer two, but one flesh. What therefore God has joined together, let no man separate.'\,'' Matthew 19:4--6. ``An overseer, then, must be above reproach, the husband of one wife . . .'' 1 Timothy 3:2.} incest,\footnote{``If there is a man who lies with his daughter-in-law, both of them shall surely be put to death; they have committed incest, their bloodguiltiness is upon them.'' Leviticus 20:12. See also all of Leviticus 20.} pedophilia,\footnote{``He said to His disciples, `It is inevitable that stumbling blocks come, but woe to him through whom they come! It would be better for him if a millstone were hung around his neck and he were thrown into the sea, than that he would cause one of these little ones to stumble.'\,'' Luke 17:1--2.} homosexuality,\footnote{``You shall not lie with a male as one lies with a female; it is an abomination.'' Leviticus 18:22.} and bestiality.\footnote{``Also you shall not have intercourse with any animal to be defiled with it, nor shall any woman stand before an animal to mate with it; it is a perversion.'' Leviticus 18:23.} Because the heart of man is deceitful above all things and desperately wicked, it is impossible to catalogue fully all forms of sexual immorality and degradation.
\end{enumerate}

\begin{enumerate}
\def\labelenumi{\arabic{enumi}.}
\setcounter{enumi}{20}
\tightlist
\item
  God offers the free gift of forgiveness for these sins through repentance and faith in Jesus, and He has called the Church to proclaim that repentance and forgiveness, seeking the healing of those harmed by these sins.\footnote{``Such were some of you; but you were washed, but you were sanctified, but you were justified in the name of the Lord Jesus Christ and in the Spirit of our God.'' 1 Corinthians 6:11.``The Spirit of the LORD God is upon me,
    Because the LORD has anointed me
    To bring good news to the afflicted;
    He has sent me to bind up the brokenhearted,
    To proclaim liberty to captives
    And freedom to prisoners.''
    Isaiah 61:1. ``A BATTERED REED HE WILL NOT BREAK OFF,
    AND A SMOLDERING WICK HE WILL NOT PUT OUT,
    UNTIL HE LEADS JUSTICE TO VICTORY.''
    Matthew 12:20.}
\end{enumerate}

\begin{enumerate}
\def\labelenumi{\arabic{enumi}.}
\setcounter{enumi}{21}
\tightlist
\item
  Presently, there is widespread disregard, even scorn, for the divine standard of sexuality and marriage as revealed in Scripture. This disregard, which is found both in the world and in many churches, endangers the family (our basic social unit) and causes much suffering to the innocent, especially children. When found in the Church, it brings shame to the name of Christ.\footnote{``However, because by this deed you have given occasion to the enemies of the LORD to blaspheme, the child also that is born to you shall surely die.'' 2 Samuel 12:14. This text demonstrates dishonor coming to God's holy name and suffering coming to the members of a man's family---both caused by the sin of the father. ``For `THE NAME OF GOD IS BLASPHEMED AMONG THE GENTILES BECAUSE OF YOU,' just as it is written.'' Romans 2:24.}
\end{enumerate}

\begin{enumerate}
\def\labelenumi{\arabic{enumi}.}
\setcounter{enumi}{22}
\tightlist
\item
  A pastor, officer, leader, teacher, employee, or member of Evangel Presbytery shall not solemnize a marriage or officiate at a wedding in any place or under any circumstances that violate God's Law or His established pattern of marriage between one male and one female, which union is intended to be monogamous and lifelong. Recognizing the universal application of God's Order of Creation and the resultant Scriptural command that wives submit to their husbands, any wedding solemnized by a pastor of Evangel Presbytery will include in the wife's vows her promise to ``obey'' her husband. This promise to ``obey'' will also be required in any wedding permitted by Evangel Presbytery's Board of Elders to be held on church property or in a church facility but officiated by a pastor unaffiliated with this congregation.
\item
  The property or any facility of Evangel Presbytery shall not be used for a marriage ceremony, wedding, or related event that violates God's Law or His established pattern of marriage between one male and one female, which union is intended to be monogamous and lifelong.
\item
  All ministries of Evangel Presbytery, including educational ministries for youth, shall adhere to and abide by this Declaration.
\item
  The Scriptures of the Old and New Testaments are the only infallible rule of faith and practice and the only authority that may bind the conscience.
\item
  This Declaration does not exhaust the extent of our beliefs and practices. The Bible itself, as the inspired and infallible Word of God that speaks with final authority concerning truth, morality, and the proper conduct of mankind, is the sole and final source of authority for all that we believe and do. For purposes of Evangel Presbytery's faith, doctrine, practice, policy, and discipline, the membership of the Presbytery (as constituted under BCO \protect\hyperlink{15.1}{15.1}) is the Presbytery's final interpretive authority on the Bible's meaning and application, including for any purpose under BCO \protect\hyperlink{15.8}{15.8}.
\end{enumerate}

\hypertarget{amending-the-constitution-of-the-church}{%
\section*{29. Amending the Constitution of the Church}\label{amending-the-constitution-of-the-church}}
\addcontentsline{toc}{section}{29. Amending the Constitution of the Church}

\protect\hypertarget{chapter-slug-29-amending-the-constitution-of-the-church}{\href{}{}}

\begin{enumerate}
\def\labelenumi{\arabic{enumi}.}
\tightlist
\item
  \protect\hypertarget{29}{\href{}{}}\protect\hypertarget{29.1}{\href{}{}}The Constitution of Evangel Presbytery, which is subject to and subordinate to the Scriptures of the Old and New Testaments, the infallible Word of God, consists of its doctrinal standards set forth in the \emph{Westminster Confession of Faith}\footnote{American revisions as adopted by the Orthodox Presbyterian Church in 1936, \url{https://evangelpresbytery.com/westminster-confession-of-faith}.}, together with the \emph{Larger}\footnote{\url{https://evangelpresbytery.com/westminster-larger-catechism}.} and \emph{Shorter Catechisms}\footnote{\url{https://evangelpresbytery.com/westminster-shorter-catechism}.}; the Book of Church Order, which comprises the Form of Government, the Rules of Discipline, and the Directory for the Worship of God; and the Apostles' Creed, Nicene Creed, Chalcedonian Creed, and Athanasian Creed; all as adopted by the Presbytery.
\end{enumerate}

\begin{enumerate}
\def\labelenumi{\arabic{enumi}.}
\setcounter{enumi}{1}
\item
  Amendments to the Book of Church Order may be made only in the following manner:

  \begin{enumerate}
  \def\labelenumii{\alph{enumii}.}
  \tightlist
  \item
    Approval of the proposed amendment by the Presbytery and its recommendation to the Sessions.
  \item
    The advice and consent of a majority of the Sessions.
  \item
    The approval and enactment at a subsequent Presbytery meeting.
  \end{enumerate}
\item
  Amendments to the Confession of Faith and Catechisms of this Church may be made only in the following manner:

  \begin{enumerate}
  \def\labelenumii{\alph{enumii}.}
  \tightlist
  \item
    The approval of the proposed amendment by the Presbytery and its recommendation to the Sessions.
  \item
    The advice and consent of three fourths of the Sessions.
  \item
    The approval and enactment at a subsequent Presbytery meeting.
  \end{enumerate}

  This paragraph can be amended only by the same method as it prescribes for the amendment of the Confession of Faith and Catechisms of the Church.
\item
  Full organic union and consolidation of Evangel Presbytery with any other ecclesiastical body can be effected only in the following manner:

  \begin{enumerate}
  \def\labelenumii{\alph{enumii}.}
  \tightlist
  \item
    The approval of the proposed union by the Presbytery and its recommendation to the Sessions.
  \item
    The advice and consent of three fourths of the Sessions.
  \item
    The approval and consummation at a subsequent Presbytery meeting.
  \end{enumerate}

  This paragraph can be amended only by the same method which is prescribed for the amendment of the Confession of Faith and Catechisms of the Church.
\item
  If by reason of the failure of a number of Sessions to act, or to report action, on any proposed amendment to the Standards, the response of the Sessions is not satisfactory to the succeeding Presbytery meeting, it may defer action for six months. In that event the Presbytery shall urge the delinquent Sessions to report their judgment to the next Presbytery meeting, which shall take final action on the proposed amendment.
\end{enumerate}

\hypertarget{the-rules-of-discipline}{%
\chapter*{The Rules of Discipline}\label{the-rules-of-discipline}}
\addcontentsline{toc}{chapter}{The Rules of Discipline}

\hypertarget{disciplineits-nature-subjects-and-ends}{%
\section*{30. Discipline---Its Nature, Subjects, and Ends}\label{disciplineits-nature-subjects-and-ends}}
\addcontentsline{toc}{section}{30. Discipline---Its Nature, Subjects, and Ends}

\protect\hypertarget{chapter-slug-30-discipline-its-nature-subjects-and-ends}{\href{}{}}

\begin{enumerate}
\def\labelenumi{\arabic{enumi}.}
\item
  \protect\hypertarget{30}{\href{}{}}Discipline is the exercise of authority given the Church by the Lord Jesus Christ to instruct and guide its members and to promote its purity and welfare. The term has two senses:

  \begin{enumerate}
  \def\labelenumii{\alph{enumii}.}
  \tightlist
  \item
    \protect\hypertarget{30.1.a}{\href{}{}}the one referring to the whole government, inspection, training, guardianship, and control which the Church maintains over its members, its officers, and its courts;
  \item
    the other a restricted and technical sense, signifying judicial process.
  \end{enumerate}
\item
  In the first and broader sense, all communing, non-communing, and associate members of the Church are subject to its discipline and entitled to the benefits thereof; but in the second and narrower sense the term discipline refers only to those who have made a profession of their faith in Christ.
\item
  In its proper usage discipline maintains: a) the glory of God; b) the purity of His Church,; and c) the keeping and reclaiming of disobedient sinners. Discipline is for the purpose of godliness;\footnote{1 Timothy 4:7.} therefore it demands a self-examination under Scripture.

  The ends of discipline, so far as it involves judicial action, are the spiritual good of the offender, the rebuke of offenses, the removal of scandal, the promotion of the purity and welfare of the Church, and the vindication of the honor of Christ.
\end{enumerate}

\begin{enumerate}
\def\labelenumi{\arabic{enumi}.}
\setcounter{enumi}{3}
\tightlist
\item
  \protect\hypertarget{30.4}{\href{}{}}The power which Christ has given the Church is for building up, and not for destruction, and is to be exercised as under a dispensation of mercy and not of wrath. As in the preaching of the Word the wicked are doctrinally separated from the good, so by discipline the Church authoritatively separates between the holy and the profane. The Church is to act as a mother who corrects her children for their good, that every one of them may be presented faultless in the day of Christ.
\end{enumerate}

\hypertarget{discipline-of-non-communing-members}{%
\section*{31. Discipline of Non-communing Members}\label{discipline-of-non-communing-members}}
\addcontentsline{toc}{section}{31. Discipline of Non-communing Members}

\protect\hypertarget{chapter-slug-31-discipline-of-non-communing-members}{\href{}{}}

\begin{enumerate}
\def\labelenumi{\arabic{enumi}.}
\tightlist
\item
  \protect\hypertarget{31}{\href{}{}}The spiritual nurture, instruction, and training of the children of the Church are committed by God primarily to their parents who are responsible to the Church for the faithful discharge of their obligations, and it is a principal duty of the Church to promote true religion in the home.
\item
  The Church should also make special provision for instructing the children in the Bible and in the church Catechisms, and to this end Sessions should establish and conduct under their authority Sunday Schools and Bible classes, and should adopt such other methods as may be found helpful. The Session shall encourage parents of the Church to guide their children in the catechising and disciplining of them in the Christian religion.
\item
  The Church should maintain constant and sympathetic relations with the children, and should encourage them on coming to years of discretion\footnote{Isaiah 7:15-16.} to make confession of the Lord Jesus Christ and to enter upon all the privileges of full church membership. Even if they are wayward they should be cherished by the Church and all diligent means used to reclaim them.
\end{enumerate}

\begin{enumerate}
\def\labelenumi{\arabic{enumi}.}
\setcounter{enumi}{3}
\tightlist
\item
  Adult non-communing members\footnote{From Trinity Reformed Church of Bloomington's Practice of Admitting Children to the Sacraments (adopted December 11, 2014): ``In cases when disability renders a profession of faith unclear, it is the responsibility of the elders to make a determination on a case-by-case basis, praying and trusting God for guidance in this exercise of the power of the keys'' (Matthew 16:19). In cases when it is impossible for a person to communicate a verbal or nonverbal profession of faith, since knowledge, assent, and trust cannot be established, admittance to the sacrament of the Lord's Supper shall not be granted. These cases may include severe autism, severe Down's syndrome, or other debilitating conditions. It is not clear that the souls who, in the providence of God, suffer these conditions are able to benefit from the grace that attends this meal. Not being able to demonstrate any capacity to remember Christ's death or to examine themselves, it is impossible to establish the faith that qualifies us for participation in the sacraments. Thus in Christian love we will trust God with His dispensation toward these children and adults, asking our heavenly Father to confer on them every saving grace.''} who receive with meekness and appreciation the oversight and instruction of the Church are entitled to special attention. Their rights and privileges under the covenant should be frequently and fully explained, and they should be warned of the sin and danger of neglecting their covenant obligations. When a non-communing member neglects the ongoing exhortation of the session to profess faith in Christ and rejects the covenantal responsibility of submission to home or church, the session may upon prior notification erase his name from the roll. This is an act of discipline without full process.
\end{enumerate}

\begin{enumerate}
\def\labelenumi{\arabic{enumi}.}
\setcounter{enumi}{4}
\tightlist
\item
  All non-communing members shall be deemed under the care of the church to which their parents belong, if they live under the parental roof and are minors; otherwise, under that of the church where they reside, or with which they ordinarily worship.
\end{enumerate}

\hypertarget{offenses}{%
\section*{32. Offenses}\label{offenses}}
\addcontentsline{toc}{section}{32. Offenses}

\protect\hypertarget{chapter-slug-32-offenses}{\href{}{}}

\begin{enumerate}
\def\labelenumi{\arabic{enumi}.}
\tightlist
\item
  \protect\hypertarget{32}{\href{}{}}\protect\hypertarget{32.1}{\href{}{}}An offense, the proper object of judicial process, is anything in the principles or practice of a church member professing faith in Christ, which is contrary to the Word of God. The \emph{Confession of Faith} and the \emph{Larger} and \emph{Shorter Catechisms} of the Westminster Assembly, together with the formularies of government, discipline, and worship, are accepted by Evangel Presbytery as standard expositions of the teachings of Scripture in relation to both faith and practice. Nothing, therefore, ought to be considered by any court as an offense, or admitted as a matter of accusation, which cannot be proved to be such from Scripture, as interpreted in these Standards.
\item
  Offenses are either personal or general, private or public; but all of them being sins against God, are therefore grounds of discipline.
\item
  Personal offenses are violations of the divine law, considered in the special relation of wrongs or injuries to particular individuals. General offenses are heresies or immoralities having no such relation, or considered apart from it.
\item
  Private offenses are those which are known only to a few persons. Public offenses are those which are notorious.
\end{enumerate}

\hypertarget{church-censures}{%
\section*{33. Church Censures}\label{church-censures}}
\addcontentsline{toc}{section}{33. Church Censures}

\protect\hypertarget{chapter-slug-33-church-censures}{\href{}{}}

\begin{enumerate}
\def\labelenumi{\arabic{enumi}.}
\tightlist
\item
  \protect\hypertarget{33}{\href{}{}}The censures which may be inflicted by church courts are admonition, suspension from the Sacraments, suspension from office, deposition from office, and excommunication. When a lower censure fails to reclaim the delinquent, it may become the duty of the court to proceed to the infliction of a higher censure.
\item
  Admonition is the formal reproof of an offender by a church court, warning him of his guilt and danger, and exhorting him to be more circumspect and watchful in the future.
\item
  Suspension is a censure which may be inflicted on either a private member or an officer of the Church. In respect to the former, it is a temporary exclusion from the Sacraments; and to the latter, from the exercise of office, and, in ordinary cases, from Sacraments also. This censure becomes necessary when very gross offenses have been committed; or when, notwithstanding admonition or rebuke, an offense is repeated or persisted in; or when probation is necessary to attest repentance and reformation, or to restore to the offender a sense of solemnity and fear when coming to the Lord's Supper. Suspension may be for a definite time, but generally it should be indefinite in duration, and its removal depends upon evidence of repentance.
\item
  Excommunication is the excision of an offender from the communion of the Church. This censure is to be inflicted only on account of gross sin or heresy and when the offender shows himself incorrigible and contumacious. The design of this censure is to operate on the offender as a means of reclaiming him, to deliver the Church from the scandal of his offense, and to inspire all with fear by the example of his discipline.
\item
  Deposition is the removal of an officer from his office, and may or may not be accompanied with the infliction of other censure.
\end{enumerate}

\hypertarget{the-parties-in-cases-of-process}{%
\section*{34. The Parties in Cases of Process}\label{the-parties-in-cases-of-process}}
\addcontentsline{toc}{section}{34. The Parties in Cases of Process}

\protect\hypertarget{chapter-slug-34-the-parties-in-cases-of-process}{\href{}{}}

\begin{enumerate}
\def\labelenumi{\arabic{enumi}.}
\tightlist
\item
  \protect\hypertarget{34}{\href{}{}}Original jurisdiction in relation to Ministers of the Gospel pertains exclusively to the Presbytery, and in relation to other church members to the Session, unless the Session shall be unable to try the person or persons accused, in which case the Presbytery shall have the right of jurisdiction.
\item
  \protect\hypertarget{34.2}{\href{}{}}It is the duty of all church Sessions and Presbyteries to exercise care over those subject to their authority; and they shall, with due diligence and great discretion, demand from such persons satisfactory explanations concerning reports regarding their Christian character. This duty is more imperative when those who deem themselves aggrieved by injurious reports shall request an investigation. If such investigation, however originating, should result in raising a strong presumption of the guilt of the party involved, the court shall institute process, and shall appoint a prosecutor to prepare the indictment and to conduct the case. This prosecutor shall be a member of the court, except that, in a case before the Session, he may be any communing member of the same congregation with the accused.
\item
  The original and only parties in a case of process are the accuser and the accused. The accuser is always Evangel Presbytery, whose honor and purity are to be maintained. The prosecutor, whether voluntary or appointed, is always the representative of the Church, and as such has all its rights in the case. In appellate courts the parties are known as appellant and appellee.
\item
  Every indictment shall begin: ``In the name of Evangel Presbytery'' and shall conclude, ``against the peace, unity and purity of the Church, and the honor and majesty of the Lord Jesus Christ as the King and Head thereof.'' In every case the Church is the injured and accusing party, against the accused.
\item
  An injured party shall not become a prosecutor of personal offenses without having tried the means of reconciliation and of reclaiming the offender, required by Christ: ``If your brother sins, go and show him his fault in private; if he listens to you, you have won your brother. But if he does not listen to you, take one or two more with you, so that BY THE MOUTH OF TWO OR THREE WITNESSES EVERY FACT MAY BE CONFIRMED.''\footnote{Matthew 18:15-16.}
\end{enumerate}

\begin{enumerate}
\def\labelenumi{\arabic{enumi}.}
\setcounter{enumi}{5}
\tightlist
\item
  A church court, however, may judicially investigate personal offenses as if general, when the interests of religion seem to demand it. So, also, those to whom private offenses are known cannot become prosecutors, without having previously endeavored to remove the scandal by private means.
\item
  When the offense is general, the cause may be conducted either by any person appearing as prosecutor, or by a prosecutor appointed by the court.
\item
  When the prosecution is instituted by the court, the previous steps required by our Lord in the case of personal offenses are not necessary. There are many cases, however, in which it will promote the interests of religion to send a committee to converse in a private manner with the offender, and endeavor to bring him to a sense of his guilt, before instituting actual process.
\item
  Great caution ought to be exercised in receiving accusations from any person who is known to indulge a malignant spirit towards the accused; who is not of good character; who is himself under censure or process; who is deeply interested in any respect in the conviction of the accused; or who is known to be litigious, rash, or highly imprudent.
\item
  Every voluntary prosecutor shall be previously warned, that if he fail to show probable cause of the charges, he must himself be censured as a slanderer of the brethren, in proportion to the malignity or rashness manifested in the prosecution.
\item
  When a member of a church court is under process, all his official functions may be suspended, at its discretion; but this shall never be done in the way of censure.
\item
  In the discussion of all questions arising when a member of a church court is under process, the accused shall exercise the rights of defendant only, not of judge.
\end{enumerate}

\hypertarget{general-provisions-applicable-to-all-cases-of-process}{%
\section*{35. General Provisions Applicable to All Cases of Process}\label{general-provisions-applicable-to-all-cases-of-process}}
\addcontentsline{toc}{section}{35. General Provisions Applicable to All Cases of Process}

\protect\hypertarget{chapter-slug-35-general-provisions-applicable-to-all-cases-of-process}{\href{}{}}

\begin{enumerate}
\def\labelenumi{\arabic{enumi}.}
\tightlist
\item
  \protect\hypertarget{35}{\href{}{}}It is incumbent on every member of a court of Jesus Christ engaged in a trial of offenders, to bear in mind the inspired injunction: ``Brethren, even if anyone is caught in any trespass, you who are spiritual, restore such a one in a spirit of gentleness; each one looking to yourself, so that you too will not be tempted.''\footnote{Galatians 6:1.}
\end{enumerate}

\begin{enumerate}
\def\labelenumi{\arabic{enumi}.}
\setcounter{enumi}{1}
\tightlist
\item
  Process against an offender shall not be commenced unless some person or persons undertake to make out the charge; or unless the court finds it necessary, for the honor of religion, itself to take the step provided for in BCO \protect\hyperlink{34.2}{34.2}.
\item
  It is appropriate that with each citation the moderator or clerk call the attention of the parties to the Rules of Discipline and assist the parties to obtain access to them. When a charge is laid before the Session or Presbytery, it shall be reduced to writing, and nothing shall be done at the first meeting of the court, unless by consent of parties, except to appoint a prosecutor, and order the indictment to be drawn, a copy of which, with the witnesses then known to support it, shall be served on the accused, and to cite all parties and their witnesses to appear and be heard at another meeting, which shall not be sooner than ten days after such citation. At the second meeting of the court the charges shall be read to the accused, if present, and he shall be called upon to say whether he be guilty or not. If he confess, the court may deal with him according to its discretion; if he plead and take issue, the trial shall be scheduled and all parties and their witnesses cited to appear. The trial shall not be sooner than fourteen (14) days after such citation. Accused parties may plead in writing when they cannot be personally present. Parties necessarily absent should have counsel assigned to them.
\item
  The citation shall be issued and signed by the Moderator or Clerk, by order and in the name of the court; he shall also issue citations to such witnesses as either party shall nominate to appear on his behalf. Indictments and citations shall be delivered in person or in another manner providing verification of the date of receipt. Compliance with these requirements shall be deemed to have been fulfilled if a party cannot be located after diligent inquiry or if a party refuses to accept delivery.
\item
  In drawing the indictment, the times, places, and circumstances should, if possible, be particularly stated, that the accused may have an opportunity to make his defense.
\item
  When an accused person shall refuse to obey a citation, he shall be cited a second time; and this second citation shall be accompanied with a notice that if he does not appear at the time appointed (unless providentially hindered, which fact he must make known to the court), or that if he appear and refuse to plead, he shall be dealt with for his contumacy, as hereinafter provided (see BCO \protect\hyperlink{36.2}{36.2}, \protect\hyperlink{36.3}{36.3}; \protect\hyperlink{37.4}{37.4}).
\item
  The time which must elapse between the serving of the first citation on the accused person, and the meeting of the court at which he is to appear, shall be at least ten days. But the time allotted for his appearance on the subsequent citation shall be left to the discretion of the court, provided that it be not less than is quite sufficient for a seasonable and convenient compliance with the citation.
\item
  \protect\hypertarget{35.8}{\href{}{}}When the offense with which an accused person stands charged took place at a distance, and it is inconvenient for the witnesses to appear before the court having jurisdiction, that court may either appoint a commission of its body, or request the coordinate court\footnote{A coordinate court is a member church of the Presbytery, or any church or presbytery holding to the Westminster Standards.} contiguous to the place where the facts occurred, to take the testimony for it. The accused shall always have reasonable notice of the time and place of the meeting of this commission or coordinate court.
\end{enumerate}

\begin{enumerate}
\def\labelenumi{\arabic{enumi}.}
\setcounter{enumi}{8}
\tightlist
\item
  When an offense, alleged to have been committed at a distance, is not likely otherwise to become known to the court having jurisdiction, it shall be the duty of the court nearest to where the facts occurred, after satisfying itself that there is probable ground of accusation, to send notice to the court having jurisdiction, which shall at once proceed against the accused; or the whole case may be remitted for trial to the court nearest to where the offense is alleged to have been committed.
\item
  Before proceeding to trial, courts ought to ascertain that their citations have been duly served.
\item
  In every process, if deemed expedient, there may be a committee appointed, which shall be called the Judicial Committee, and whose duty it shall be to digest and arrange all the papers, and to prescribe, under the direction of the court, the whole order of the proceedings. The members of this committee shall be entitled, notwithstanding their performance of this duty, to sit and vote in the case as members of the court.
\item
  When the trial is about to begin, it shall be the duty of the Moderator solemnly to announce from the chair that the court is about to pass to the consideration of the cause, and to enjoin on the members to recollect and regard their high character as judges of a court of Jesus Christ, and the solemn duty in which they are about to engage.
\item
  In order that the trial may be fair and impartial, the witnesses shall be examined in the presence of the accused, or at least after he shall have received due citation to attend. Witnesses may be cross-examined by both parties, and any questions asked which are pertinent to the issue.
\item
  On all questions arising in the progress of a trial, the discussion shall first be between the parties; and when they have been heard, they may be required to withdraw from the court until the members deliberate upon and decide the point.
\item
  When a court of first resort proceeds to the trial of a cause, the following order shall be observed: 1, The Moderator shall charge the court. 2, The indictment shall be read, and the answer of the accused heard. 3, The witnesses for the prosecutor and then those for the accused shall be examined. 4, The parties shall be heard; first, the prosecutor, and then the accused, and the prosecutor shall close. 5, The roll shall be called, and the members may express their opinion in the cause. 6, The vote shall be taken, the verdict announced and judgment entered on the records.
\item
  Either party may, for cause, challenge the right of any member to sit in the trial of the case, which question shall be decided by the members of the court other than the one challenged.
\item
  Pending the trial of a case, any member of the court who shall express his opinion of its merits to either party, or to any person not a member of the court; or who shall absent himself from any sitting without the permission of the court, or satisfactory reasons rendered, shall be thereby disqualified from taking part in the subsequent proceedings.
\item
  Minutes of the trial shall be kept by the Clerk, which shall exhibit the charges, the answer, all the testimony, and all such acts, orders, and decisions of the court relating to the case, as either party may desire, and also the judgment. The Clerk shall, without delay, assemble the Record of the Case which shall consist of the charges, the answer, the citations and returns thereto, and the minutes herein required to be kept. The parties shall be allowed copies of the Record of the Case at their own expense when they demand them. When a case is removed by appeal or complaint, the lower court shall transmit ``the record'' thus prepared to the higher court with the addition of the notice of appeal or complaint, and the reasons thereof, if any shall have been filed. Nothing which is not contained in this ``record'' shall be taken into consideration in the higher court. On the final decision of a case in a higher court, its judgment shall be sent down to the court in which the case originated.
\item
  No professional counsel shall be permitted as such to appear and plead in cases of process in any court; but an accused person may, if he desires it, be represented before the Session by any communing member of the same particular church; or before any other court, by any member of the court. A member of the court so employed shall not be allowed to sit in judgment in the case.
\item
  Process, in case of scandal, shall commence within the space of one year after the offense was committed or from the court's discovery of the offense. When, however, a church member shall commit an offense, after removing to a place far distant from his former residence, and where his connection with the church is unknown, in consequence of which circumstances process cannot be instituted within the time above specified, the recent discovery of the church membership of the individual shall be considered as equivalent to the recent discovery of the offense itself. The same principle, in like circumstances, shall also apply to Ministers.
\end{enumerate}

\hypertarget{special-rules-pertaining-to-process-before-sessions}{%
\section*{36. Special Rules Pertaining to Process before Sessions}\label{special-rules-pertaining-to-process-before-sessions}}
\addcontentsline{toc}{section}{36. Special Rules Pertaining to Process before Sessions}

\protect\hypertarget{chapter-slug-36-special-rules-pertaining-to-process-before-sessions}{\href{}{}}

\begin{enumerate}
\def\labelenumi{\arabic{enumi}.}
\tightlist
\item
  \protect\hypertarget{36}{\href{}{}}Process against all church members, other than Ministers of the Gospel, shall be entered before the Session of the church to which such members belong, except in cases of appeal. However, if the Session refuses to act in doctrinal cases or instances of public scandal and two other Sessions of churches in the same Presbytery request the Presbytery of which the church is a member to initiate proper or appropriate action in a case of process and thus assume jurisdiction and authority, the Presbytery shall do so.
\item
  \protect\hypertarget{36.2}{\href{}{}}When an accused person is found contumacious, he shall be immediately suspended from the sacraments (and if an officer from his office) for his contumacy. Record shall be made of that fact and of the charges under which he was arraigned, and the censure may be made public, should this be deemed expedient by the Session. The censure shall in no case be removed until the offender has not only repented of his contumacy, but has also given satisfaction in relation to the charges against him.
\item
  \protect\hypertarget{36.3}{\href{}{}}If after further endeavor by the court to bring the accused to a sense of his guilt, he persists in his contumacy, he shall be excommunicated from the Church.
\item
  When it is not feasible immediately to commence process against an accused church member, the Session may, if it thinks the edification of the Church requires it, prevent the accused from approaching the Lord's table until the charges against him can be examined.
\end{enumerate}

\hypertarget{special-rules-pertaining-to-process-against-a-minister}{%
\section*{37. Special Rules Pertaining to Process against a Minister}\label{special-rules-pertaining-to-process-against-a-minister}}
\addcontentsline{toc}{section}{37. Special Rules Pertaining to Process against a Minister}

\protect\hypertarget{chapter-slug-37-special-rules-pertaining-to-process-against-a-minister}{\href{}{}}

\begin{enumerate}
\def\labelenumi{\arabic{enumi}.}
\item
  \protect\hypertarget{37}{\href{}{}}Process against a Minister shall be entered before the Presbytery of which he is a member.
\item
  As no Minister ought, on account of his office, to be screened in his sin, or slightly censured, so scandalous charges ought not to be received against him on slight grounds.
\item
  If anyone knows a Minister to be guilty of a private offense, he should warn him in private. But if the offense be persisted in, or become public, he should bring the case to the attention of some other Minister of the Presbytery for his advice.
\item
  \protect\hypertarget{37.4}{\href{}{}}When a minister accused of an offense is found contumacious, he shall be immediately suspended from the sacraments and his office for his contumacy. Record shall be made of the fact and of the charges under which he was arraigned, and the censure shall be made public. The censure shall in no case be removed until the offender has not only repented of his contumacy, but has also given satisfaction in relation to the charges against him.

  If after further endeavor by the court to bring the accused to a sense of his guilt, he persists in his contumacy, he shall be deposed and excommunicated from the Church.
\item
  Heresy and schism may be of such a nature as to warrant deposition; but errors ought to be carefully considered, whether they strike at the vitals of religion, and are industriously spread, or whether they arise from the weakness of the human understanding, and are not likely to do much injury.
\item
  If the Presbytery finds on trial that the matter complained of amounts to no more than such acts of infirmity as may be amended, so that little or nothing remains to hinder the Minister's usefulness, it shall take all prudent measures to remove the scandal.
\item
  When a Minister, pending a trial, shall make confession, if the matter be base, flagitious, and of a disqualifying nature, however penitent he may appear to the satisfaction of all, the court shall, without delay, impose definite suspension or depose him from the ministry.
\item
  \protect\hypertarget{37.8}{\href{}{}}A Minister under indefinite suspension from his office or deposed for scandalous conduct shall not be restored, even on the deepest sorrow for his sin, until he shall exhibit for a considerable time such an eminently exemplary, humble, and edifying walk and testimony as shall heal the wound made by his scandal. A deposed Minister shall in no case be restored until it shall appear that the general sentiment of the Church is strongly in his favor, and demands his restoration; and then only by the court inflicting the censure, or with its consent. The removal of deposition requires a three-fourths (3/4) vote of the court inflicting the censure, or a three-fourths (3/4) vote of the court to which the majority of the original court delegates that authority.
\item
  When a Minister is deposed his pastoral relation shall be dissolved; but when he is suspended from office, it shall be left to the discretion of the Presbytery whether the censure shall include the dissolution of the pastoral relation.
\item
  \protect\hypertarget{37.10}{\href{}{}}Whenever a Minister of the Gospel shall habitually fail to be engaged in the regular discharge of his official functions, it shall be the duty of the Presbytery, at a stated meeting, to inquire into the cause of such dereliction, and if necessary, to institute judicial proceedings against him for breach of his covenant engagement. If it shall appear that his neglect proceeds only from his lack of acceptance by the church (i.e., the people do not accept him), Presbytery may, upon the same principle upon which it withdraws license from a licentiate for want of evidence of the divine call, divest him of his office without censure, even against his will, a majority of two-thirds being necessary for this purpose.

  In such a case, the Clerk shall, under the order of the Presbytery, forthwith deliver to the individual concerned a written notice that, at the next stated meeting, the question of his being so dealt with is to be considered. This notice shall distinctly state the grounds for this proceeding. The party thus notified shall be heard in his own defense; and if the decision pass against him he may appeal, as if he had been tried after the usual forms.

  This principle may apply, with any necessary changes, to Ruling Elders and Deacons.
\end{enumerate}

\hypertarget{evidence}{%
\section*{38. Evidence}\label{evidence}}
\addcontentsline{toc}{section}{38. Evidence}

\protect\hypertarget{chapter-slug-38-evidence}{\href{}{}}

\begin{enumerate}
\def\labelenumi{\arabic{enumi}.}
\tightlist
\item
  \protect\hypertarget{38}{\href{}{}}All persons of proper age and intelligence are competent witnesses, except such as do not believe in the existence of God, or a future state of rewards and punishments. The accused party may be allowed, but shall not be compelled, to testify; but the accuser shall be required to testify, on the demand of the accused. Either party has the right to challenge a witness whom he believes to be incompetent, and the court shall examine and decide upon his competency. It belongs to the court to judge of the degree of credibility to be attached to all evidence.
\item
  A husband or wife shall not be compelled to bear testimony the one against the other in any court.
\item
  The testimony of more than one witness shall be necessary in order to establish any charge; yet if, in addition to the testimony of one witness, corroborative evidence be produced, the offense may be considered to be proved.
\item
  No witness afterwards to be examined, except a member of the court, shall be present during the examination of another witness on the same case, if either party object.
\item
  Witnesses shall be examined first by the party introducing them; then cross-examined by the opposite party; after which any member of the court, or either party, may put additional interrogatories. But no question shall be put or answered except by permission of the Moderator, subject to an appeal to the court; and the court shall not permit questions frivolous or irrelevant to the charge at issue.
\item
  The oath or affirmation to a witness shall be administered by the Moderator in the following or like terms: ``Do you solemnly swear, in the presence of God, that you will declare the truth, the whole truth, and nothing but the truth, according to the best of your knowledge in the matter in which you are called to witness, as you shall answer it to the great Judge of the living and the dead?'' If, however, at any time a witness should present himself before a court, who for conscientious reasons prefers to swear or affirm in any other manner, he should be allowed to do so.
\item
  All testimony shall be recorded (transcription, audio or video recording, or some other electronic means) and witnesses informed of such prior to testifying. Such recording becomes part of the Record of the Case. However, in order to be referenced in written or oral briefs, such recording must be transcribed and the transcription authenticated by the trial court. The court of final appeal may assess the cost of transcription equitably among the parties.
\item
  The records of a court, or any part of them, whether original or transcribed, if regularly authenticated by the Moderator and Clerk, or by either of them, shall be deemed good and sufficient evidence in every other court.
\item
  In like manner, testimony taken by one court, and regularly certified, shall be received by every other court as no less valid than if it had been taken by itself.
\item
  When it is not convenient for a court to have the whole, or perhaps any part of the testimony in any particular case, taken in its presence, a commission shall be appointed, or coordinate court, as described in BCO \protect\hyperlink{35.8}{35.8}, requested, to take the testimony in question, which shall be considered as if taken in the presence of the court; of which commission or coordinate court, and of the time and place of its meeting, due notice shall be given to the opposite party, that he may have an opportunity of attending. And if the accused shall desire, on his part, to take testimony at a distance, for his own exculpation, he shall give notice to the court of the time and place at which it is proposed to take it that a commission or coordinate court, as in the former case, may be appointed for the purpose. Or the testimony may be taken on written interrogatories, by filing the same with the Clerk of the court having jurisdiction of the case, and giving two weeks' notice thereof to the adverse party, during which time he may file cross-interrogatories, if he desire it; and the testimony shall then be taken by the commission or coordinate court in answer to the direct and cross-interrogatories, if such are filed, and no notice need be given of the time and place of taking the testimony.
\item
  A member of the court who has given testimony in a case becomes disqualified for sitting as a judge if either party makes objection.
\item
  An officer or private member of the church refusing to testify may be censured for contumacy.
\item
  If after trial before any court new testimony be discovered which the accused believes important, it shall be his right to ask a new trial and it shall be within the power of the court to grant his request.
\item
  \protect\hypertarget{38.14}{\href{}{}}If in the prosecution of an appeal, new evidence is offered, which, in the judgment of the appellate court, has an important bearing on the case, it shall be competent for that court to refer the case to the lower court for a new trial; or, with the consent of parties, to admit the evidence and proceed with the case.
\end{enumerate}

\hypertarget{the-infliction-of-church-censures}{%
\section*{39. The Infliction of Church Censures}\label{the-infliction-of-church-censures}}
\addcontentsline{toc}{section}{39. The Infliction of Church Censures}

\protect\hypertarget{chapter-slug-39-the-infliction-of-church-censures}{\href{}{}}

\begin{enumerate}
\def\labelenumi{\arabic{enumi}.}
\item
  \protect\hypertarget{39}{\href{}{}}When any member or officer of the church shall be found guilty of an offense the court shall proceed with all tenderness and shall deal with its offending brother in the spirit of meekness, the members considering themselves lest they also be tempted.
\item
  Church censures and the modes of administering them should be suited to the nature of the offenses; for private offenses censure should be administered in the presence of the court alone, or in private by one or more members of the court, but in the case of public offenses, the degree of censure and mode of administering it shall be within the discretion of the court, acting in accordance with paragraphs below which deal with particular censures.
\item
  The censure of admonition should be administered in private by one or more members of the court if the offense is known only to a few and is not aggravated in character. If the offense is public the admonition should be administered by the Moderator in presence of the court and may also be announced in public should the court deem it expedient.
\item
  Definite suspension from Sacraments or office should be administered in the presence of the court alone, or in open session of the court, as it may deem best, and public announcement thereof shall be at the court's discretion.
\item
  Indefinite suspension from Sacraments or office should be administered after the manner prescribed for definite suspension, but with added solemnity, that it may be the means of impressing the mind of the delinquent with a proper sense of his danger, and under the blessing of God of leading him to repentance. When the court has resolved to pass this sentence, the Moderator shall address the offending brother to the following purpose:

  \begin{quote}
  ``Whereas, You, \_\_\_\_\_\_\_\_\_\_\_\_\_\_\_\_\_\_\_\_\_\_\_(here describe the person as a Minister, Ruling Elder, Deacon, or private member of the church), are convicted by sufficient proof {[}or, are guilty by your own confession{]}, of the sin of \_\_\_\_\_\_\_\_\_\_\_\_\_\_\_\_\_\_\_\_\_\_\_\_\_\_\_\_\_\_\_\_(here insert the offense), we, the \_\_\_\_\_\_\_\_\_\_\_\_\_\_\_\_\_\_\_\_\_\_ Presbytery {[}or church Session{]} , in the name and by the authority of the Lord Jesus Christ, do now declare you suspended from the sacraments of the Church {[}and from the exercise of your office{]}, until you give satisfactory evidence of repentance.''
  \end{quote}

  To this shall be added such advice or admonition as may be judged necessary, and the whole shall be concluded with prayer to Almighty God that He would follow this act of discipline with His blessing.
\item
  Excommunication shall be administered with a public announcement to the church. In administering this censure the Moderator or a member of the Session shall make a statement of the several steps which have been taken with respect to the offending brother, and of the decision to cut him off from the communion of the Church. He shall then show from Matthew 18:15-18, and I Cor. 5:1-5, the authority of the Church to cast out unworthy members, and shall explain the nature, use, and consequences of this censure. He shall then administer the censure in the following or like words:

  \begin{quote}
  ``Whereas, \_\_\_\_\_\_\_\_\_\_\_\_\_\_\_\_\_\_\_\_, a member of this church, has been, by sufficient proof, convicted of the sin of \_\_\_\_\_\_\_\_\_\_\_\_\_\_\_\_\_\_\_\_\_\_\_\_\_\_\_, and after much admonition and prayer, obstinately refuses to hear the Church, and has manifested no evidence of repentance: Therefore, in the name and by the authority of the Lord Jesus Christ, we, the Session of \_\_\_\_\_\_\_\_\_\_\_\_\_\_\_\_\_\_\_\_., do pronounce him to be excluded from the sacraments, and cut off from the fellowship of the Church.''
  \end{quote}

  Prayer shall then be made that by God's blessing this solemn action of the court may issue in the repentance and restoration of the offender, and in the establishment of all true believers. The members of the church shall be instructed in how to relate to the excommunicated member as defined by Scripture.\footnote{1 Corinthians 5:9-13; Matthew 18:17.}
\end{enumerate}

\begin{enumerate}
\def\labelenumi{\arabic{enumi}.}
\setcounter{enumi}{6}
\item
  The censure of deposition shall be administered by the Moderator in the following or like words:

  \begin{quote}
  ``Whereas, \_\_\_\_\_\_\_\_\_\_\_\_\_\_\_\_\_\_\_\_\_\_\_\_, a Minister of this Presbytery {[}or a Ruling Elder or Deacon of this church{]}, has been proved, by sufficient evidence to be guilty of the sin of \_\_\_\_\_\_\_\_\_\_\_\_\_\_\_\_\_\_\_\_\_\_\_\_\_\_\_\_, we, the Presbytery {[}or church Session{]}, of \_\_\_\_\_\_\_\_\_\_\_\_\_\_\_\_\_\_\_\_\_\_\_\_\_\_\_, do adjudge him disqualified for the office of the Christian ministry {[}or Ruling Eldership, or Deaconship{]}, and therefore we do hereby, in the name and by the authority of the Lord Jesus Christ, depose from the office of a Christian Minister {[}or Ruling Elder, or Deacon{]}, the said \_\_\_\_\_\_\_\_\_\_\_\_\_\_\_\_\_\_, and do prohibit him from exercising any of the functions thereof.''
  \end{quote}

  If the censure include suspension or excommunication, the Moderator shall proceed to say:

  \begin{quote}
  ``We do moreover, by the same authority, suspend the said \_\_\_\_\_\_\_\_\_\_\_\_\_\_\_\_\_\_\_, from the sacraments of the Church, until he shall exhibit satisfactory evidence of sincere repentance,'' or ``exclude the said \_\_\_\_\_\_\_\_\_\_\_\_\_\_\_\_\_\_\_\_\_\_, from the sacraments, and cut him off from the fellowship of the Church.''
  \end{quote}

  The sentence of deposition ought to be inflicted with solemnities similar to those already prescribed in the case of excommunication.
\end{enumerate}

\hypertarget{the-removal-of-censure}{%
\section*{40. The Removal of Censure}\label{the-removal-of-censure}}
\addcontentsline{toc}{section}{40. The Removal of Censure}

\protect\hypertarget{chapter-slug-40-the-removal-of-censure}{\href{}{}}

\begin{enumerate}
\def\labelenumi{\arabic{enumi}.}
\item
  \protect\hypertarget{40}{\href{}{}}A person who has been definitely suspended from Sacraments or office shall be restored by the court at the end of the term of his suspension by declaring words of the following import to him:

  \begin{quote}
  ``Whereas, You, \_\_\_\_\_\_\_\_\_\_\_\_\_\_\_\_\_\_\_\_, have been debarred from the Sacraments of the Church {[}and/or from the office of Minister or Ruling Elder or Deacon{]}, but have now fulfilled the time of your censure, we, of the \_\_\_\_\_\_\_\_\_\_\_\_\_\_\_\_\_\_ {[}Presbytery \emph{or}~Church Session{]} do hereby, in the name and by the authority of the Lord Jesus Christ, absolve you from the sentence of suspension and do restore you to participation in the Sacraments {[}and/or~the exercise of your said office, and all the functions thereof{]}.''
  \end{quote}
\item
  After any person has been indefinitely suspended from the sacraments, it is proper that the rulers of the church should frequently converse with him as well as pray with him and for him, that it would please God to give him repentance.
\item
  When the court shall be satisfied as to the reality of the repentance of an indefinitely suspended offender, he shall be admitted to profess his repentance, either in the presence of the court alone, or publicly, and be restored to the sacraments of the Church, and, if appropriate, to his office, if such be the judgment of the court, which restoration shall be declared to the penitent in words of the following import:

  \begin{quote}
  ``Whereas, You, \_\_\_\_\_\_\_\_\_\_\_\_\_, have been debarred from the sacraments of the Church {[}and/or from the office of the Gospel Ministry \emph{or}~Ruling Eldership \emph{or}~Deaconship{]}, but have now manifested such repentance as satisfies the church, we, the Session {[}\emph{or} Presbytery{]} of \_\_\_\_\_\_\_\_\_\_\_\_\_\_\_\_, do hereby, with great joy, in the name and by the authority of the Lord Jesus Christ absolve you from the said sentence of suspension from the Sacraments {[}and/or your office{]}, and do restore you to the full communion of the Church {[}and/or the exercise of your said office, and all the functions thereof{]}.''
  \end{quote}

  After which there shall be prayer and thanksgiving.
\item
  When an excommunicated person shall be so affected with his state as to be brought to repentance, and to desire to be readmitted to the communion of the Church, the Session, having obtained sufficient evidence of his sincere penitence, shall proceed to restore him. This may be done in the presence of the court, or of the congregation as seems best to the Session.

  On the day appointed for his restoration, the Minister shall call upon the excommunicated person, and propose to him in the presence of the court, or of the congregation, the following questions:

  \begin{quote}
  ``Do you, from a deep sense of your great wickedness, freely confess your sin in thus rebelling against God, and in refusing to hear His Church; and do you acknowledge that you have been in justice and mercy cut off from the communion of the Church? Answer.---I do.''

  ``Do you now voluntarily profess your sincere repentance and contrition for your sin and obstinacy; and do you humbly ask the forgiveness of God and His Church? Answer.---I do.''

  ``Do you sincerely promise, through divine grace, to live in all humbleness of mind and circumspection; and to endeavor to adorn by a holy life the doctrine of God our Savior? Answer.---I do.''
  \end{quote}

  Here the Minister shall give the penitent a suitable exhortation, encouraging and comforting him. Then he shall pronounce the sentence of restoration in the following or like words:

  \begin{quote}
  ``Whereas, You, \_\_\_\_\_\_\_\_\_\_\_\_\_\_\_, have been shut out from the communion of the Church, but have now manifested such repentance as satisfies the Church; in the name of the Lord Jesus Christ, and by His authority, we, the Session of this church, do declare you absolved from the sentence of excommunication formerly pronounced against you; and with great joy we do restore you to the communion of the Church, that you may be a partaker of all the benefits of the Lord Jesus to your eternal salvation.''
  \end{quote}

  The whole shall be concluded with prayer and thanksgiving.
\item
  The restoration of a deposed officer, after public confession has been made in a manner similar to that prescribed in the case, of the removal of censure from an excommunicated person, shall be announced to him by the Moderator in the following form, namely:

  \begin{quote}
  ``Whereas, You, \_\_\_\_\_\_\_\_\_\_\_\_\_\_, formerly a Minister of this Presbytery {[}or a Ruling Elder, or Deacon of this church{]}, have been deposed from your office, but have now manifested such repentance as satisfies the Church; in the name of the Lord Jesus Christ, and by His authority, we, the Presbytery of \_\_\_\_\_\_\_\_\_\_\_\_\_\_\_\_\_\_, {[}or the Session of this church{]}, do with great joy declare you absolved from the said sentence of deposition formerly pronounced against you; and we do furthermore restore you to your said office, and to the exercise of all the functions thereof, whenever you may be orderly called thereto.''
  \end{quote}

  After which there shall be prayer and thanksgiving, and the members of the court shall extend to him the right hand of fellowship.
\item
  When a Ruling Elder or Deacon has been absolved from the censure of deposition, he cannot be allowed to resume the exercise of his office in the church without re-election by the people. The removal of deposition requires a three-fourths (3/4) vote of the court inflicting the censure, or a three-fourths (3/4) vote of the court to which the majority of the original court delegates that authority.
\item
  When a person under censure shall remove to a part of the country remote from the court by which he was sentenced, and shall desire to profess repentance and obtain restoration, it shall be lawful for the court, if it deems it expedient, to transmit a certified copy of its proceedings to the Session (or Presbytery) where the delinquent resides, which shall take up the case and proceed with it as though it had originated with itself.
\item
  In the restoration of a Minister who is under indefinite suspension from the Sacraments, and/or his office, or has been deposed, it is the duty of the Presbytery to proceed with great caution. It should first admit him to the sacraments, if he has been debarred from them, and afterwards should grant him the privilege of preaching on probation for a time, so as to test the sincerity of his repentance and the prospect of his usefulness, and when satisfied in these respects the Presbytery shall take steps to restore him to his office. But the case shall always be under judicial consideration until the sentence of restoration has been pronounced.
\item
  In the case of the removal of censures from, or the restoration of, a minister, jurisdiction shall be as follows:

  \begin{enumerate}
  \def\labelenumii{\alph{enumii}.}
  \tightlist
  \item
    If the censure(s) does not include excommunication, the presbytery inflicting the censure(s) shall retain the authority to remove the censure(s) and, at its discretion, restore him to office. This authority is retained by the presbytery even when a divested or deposed minister is assigned, under the provisions of BCO \protect\hyperlink{49.7}{49.7}, to a session.
  \item
    If the censure includes excommunication, the penitent may only be restored to the communion of the church through a session (BCO \protect\hyperlink{1.3}{1.3}; \protect\hyperlink{7}{7}; \protect\hyperlink{62}{62}, \protect\hyperlink{63}{63}). Once the penitent is restored, and therefore a member of a local church, the authority to remove any other censure(s) in respect to office, concurrently imposed with that of excommunication shall belong to the court originally imposing such censure(s).
  \end{enumerate}
\end{enumerate}

\hypertarget{cases-without-process}{%
\section*{41. Cases without Process}\label{cases-without-process}}
\addcontentsline{toc}{section}{41. Cases without Process}

\protect\hypertarget{chapter-slug-41-cases-without-process}{\href{}{}}

\begin{enumerate}
\def\labelenumi{\arabic{enumi}.}
\item
  \protect\hypertarget{41}{\href{}{}}When any person shall come forward and make his offense known to the court, a full statement of the facts shall be recorded and judgment rendered without process.
\item
  A Minister of the Gospel against whom there are no charges, if fully satisfied in his own conscience that God has not called him to the ministry, or if he has satisfactory evidence of his inability to serve the Church with acceptance, may report these facts at a stated meeting of Presbytery. At the next stated meeting, if after full deliberation the Presbytery shall concur with him in judgment, it may divest him of his office without censure. This provision shall in like manner apply with any necessary changes to the case of Ruling Elders and Deacons; but in all such cases the Session of the church to which the Ruling Elder or the Deacon who seeks demission belongs shall act as the Presbytery acts in similar cases where a Minister is concerned.
\item
  When a member or officer shall renounce the communion of this Church by joining some other branch of the visible Church, if in good standing, the irregularity shall be recorded, his new membership acknowledged, and his name erased from the roll. But if there is a record of an investigation in process, or there are charges concerning the member or minister, the court of original jurisdiction may retain his name on the roll and conduct the case, communicating the outcome upon completion of the proceedings to that member or minister. If the court does not conduct the case, his new membership shall be acknowledged, his name removed from the roll, and, at the request of the receiving branch, the matters under investigation or the charges shall be communicated to them.

  When a member or minister of this Church shall attempt to withdraw from the communion of this branch of the visible Church by affiliating with a body judged by the court of original jurisdiction as failing to maintain the Word and Sacraments in their fundamental integrity, that member or minister shall be warned of his danger and if he persists, his name shall be erased from the roll, thereby, so far as this Church is concerned, he is deemed no longer to be a member in any body which rightly maintains the Word and Sacraments in their fundamental integrity, and if an officer, thereby withdrawing from him all authority to exercise his office as derived from this Church. When so acting the court shall make full record of the matter and shall notify the offender of its action.
\item
  \protect\hypertarget{41.4}{\href{}{}}When a member of a particular church has willfully neglected the church for an extended period of time, or has made it known that he has no intention of fulfilling the church vows, then the Session, continuing to exercise pastoral discipline (BCO \protect\hyperlink{30.1.a}{30.1.a} and BCO \protect\hyperlink{30.4}{30.4}) in the spirit of Galatians 6:1, shall remind the member, if possible both in person and in writing, of the declarations and promises by which he entered into a solemn covenant with God and His Church (BCO \protect\hyperlink{62.2.5}{62.2, questions 5--6}), and warn him that, if he persists, his name shall be erased from the roll.

  If after diligently pursuing such pastoral discipline, and after further inquiry and due delay, the Session is of the judgment that the member will not fulfill his membership obligations in this or any other branch of the Visible Church (cf.~BCO \protect\hyperlink{3.2}{3.2}), then the Session shall erase his name from the roll. This erasure is an act of pastoral discipline (BCO \protect\hyperlink{30.1.a}{30.1.a}) without process. The Session shall notify the person, if possible, whose name has been removed. Notwithstanding the above, if a member thus warned makes a written request for process (see BCO \protect\hyperlink{34}{34--36}; \protect\hyperlink{38}{38--39}), the Session shall grant such a request. Further, if the Session determines that any offense of such a member is of the nature that process is necessary, the Session may institute such process.
\end{enumerate}

\hypertarget{modes-in-which-the-proceedings-of-lower-courts-come-under-the-supervision-of-higher-courts}{%
\section*{42. Modes in Which the Proceedings of Lower Courts Come Under the Supervision of Higher Courts}\label{modes-in-which-the-proceedings-of-lower-courts-come-under-the-supervision-of-higher-courts}}
\addcontentsline{toc}{section}{42. Modes in Which the Proceedings of Lower Courts Come Under the Supervision of Higher Courts}

\protect\hypertarget{chapter-slug-42-modes-in-which-the-proceedings-of-lower-courts-come-under-the-supervision-of-higher-courts}{\href{}{}}

\begin{enumerate}
\def\labelenumi{\arabic{enumi}.}
\tightlist
\item
  \protect\hypertarget{42}{\href{}{}}The acts and decisions of a lower court are brought under the supervision of a higher court in one or another of the following modes:

  \begin{enumerate}
  \def\labelenumii{\alph{enumii}.}
  \tightlist
  \item
    Review and Control;
  \item
    Reference;
  \item
    Appeal;
  \item
    Complaint.
  \end{enumerate}
\item
  When the proceedings of a lower court are before a higher court the members of the lower court shall not lose the right to sit, deliberate, and vote in the higher court, except in cases of appeal or complaint.
\item
  While affirming that the Scripture is ``the supreme judge by which all controversies of religion are to be determined,''\footnote{\href{https://evangelpresbytery.com/westminster-confession-of-faith/\#1.10}{WCF 1.10}.} and that the Constitution of Evangel Presbytery is ``subordinate to the Scriptures of the Old and New Testaments, the infallible Word of God'' (BCO \protect\hyperlink{29.1}{29.1}), and while affirming also that this Constitution is fallible,\footnote{\href{https://evangelpresbytery.com/westminster-confession-of-faith/\#31.3}{WCF 31.3}.} Evangel Presbytery affirms that this subordinate and fallible Constitution has been ``adopted by the church'' (BCO \protect\hyperlink{29.1}{29.1}) ``as standard expositions of the teachings of Scripture in relation to both faith and practice'' (BCO \protect\hyperlink{32.1}{32.1}) and as setting forth a form of government and discipline ``in conformity with the general principles of biblical polity'' (BCO \protect\hyperlink{23.6.c}{23.6.c}). To insure that this Constitution is not amended, violated or disregarded in judicial process, any review of the judicial proceedings of a lower court by a higher court shall be guided by the following principles:

  \begin{enumerate}
  \def\labelenumii{\alph{enumii}.}
  \tightlist
  \item
    A higher court, reviewing a lower court, should limit itself to the issues raised by the parties to the case in the original (lower) court. Further, the higher court should resolve such issues by applying the Constitution of the church, as previously established through the constitutional process.
  \item
    A higher court should ordinarily exhibit great deference to a lower court regarding those factual matters which the lower court is more competent to determine, because of its proximity to the events in question, and because of its personal knowledge and observations of the parties and witnesses involved. Therefore, a higher court should not reverse a factual finding of a lower court, unless there is clear error on the part of the lower court.
  \item
    A higher court should ordinarily exhibit great deference to a lower court regarding those matters of discretion and judgment which can only be addressed by a court with familiar acquaintance of the events and parties. Such matters of discretion and judgment would include, but not be limited to: the moral character of candidates for sacred office, the appropriate censure to impose after a disciplinary trial, or judgment about the comparative credibility of conflicting witnesses. Therefore, a higher court should not reverse such a judgment by a lower court, unless there is clear error on the part of the lower court.
  \item
    The higher court does have the power and obligation of judicial review, which cannot be satisfied by always deferring to the findings of a lower court. Therefore, a higher court should not consider itself obliged to exhibit the same deference to a lower court when the issues being reviewed involve the interpretation of the Constitution of the Church. Regarding such issues, the higher court has the duty and authority to interpret and apply the Constitution of the Church according to its best abilities and understanding, regardless of the opinion of the lower court.
  \end{enumerate}
\end{enumerate}

\hypertarget{general-review-and-control}{%
\section*{43. General Review and Control}\label{general-review-and-control}}
\addcontentsline{toc}{section}{43. General Review and Control}

\protect\hypertarget{chapter-slug-43-general-review-and-control}{\href{}{}}

\begin{enumerate}
\def\labelenumi{\arabic{enumi}.}
\tightlist
\item
  \protect\hypertarget{43}{\href{}{}}It is the right and duty of every court above the Session to review, at least once a year, the records of the court next below, and if any lower court fails to present its records for this purpose, the higher court may require them to be produced immediately, or at any time fixed by this higher court.
\item
  In reviewing records of a lower court the higher court is to examine:

  \begin{enumerate}
  \def\labelenumii{\alph{enumii}.}
  \tightlist
  \item
    Whether the proceedings have been correctly recorded;
  \item
    whether they have been regular and in accordance with the Constitution;
  \item
    whether they have been wise, equitable, and suited to promote the welfare of the Church;
  \item
    whether the lawful injunctions of the higher court have been obeyed.
  \end{enumerate}
\item
  It is ordinarily sufficient for the higher court merely to record in its own minutes and in the records reviewed, whether it approves, disapproves, or corrects the records in any particular; but should any serious irregularity be discovered the higher court may require its review and correction by the lower. Proceedings in judicial cases, however, shall not be dealt with under review and control when notice of appeal or complaint has been given the lower court; and no judgment of a lower court in a judicial case shall be reversed except by appeal or complaint.
\item
  Courts may sometimes entirely neglect to perform their duty, by which neglect heretical opinions or corrupt practices may be allowed to gain ground; or offenders of a very gross character may be suffered to escape; or some circumstances in their proceedings of very great irregularity may not be distinctly recorded by them; in any of which cases their records will by no means exhibit to the higher court a full view of their proceedings. If, therefore, the next higher court be well advised that any such neglect or irregularity has occurred on the part of the lower court, it is incumbent on it to take cognizance of the same, and to examine, deliberate, and judge in the whole matter as completely as if it had been recorded, and thus brought up by the review of the records.
\item
  When any court having appellate jurisdiction shall be advised, either by the records of the court next below or by memorial, either with or without protest, or by any other satisfactory method, of any important delinquency or grossly unconstitutional proceedings of such court, the first step shall be to cite the court alleged to have offended to appear by representative or in writing, at a specified time and place, and to show what it has done or failed to do in the case in question. The court thus issuing the citation may reverse or redress the proceedings of the court below in other than judicial cases; or it may censure the delinquent court; or it may remit the whole matter to the delinquent court, with an injunction to take it up and dispose of it in a constitutional manner; or it may stay all further proceedings in the case; as circumstances may require.
\item
  In process against a lower court, the trial shall be conducted according to the rules provided for process against individuals, so far as they may be applicable.
\end{enumerate}

\hypertarget{references}{%
\section*{44. References}\label{references}}
\addcontentsline{toc}{section}{44. References}

\protect\hypertarget{chapter-slug-44-references}{\href{}{}}

\begin{enumerate}
\def\labelenumi{\arabic{enumi}.}
\tightlist
\item
  \protect\hypertarget{44}{\href{}{}}A reference is a written representation and application made by a lower court to a higher for advice, or other action, on a matter pending before the lower court, and is ordinarily to be made to the next higher court.
\item
  Among proper subjects for reference are matters that are new, delicate, or difficult; or on which the members of the lower court are very seriously divided; or which relate to questions involving the Constitution and legal procedure respecting which the lower court feels the need of guidance.
\item
  \protect\hypertarget{44.3}{\href{}{}}In making a reference the lower court may ask for advice only, or for final disposition of the matter referred; and in particular it may refer a judicial case with request for its trial and decision by the higher court.
\item
  A reference may be presented to the higher court by one or more representatives appointed by the lower court for this purpose, and it should be accompanied with so much of the record as shall be necessary for proper understanding and consideration of the matter referred.
\item
  Although references are sometimes proper, yet in general it is better that every court should discharge the duty assigned it under the law of the Church. A higher court is not required to accede to the request of the lower, but it should ordinarily give advice when so requested.
\item
  When a court makes a reference, it ought to have all the testimony and other documents duly prepared, produced, and in perfect readiness, so that the higher court may be able to fully consider and issue the case with as little difficulty or delay as possible.
\end{enumerate}

\hypertarget{appeals}{%
\section*{45. Appeals}\label{appeals}}
\addcontentsline{toc}{section}{45. Appeals}

\protect\hypertarget{chapter-slug-45-appeals}{\href{}{}}

\begin{enumerate}
\def\labelenumi{\arabic{enumi}.}
\tightlist
\item
  \protect\hypertarget{45}{\href{}{}}An appeal is the transfer to a higher court of a judicial case on which judgment has been rendered in a lower court, and is allowable only to the party against whom the decision has been rendered. The parties shall be known as the appellant and appellee. An appeal cannot be made to any court other than the next higher, except with its consent.
\item
  Only those who have submitted to a regular trial are entitled to an appeal. Those who have not submitted to a regular trial are not entitled to an appeal.
\item
  The grounds of appeal are such as the following: any irregularity in the proceedings of the lower court; refusal of reasonable indulgence to a party on trial; receiving improper, or declining to receive proper, evidence; hurrying to a decision before all the testimony is taken; manifestation of prejudice in the case; and mistake or injustice in the judgment and censure.
\item
  Notice of appeal may be given the court before its adjournment. Written notice of appeal, with supporting reasons, shall be filed by the appellant with both the clerk of the lower court and the clerk of the higher court, within thirty days of notification of the last court's decision.Notification shall be deemed to have occurred on the day of mailing (if certified, registered or express mail of a national postal service or any private service where verifying receipt is utilized), the day of hand delivery, or the day of confirmed receipt in the case of e-mail or facsimile. Furthermore, compliance with such requirements shall be deemed to have been fulfilled if a party cannot be located after diligent inquiry or if a party refuses to accept delivery. No attempt should be made to circularize the courts to which appeal is being made by either party before the case is heard.
\item
  It shall be the duty of the clerk of the lower court to file with the clerk of the higher court, not more than thirty (30) days after receipt of notice of appeal, a copy of all proceedings in connection with the case, including the notice of appeal and reasons therefor, the response of the lower court, the evidence, and any papers bearing on the case, which together shall be known as ``the Record of the Case'', and the higher court shall not admit or consider anything not found in this ``Record'' without the consent of the parties in the case. Should new evidence come to light the case shall be remanded to the lower court from which the appeal was made, unless both parties consent to admit the new evidence and proceed with the case (cf.~BCO \protect\hyperlink{38.14}{38.14}).
\item
  Notice of appeal shall have the effect of suspending the judgment of the lower court until the case has been finally decided in the higher court. If, however, the censure is suspension from the Sacraments or excommunication, or deposition from office, the court may, for sufficient reasons duly recorded, put the censure into effect until the case is finally decided.
\item
  After a higher court has decided that an appeal is in order and should be entertained by the court, the court shall hear the case, or appoint a commission to do so. At the hearing, after the Record has been read, each side should be allotted not over thirty (30) minutes for oral argument, the appellant having the right of opening and closing the argument. After the hearing has been concluded, the court or commission should go into closed session, and discuss the merits of the case. The vote then should be taken, without further debate, on each specification in this form: Shall this specification of error be sustained? If the court or commission deem it wise, it may adopt a minute explanatory of its action, which shall become a part of its Record of the Case. The court or commission shall designate one of its members to write the opinion, which opinion shall be adopted by the court or commission as its opinion.
\item
  The decision of the higher court may be to confirm or to reverse, in whole or in part, the judgment of the lower court; or to remit the case to the lower court for the purpose of amending the record, should it appear incorrect or defective; or to send the case back for a new trial. In every case a full record shall be made, and a copy of it shall be sent to the lower court.
\item
  An appellant shall be considered to have abandoned his appeal if he fails to appear before the higher court, in person or by counsel, for a hearing thereof, after he has been properly notified; but an appellant may waive, in writing, his right to appear with permission of the court and not be considered to have abandoned his case. In case of such failure to appear, the judgment of the lower court will stand unless the appellant gives to the court a prompt and satisfactory explanation.
\item
  If an appellant manifests a litigious or otherwise unchristian spirit in the prosecution of his appeal, he shall receive a suitable rebuke by the appellate court.
\item
  If a lower court shall neglect to send up ``the record of the case,'' or any part of it, to the injury of the appellant, it shall receive a proper rebuke from the higher court, and the judgment from which the appeal has been taken shall be suspended, until ``the record'' is produced upon which the issue can be fairly tried.
\end{enumerate}

\hypertarget{complaints}{%
\section*{46. Complaints}\label{complaints}}
\addcontentsline{toc}{section}{46. Complaints}

\protect\hypertarget{chapter-slug-46-complaints}{\href{}{}}

\begin{enumerate}
\def\labelenumi{\arabic{enumi}.}
\tightlist
\item
  \protect\hypertarget{46}{\href{}{}}A complaint is a written representation made against some act or decision of a court of the Church. It is the right of any communing member of the Church in good standing to make complaint against any action of a court to whose jurisdiction he is subject, except that no complaint is allowable in a judicial case in which an appeal is pending.
\item
  A complaint shall first be made to the court whose act or decision is alleged to be in error. Written notice of complaint, with supporting reasons, shall be filed with the clerk of the court within sixty (60) days following the meeting of the court. The court shall consider the complaint at its next stated meeting, or at a called meeting prior to its next stated meeting. No attempt should be made to circularize the court to which complaint is being made by either party.
\item
  If, after considering a complaint, the court alleged to be delinquent or in error is of the opinion that it has not erred, and denies the complaint, the complainant may take that complaint to the next higher court. If the lower court fails to consider the complaint against it by or at its next stated meeting, the complainant may take that complaint to the next higher court. Written notice thereof shall be filed with both the clerk of the lower court and the clerk of the higher court within thirty (30) days of notification of the last court's decision. Notification shall be deemed to have occurred on the day of mailing (if certified, registered or express mail of a national postal service or any private service where verifying receipt is utilized), the day of hand delivery, or the day of confirmed receipt in the case of e-mail or facsimile. Furthermore, compliance with such requirements shall be deemed to have been fulfilled if a party cannot be located after diligent inquiry or if a party refuses to accept delivery.
\item
  Notice of complaint shall not have the effect of suspending the action against which the complaint is made, unless one-third (1/3) of the members present when the action was taken shall vote for its suspension, until the final decision in the higher court.
\item
  The court against which complaint is made shall appoint one or more representatives to defend its action before the higher court, and the parties in the case shall be known as complainant and respondent. The complainant himself may present his complaint, or he may obtain the assistance of a communing member of Evangel Presbytery, who is in good standing, in presenting his complaint.
\item
  It shall be the duty of the clerk of the lower court to file with the clerk of the higher court, not more than thirty (30) days after receipt of notice of complaint, a copy of all its proceedings in connection with the complaint including the notice of complaint and supporting reasons, the response of the lower court, if any, and any papers bearing on the complaint. If the clerk of the lower court shall neglect to send up the proceedings on the complaint, he shall receive a proper rebuke from the higher court, and the act or decision complained against shall be suspended until the proceedings are produced so that the higher court can fairly consider the complaint.
\item
  The complainant shall be considered to have abandoned his complaint if he fails to appear before the higher court, in person or by counsel, for a hearing thereof, after he has been properly notified; but a complainant may waive, in writing, his right to appear with permission of the court and not be considered to have abandoned his case. In case of such failure to appear, the judgment of the lower court will stand unless the complainant gives to the court a prompt and satisfactory explanation.
\item
  Subject to the provisions below, after the higher court has decided that the notice filed with its clerk was timely and that the complaint is otherwise in order for it to be heard by the higher court, it shall hear the complaint, or appoint a commission to do so. Ordinarily the court or its commission shall schedule a hearing in a manner that reasonably accommodates the schedules of the respective parties and affords each party a prior opportunity to file a written brief upon such terms and in accord with a briefing schedule established by the court or its commission in the reasonable exercise of its discretion.
\item
  At the hearing, after all the papers bearing on the complaint have been read, the complainant and respondent will be given the opportunity to present argument, the complainant having the right of opening and closing the argument. After the hearing has been concluded, the court or the commission should go into closed session, and discuss and consider the merits of the complaint. The vote should then or later be taken as to what disposition should be made of the complaint, and the complainant and respondent notified of the court's decision.
\item
  The higher court has power, in its discretion, to annul the whole or any part of the action of a lower court against which complaint has been made, or to send the matter back to the lower court with instructions for a new hearing. If the higher court rules a lower court erred by not indicting someone, and the lower court refers the matter back to the higher court, it shall accept the reference if it is a doctrinal case or case of public scandal (see BCO \protect\hyperlink{44.3}{44.3}).
\end{enumerate}

\hypertarget{voting-in-appeals-and-complaints}{%
\section*{47. Voting in Appeals and Complaints}\label{voting-in-appeals-and-complaints}}
\addcontentsline{toc}{section}{47. Voting in Appeals and Complaints}

\protect\hypertarget{chapter-slug-47-voting-in-appeals-and-complaints}{\href{}{}}

\begin{enumerate}
\def\labelenumi{\arabic{enumi}.}
\tightlist
\item
  \protect\hypertarget{47}{\href{}{}}In voting upon a complaint, the vote shall be either to sustain, to sustain in part, or not to sustain.
\item
  The effect of a vote to sustain shall be to sustain each and all of the items or counts of the complaint; that of a vote not to sustain shall be to annul each and all of the items or counts of the complaint; and that of a vote to sustain in part shall be to sustain one or more specific items or counts of the complaint.
\item
  Those voting to sustain in part shall be required when voting to state what item or items, count or counts of the complaint they desire to sustain.
\item
  In making up the vote on the complaint only those items or counts shall be declared to be sustained for the sustaining of which a majority of the votes cast has been given.
\end{enumerate}

\hypertarget{dissents-protests-and-objections}{%
\section*{48. Dissents, Protests, and Objections}\label{dissents-protests-and-objections}}
\addcontentsline{toc}{section}{48. Dissents, Protests, and Objections}

\protect\hypertarget{chapter-slug-48-dissents-protests-and-objections}{\href{}{}}

\begin{enumerate}
\def\labelenumi{\arabic{enumi}.}
\tightlist
\item
  \protect\hypertarget{48}{\href{}{}}Any member of a court who had a right to vote on a question, and is not satisfied with the action taken by that court, is entitled to have a dissent or protest recorded. None can join in a dissent or protest against an action of any court except those who had a right to vote in the case. Any member who did not have the right to vote on an appeal or complaint, and is not satisfied with the action taken by the court, is entitled to have an objection recorded. A dissent, protest, or objection shall be filed with the clerk of the lower court within thirty days following the meeting of the lower court.
\item
  A dissent is a declaration on the part of one or more members of a minority in a court, expressing a different opinion from that of the majority in a particular case. A dissent unaccompanied with reasons shall be entered on the records of the court.
\item
  A protest is a more solemn and formal declaration by members of a minority, bearing their testimony against what they deem a mischievous or erroneous judgment, and is generally accompanied with a detail of the reasons on which it is founded.
\item
  An objection is a declaration by one or more members of a court who did not have the right to vote on an appeal or complaint, expressing a different opinion from the decision of the court and may be accompanied with the reasons on which it is founded.
\item
  If a dissent, protest, or objection be couched in temperate language, and be respectful to the court, it shall be recorded; and the court may, if deemed necessary, put an answer to the dissent, protest, or objection on the records along with it. But here the matter shall end, unless the parties protesting obtain permission to withdraw their dissent, protest, or objection absolutely, or for the sake of amendment.
\end{enumerate}

\hypertarget{jurisdiction}{%
\section*{49. Jurisdiction}\label{jurisdiction}}
\addcontentsline{toc}{section}{49. Jurisdiction}

\protect\hypertarget{chapter-slug-49-jurisdiction}{\href{}{}}

\begin{enumerate}
\def\labelenumi{\arabic{enumi}.}
\tightlist
\item
  \protect\hypertarget{49}{\href{}{}}When a church member shall remove his residence beyond the bounds of the congregation of which he is a member, so that he can no longer regularly attend its services, it shall be his duty to transfer his membership by presenting a certificate of dismissal from the Session of the church of which he is a member to the church with which he wishes to unite. When the church of which he is a member has no Session, or for other good reasons it seems impossible for the member to secure a certificate of dismissal, he may be received by the Session upon other satisfactory testimonials, in which case the church of which he was a member shall be duly notified.
\item
  When a church member shall remove his residence beyond the bounds of the church of which he is a member into the bounds of another, it shall be the duty of the Pastor and Ruling Elders of the church of which he is a member, as far as possible, to continue pastoral oversight of him and to inform him that according to the teaching of our Book of Church Order it is his duty to transfer his membership as soon as practicable to the church in whose bounds he is living. It shall also be the duty of the church from whose bounds the member moved to notify the Pastor and Ruling Elders of the church into whose bounds he has moved and request them to take pastoral oversight of the member, with a view to having him transfer his membership. If a member, after having thus been advised, shall neglect for one year to have his membership transferred, the Session shall then proceed, according to BCO \protect\hyperlink{41.4}{41.4}, except in special cases such as: servicemen, students, etc. The name of any member whose residence has been unknown for one year to the Session shall be removed from the roll and such names are not to be counted in the annual statistical reports, though act of removal should be recorded in the Session's minutes. If such a person at a latter date should appear or desire transfer of his letter, the Session will inform the governing body of the inquiring church of their action in removing said person from their roll.
\item
  Members of one church dismissed to join another shall be held to be under the jurisdiction of the Session dismissing them until they form a regular connection with that to which they have been dismissed.
\item
  Associate members are those believers temporarily residing in a location other than their permanent homes. Such believers may become associate members of a particular church without ceasing to be communing members of their home churches. An associate member shall have all the right and privileges of that church, with the exception of voting in a congregational or corporation meeting, and holding an office in that church.
\item
  When a Presbytery shall dismiss a Minister, licentiate, or candidate, the name of the Presbytery to which he is dismissed shall be given in the certificate, and he shall remain under the jurisdiction of the Presbytery dismissing him until received by the other.
\item
  No certificate of dismissal from either a Session or a Presbytery shall be valid testimony of good standing for a longer period than one year, unless its earlier presentation be hindered by some providential cause; and such certificates given to persons who have left the bounds of the Session or Presbytery granting them, shall certify the standing of such persons only to the time of their leaving those bounds.
\item
  \protect\hypertarget{49.7}{\href{}{}}When a Presbytery shall divest a Minister of his office without censure, or depose him without excommunication, it shall assign him to membership in some particular church, subject to the approval of the Session of that church.
\end{enumerate}

\hypertarget{directory-for-the-worship-of-god}{%
\chapter*{Directory for the Worship of God}\label{directory-for-the-worship-of-god}}
\addcontentsline{toc}{chapter}{Directory for the Worship of God}

\hypertarget{preface-to-the-directory}{%
\section*{Preface to the Directory}\label{preface-to-the-directory}}
\addcontentsline{toc}{section}{Preface to the Directory}

\protect\hypertarget{dfw-preface}{\href{}{}}The purpose of this Directory is to express the Church's common understanding of the principles and practice of public worship that is Reformed according to the Scriptures and, subordinately, to the Confession and Catechisms. Where practices are understood by the Church to be required by the Word of God, either expressly or by good and necessary consequence, they are mandated. In matters of circumstance and form in worship not specifically provided for in Scripture, the Directory provides guidance for their ordering according to the light of nature and Christian prudence, consonant with the general rules of the Word.

The Directory seeks to make clear this distinction in its use of language. The following denotations used in the Directory are to be understood as indicated. The first category denotes practices that are required by the Word of God.

\begin{itemize}
\tightlist
\item
  Practices that are mandated are denoted by ``shall,'' ``will,'' ``is to be,'' ``must,'' and ``are to be.''
\end{itemize}

The following three categories denote practices that are not mandated:

\begin{itemize}
\item
  Practices that are strongly recommended are denoted by ``should,'' ``ought to,'' ``is desirable,'' and ``is advisable.''
\item
  Practices that are commended as suitable are denoted by ``is appropriate,'' ``is well,'' and ``is fitting.''
\item
  Practices that are permissible are denoted by ``may.''
\end{itemize}

Other imperative forms occur in the Directory, and sometimes the forms in the list above are varied by modifying words or are put in the negative, either of which alters their force. For example, ``may not'' and ``may only'' are mandatory prohibitions, even though ``may'' is permissive. The meaning of these additional and altered forms is to be determined by the rules of English usage, with due respect to the distinctions outlined above.

The Suggested Forms for Particular Services are, by definition, suggested. The distinctions outlined above do not apply to the Suggested Forms.

Scripture quotations in the Directory and the Suggested Forms are drawn from the New American Standard Bible (1995) with a few variations, indicated by brackets, where deemed advisable for current understanding, without prejudice to other translations. In the use of the Directory, any accurate, faithful translation may be substituted.

\hypertarget{gods-institution-of-public-worship}{%
\section*{50. God's Institution of Public Worship}\label{gods-institution-of-public-worship}}
\addcontentsline{toc}{section}{50. God's Institution of Public Worship}

\protect\hypertarget{chapter-slug-50-gods-institution-of-public-worship}{\href{}{}}

\begin{enumerate}
\def\labelenumi{\arabic{enumi}.}
\tightlist
\item
  \protect\hypertarget{50}{\href{}{}}The living and true God, our triune Creator, has instituted the worship of Himself by all people everywhere in spirit and in truth.

  \begin{enumerate}
  \def\labelenumii{\alph{enumii}.}
  \tightlist
  \item
    Because man's chief end is to glorify God and fully to enjoy Him forever, all of life is to be worshipful. Nevertheless, worship itself consists primarily in specific acts of communion with God.
  \item
    Only those people whose hearts have been made new through God's grace by the work of the Holy Spirit can truly worship God.
  \item
    While believers are to worship in secret as individuals and in private as families, they are also to worship as churches in assemblies of public worship, which are not carelessly or willfully to be neglected or forsaken. Public worship occurs when God, by His Word and Spirit, through the lawful government of the church, calls His people to assemble to worship Him together.
  \end{enumerate}
\item
  In His Word, God has specially appointed one day in seven as a Sabbath to be kept holy to Him. It is the duty of every one to remember the Sabbath day, to keep it holy. From the beginning of the world to the resurrection of Christ, the Sabbath was the last day of the week, marking the completion of six days of work, anticipating eternal rest in the coming Messiah. By raising Christ from the dead on the first day of the week, God sanctified that day. And from the time of the apostles, the church, accordingly, has kept the first day of the week holy as the Christian Sabbath, the Lord's Day, and as the day on which it is to assemble for worship. Now each weekly cycle begins with the people of God resting in Christ in the worship of His name, followed by six days of work. The Lord's Day thus both depicts that the Christian's rest has already begun in Christ, and anticipates the eternal rest of His sons and daughters in the new heaven and the new earth.
\item
  God's covenant people are to devote the entire Lord's Day as holy to the Lord.

  \begin{enumerate}
  \def\labelenumii{\alph{enumii}.}
  \tightlist
  \item
    In order to sanctify the day, it is necessary for them to prepare for its approach. They should attend to their ordinary affairs beforehand, so that they may not be hindered from setting the Sabbath apart to God.
  \item
    It is advisable for each individual and family to prepare for communion with God in His public ordinances. Therefore, they ought to do this by reading the Scriptures, by holy meditation, and by prayer, especially for God's blessing on the ministry of the Word and sacraments.
  \item
    They are then to observe a holy rest all the day from their own works, words, and thoughts concerning their everyday employment and recreations, and to devote themselves to delighting in the public and private exercises of communion with God and His people, in showing mercy and doing good in His name, and in works of necessity.
  \item
    They shall so order works of necessity on that day that they do not improperly detain others from the public worship of God, nor otherwise hinder them from sanctifying the Sabbath.
  \end{enumerate}
\item
  The Lord's Day is a day of holy convocation, the day on which the Lord calls His people to assemble for public worship. Although it is fitting and proper that the members of Christ's church assemble for worship on other occasions also, which are left to the discretion of particular sessions, the Lord calls the whole congregation of each local church to the sacred duty and high privilege of assembling for public worship each Lord's Day. He expressly commands His people to draw near to Him, not forsaking the assembling of themselves together. Therefore, both worship and fellowship are necessary on the Lord's Day.
\end{enumerate}

\hypertarget{the-nature-of-public-worship}{%
\section*{51. The Nature of Public Worship}\label{the-nature-of-public-worship}}
\addcontentsline{toc}{section}{51. The Nature of Public Worship}

\protect\hypertarget{chapter-slug-51-the-nature-of-public-worship}{\href{}{}}

\begin{enumerate}
\def\labelenumi{\arabic{enumi}.}
\tightlist
\item
  \protect\hypertarget{51}{\href{}{}}An assembly of public worship is not merely a gathering of God's children with each other, but is, before all else, a meeting of the triune God with His covenant people. In the covenant, God promises His chosen ones that He will dwell among them as their God and they will be His people.

  \begin{enumerate}
  \def\labelenumii{\alph{enumii}.}
  \tightlist
  \item
    The triune God is present in public worship, not only by virtue of the divine omnipresence, but, much more intimately, as the faithful covenant Savior. Through Christ, God's people have access by one Spirit to the Father.
  \item
    In an assembly of public worship, the triune God is not only the One to whom worship is directed, but also the One who is active in the worship of the church. Through His public ordinances, the covenant God actively works to engage His people in communion with Himself. In public worship, God communes with His people, and they with Him, in a manner which expresses the close relationships of the Father and His redeemed children, of the Son and His beloved Bride, and of the Holy Spirit and the living temple in which He dwells.
  \item
    Pastors and ruling elders are to endeavor to inculcate in themselves and in the congregation expectations for, attitudes concerning, and behavior during public worship which are appropriate to the glorious fact that public worship is covenantal communion between God and His people in His public ordinances.
  \end{enumerate}
\item
  Because Christ is the Mediator of the covenant, no one draws near to God except through Him alone.

  \begin{enumerate}
  \def\labelenumii{\alph{enumii}.}
  \tightlist
  \item
    God's people enter the Most Holy Place, the heavenly sanctuary, by the redeeming blood of Jesus, by the new and living way opened for them through the curtain, that is, His flesh. They draw near through Him as their Great High Priest, who has not entered a man-made sanctuary but heaven itself, now to appear for them in God's presence.
  \item
    Public worship is to be conducted in a manner that plainly expresses conscious reliance upon the mediation and merits of Jesus Christ. To this end, it is well that there be a prayer of confession of sin early in the worship service. It is fitting that the minister, as God's ambassador, then declare an assurance of God's grace in Christ, reminding each worshiper that he can have boldness to approach the holy God only through the mediation and merits of Jesus Christ.
  \end{enumerate}
\item
  By the Spirit of the exalted Christ, God draws near to His people and they draw near to their God. They come by grace to Mount Zion, the heavenly Jerusalem, joining innumerable angels and all the people of God in joyous and reverent communion with Him.

  \begin{enumerate}
  \def\labelenumii{\alph{enumii}.}
  \tightlist
  \item
    God's people not only are to come into His presence with a deep sense of awe at the thought of His perfect holiness and their own exceeding sinfulness, but also are to enter into His gates with thanksgiving and into His courts with praise for the great salvation that He has so graciously wrought for them through His only begotten Son and which He applies to them by His Holy Spirit. All are therefore to worship with sincere devotion, reverence, and expectation.
  \item
    Public worship is to be conducted in reliance on the gracious working of the Spirit of the exalted Christ, which alone can make anyone capable of such sincerity, reverence, devotion, awe, expectation, and joy. Hence, from its beginning to its end, public worship should be conducted in that simplicity which manifests dependence on the Spirit of Christ to bless His own ordinances.
  \item
    Accordingly, the whole congregation should assemble promptly, that all may be present and may join together for the entire worship service. Unless necessary, none should depart until after the benediction. All should refrain from any behavior that would distract other worshipers or detract from their communion with God.
  \end{enumerate}
\item
  In public worship, God's people draw near to their God unitedly as His covenant people, the body of Christ.

  \begin{enumerate}
  \def\labelenumii{\alph{enumii}.}
  \tightlist
  \item
    For this reason, the covenant children should be present so far as possible, as well as adults. Out-of-service childcare programs, while permissible and in many cases a necessary good, should lead towards and promote a child's full participation in the service as soon as possible. Because God makes His covenant with believers and their children, it is fitting for households to sit together.
  \item
    For the same reason, no favoritism may be shown to any who attend. Nor may any member of the church presume to exalt himself above others as though he were more spiritual, but each shall esteem others better than himself.
  \item
    The unity and catholicity\footnote{I.e., universal or worldwide.} of the covenant people are to be manifest in public worship. Accordingly, the service is to be conducted in a manner that enables and expects all the members of the covenant community---male and female, old and young, poor and rich, uneducated and educated, healthy and infirm, people from every tribe, tongue, and nation---to worship together.
  \item
    Because God's people worship, not as an aggregation of individuals, but as a congregation of those who are members of one another in Christ, public worship is to be conducted as a corporate activity in which all the members participate as the body of Christ.
  \end{enumerate}
\end{enumerate}

\begin{enumerate}
\def\labelenumi{\arabic{enumi}.}
\setcounter{enumi}{4}
\tightlist
\item
  The triune God assembles His covenant people for public worship in order to manifest and renew their covenant bond with Him and one another. The Holy Spirit engages them and draws them into the Father's presence as a living sacrifice in Christ. God Himself has fellowship with them, strengthening and guiding them for life in His presence and service in His kingdom.

  \begin{enumerate}
  \def\labelenumii{\alph{enumii}.}
  \tightlist
  \item
    Public worship should be conducted in a manner that reflects God's initiative in the covenant itself, making clear that God establishes and renews His covenant with His people, assuring God's people of those things which they so easily forget unless Christ crucified is portrayed before their eyes week after week, cultivating the expectation that God Himself meets His people in Christ as the Holy Spirit works through the public ordinances, always keeping central the persons and works of the triune God.
  \item
    Consequently, it is well that public worship be so conducted that it is apparent that God summons His church to assemble in His presence, that He assures His people of His receiving and cleansing them through Christ the Mediator, that He consecrates them to Himself and His service by His Word, that He communes with them and gives them grace to help in time of need through His means of grace, and that He sends them out to serve with His blessing at the benediction.
  \end{enumerate}
\item
  The triune God reveals the way of knowing and worshiping Him in His Word, the Holy Scriptures of the Old and New Testaments, which is the only infallible rule of faith and practice.

  \begin{enumerate}
  \def\labelenumii{\alph{enumii}.}
  \tightlist
  \item
    The principles of public worship must be derived from the Bible---either as they are expressly set down in Scripture or by good and necessary consequence may be deduced from Scripture---and from no other source. The purpose of this directory is not contrary to that purpose, as it points us to what Scripture teaches concerning worship. Thus this directory and our confessions are standards always understood to be subordinate to the Word of God.
  \item
    God may not be worshiped according to human imaginations or inventions or in any way not prescribed by His Word, nor may the church require her members to participate in elements of worship that God's Word does not require. Only when the elements of worship are those appointed in God's Word, and the circumstances and forms of worship are consonant with God's Word, is there true freedom to know God as He is and to worship Him as He desires to be worshiped.
  \end{enumerate}
\item
  The end of public worship is the glory of the triune God. To that end, Christ builds His church by perfecting the saints and adding to its membership such as are being saved---all to the glory of God. To this end, it is right that non-believers participate in worship because of the natural effect of worship and preaching in calling men to faith.\footnote{1 Corinthians 14:24-25.}

  \begin{enumerate}
  \def\labelenumii{\alph{enumii}.}
  \tightlist
  \item
    Through public worship on the Lord's Day, God calls His people to serve Him all the days of the week in their every activity, and enables them, whether they eat or drink or whatever they do, to do all to the glory of God.
  \item
    God's people are to be led to engage in all the elements of worship with a single-minded focus on God's glory and with a humble and dependent expectation that the exalted Lord Jesus Christ Himself will edify them and build His church through His appointed means of grace---all to the glory of God.
  \end{enumerate}
\end{enumerate}

\hypertarget{the-parts-of-public-worship}{%
\section*{52. The Parts of Public Worship}\label{the-parts-of-public-worship}}
\addcontentsline{toc}{section}{52. The Parts of Public Worship}

\protect\hypertarget{chapter-slug-52-the-parts-of-public-worship}{\href{}{}}

\begin{enumerate}
\def\labelenumi{\arabic{enumi}.}
\tightlist
\item
  \protect\hypertarget{52}{\href{}{}}Because a service of public worship is in its essence a meeting of the triune God with His chosen people (some of whom have not yet come to faith and repentance), a worship service consists of two principal parts: those elements which are done on behalf of God (through a representative voice) and those elements which are done by the congregation (through their own or a representative voice).

  \begin{enumerate}
  \def\labelenumii{\alph{enumii}.}
  \tightlist
  \item
    By His Spirit working through the ministry of the Word, God addresses His people in the call to worship, in the salutation\footnote{A salutation is a greeting from the Lord, taken from the inspired greetings, like those of the Apostle Paul in his epistles. See Ephesians 1:2; Philippians 1:2.} and benediction, in the reading and preaching of the Word, and in the sacraments.
  \item
    His people, enabled by the Holy Spirit, address God in prayer, in song, in offerings, in hearing the Word, in confession, and in receiving and partaking of the sacraments.
  \item
    It is advisable that these two parts be interspersed in the service.
  \end{enumerate}
\end{enumerate}

\begin{enumerate}
\def\labelenumi{\arabic{enumi}.}
\setcounter{enumi}{1}
\tightlist
\item
  The triune God is not a passive spectator in public worship, but actively works in each element of the service of worship. Neither are the people of God to be passive spectators in public worship, but by faith are to participate actively in each element of the service of worship.

  \begin{enumerate}
  \def\labelenumii{\alph{enumii}.}
  \tightlist
  \item
    Public worship should be conducted in a manner that enables and expects God's people by faith actively to embrace the blessing of the Lord in the salutation and benediction; to pray with him who leads in prayer, so that the prayer being uttered aloud becomes their prayer, joining the one speaking to God in their behalf by saying ``amen'' with him at the conclusion of his prayer; to attend, in the reading of God's Word, to what God reveals of Himself, His redeeming actions for them, and His will for their lives; to confess together with all the people the faith of the church; to heed the Word of truth as the sermon is preached and to appropriate it to their lives as God, through His servant, proclaims and applies it; to sing psalms, hymns, and spiritual songs to the praise of God and the edification of one another; to offer their possessions and themselves together as a living sacrifice to the Lord.
  \item
    Accordingly, it is appropriate that worshipers respond with brief spoken or sung expressions of praise or affirmation such as ``hallelujah'' or ``amen.'' The former is a heartfelt declaration that the living God alone is worthy of adoration. The latter grows out of the responsibility of God's people to affirm solemnly and earnestly the truthfulness of His Word and the permanence of His character. It is especially fitting for the congregation to join in an ``amen'' at such times as a response to a blessing, a Scripture reading, a psalm or hymn, a confession of faith, or a prayer. When believers sing or say ``amen,'' they are testifying to their wholehearted agreement with what has been spoken as being in harmony with God's permanently valid Word.
  \end{enumerate}
\item
  The Lord Jesus Christ has not prescribed a set order for public worship and care must be taken to oppose schismatic movements centered around men's preferences in this matter. It may not be forgotten, however, that there is true liberty only where the rules of God's Word are observed and the Spirit of the Lord is, so that all things are done decently and in order, and God's people approach Him with reverence and in the beauty of holiness. While Christ has not prescribed a set order for public worship, this does not mean that it is fitting to ignore proper and scriptural patterns of worship that have been historically observed by the church, particularly in the Reformed tradition. The order of worship should be so structured that there will be a purposeful movement on the part of the congregation from one element of the service to the next. When each act of worship is full of meaning, the order of the elements will assume a coherent, edifying form.
\item
  The session does well to ensure that the public worship assembly space is so arranged as to reflect and reinforce God's initiative in drawing near to and gathering His people through the ministry of the Word and sacraments. Because the pulpit, baptismal waters, and communion table facilitate the part of worship which is performed on behalf of God, it is fitting that they be positioned so as to draw the focus of the congregation upon the Word and sacraments, and that they be easily accessible and visible to the entire congregation throughout the worship service. Because the Word is primary and the sacraments serve to seal the Word, it is fitting that the pulpit be in the position of prominence.
\end{enumerate}

\hypertarget{the-oversight-and-conduct-of-public-worship}{%
\section*{53. The Oversight and Conduct of Public Worship}\label{the-oversight-and-conduct-of-public-worship}}
\addcontentsline{toc}{section}{53. The Oversight and Conduct of Public Worship}

\protect\hypertarget{chapter-slug-53-the-oversight-and-conduct-of-public-worship}{\href{}{}}

\begin{enumerate}
\def\labelenumi{\arabic{enumi}.}
\tightlist
\item
  \protect\hypertarget{53}{\href{}{}}Public worship is Christian, not only when the worshipers consciously recognize that Christ is the Mediator by whom alone they can come unto God, but also when they honor the exalted Christ as the living and only Head of the church, who rules over public worship.

  \begin{enumerate}
  \def\labelenumii{\alph{enumii}.}
  \tightlist
  \item
    He rules over public worship by His Word and Spirit, not only directly, but also through the ministry of officers in their ruling, serving, teaching, and leading His church.
  \item
    The exalted Christ thus applies Himself and His benefits to the elect through His Spirit working in human hearts by and with His Word, especially in its public reading, its preaching, its sealing by the sacraments, and as it is received in faith by prayer.
  \end{enumerate}
\item
  For this reason:

  \begin{enumerate}
  \def\labelenumii{\alph{enumii}.}
  \tightlist
  \item
    The session is responsible to give immediate oversight to the conduct of public worship in the local church.
  \item
    Public worship is ordinarily to be led by those who have been ordained to represent the Lord Jesus Christ in the administration of His Word and sacraments. The pastor of the church is ordinarily responsible to plan (or oversee the planning of) and lead public worship.
  \item
    Men who have been licensed by a presbytery to preach the Gospel may plan and conduct worship as probationers in order that the churches may form a better judgment respecting the fitness of those by whom they are to be instructed and governed. They may not, however, pronounce the salutation or the benediction or administer the sacraments.
  \item
    When the session deems it fitting, ruling elders and other approved men may lead the congregation in prayer, read the Scriptures to the congregation, lead unison or antiphonal readings of Scripture by the congregation, lead congregational singing, or, on occasion, exhort the congregation as part of public worship. They may not, however, pronounce the salutation or the benediction, or administer the sacraments.
  \item
    On occasion, with the approval of the session and under the close supervision of a minister, exceptions may be made for other men (even the young), and especially those being prepared for the Gospel ministry in Christ's church who are either members of the congregation governed by that session or are ministerial interns under that session. They may not, however, pronounce the salutation or the benediction or administer the sacraments.
  \item
    The session may choose to delegate leadership of the musical portions of the service to unordained, godly men gifted for that work, serving under the oversight of the pastor.
  \item
    No others should take such leadership in overseeing or conducting public worship.
  \item
    There are times when it is appropriate for women to participate vocally in a worship service without it being a violation of Scripture's command for women to remain silent.\footnote{1 Corinthians 14:34.} Women should be encouraged to sing, to share prayer requests, to give testimonies of God's faithfulness, and to pray.
  \item
    Only church officers should distribute the elements of the Lord's Supper because this activity involves the exercise of pastoral authority over the congregation and is a picture of our heavenly Father's provision of fathers to feed His flock.
  \end{enumerate}
\end{enumerate}

\hypertarget{elements-of-ordinary-public-worshipfrom-god-to-the-people}{%
\section*{54. Elements of Ordinary Public Worship---From God to the People}\label{elements-of-ordinary-public-worshipfrom-god-to-the-people}}
\addcontentsline{toc}{section}{54. Elements of Ordinary Public Worship---From God to the People}

\protect\hypertarget{chapter-slug-54-elements-of-ordinary-public-worship-from-god-to-the-people}{\href{}{}}

\begin{enumerate}
\def\labelenumi{\arabic{enumi}.}
\tightlist
\item
  \protect\hypertarget{54}{\href{}{}}\textbf{The Call to Worship}

  \begin{enumerate}
  \def\labelenumii{\alph{enumii}.}
  \tightlist
  \item
    God having summoned His people to assemble in His presence to worship Him on the Lord's Day, there ought to be a call to the congregation, in God's own words, to worship Him. He who performs this element serves as God's representative voice; accordingly, it ordinarily should be performed by a minister of the Word.
  \item
    It is fitting that the congregation respond to the call to worship in words of Scripture, or with singing, or with prayer, or with all of these.
  \end{enumerate}
\item
  \textbf{The Public Reading of God's Word}

  \begin{enumerate}
  \def\labelenumii{\alph{enumii}.}
  \tightlist
  \item
    Because the hearing of God's Word is a means of grace, the public reading of the Holy Scriptures is an essential element of public worship. He who performs this serves as God's representative voice. Thus, it ordinarily should be performed by a minister of the Word. Through this reading, God speaks directly to the congregation in His own words. For this reason, readers other than the pastor should ordinarily refrain from interspersing the reading of God's Word with human comments. He should use an accurate, faithful translation in the language of the people. In English, translations such as the NASB95, NKJV, and KJV fulfill these requirements. Modern translations that pander to the corruptions of the day, such as minimizing or removing the male semantic meaning components of God's words, are forbidden. The minister should read clearly and with understanding, and the congregation should attend to the reading with the deepest reverence.
  \item
    It is desirable that portions from both the Old and the New Testaments be read each Lord's Day. It is also well that the law of God be read frequently.
  \item
    The public reading of the Scriptures to the congregation is to be distinguished from the unison or antiphonal reading of certain portions of Scripture by the minister and the congregation together. In the former, God addresses His people; in the latter, God's people address their God, expressing in the words of Scripture their own contrition, adoration, gratitude, and other holy sentiments. The Psalms of Scripture are especially appropriate for this purpose.
  \end{enumerate}
\item
  \textbf{The Preaching of God's Word}

  \begin{enumerate}
  \def\labelenumii{\alph{enumii}.}
  \tightlist
  \item
    The preaching of the Word, the power of God unto salvation, is of central importance in the public worship of God. It is therefore a matter of supreme importance that the minister preach only the Word of God, not the wisdom of man, and that he handle the Word of God correctly, always setting forth Jesus Christ, the Author and Finisher of our faith. In the sermon, God addresses the congregation by the mouth of His servant, and through His Spirit opens the ears of His people.
  \item
    The preacher shall prepare each sermon prayerfully and diligently. He must take pains to expound a particular text of Scripture, bringing in other texts as applicable, carefully explaining the meaning, and diligently applying the particular text(s) for the salvation and edification of his hearers. He should take care in preaching that his exposition and application of the Scriptures be clear and simple, having regard to the capacity of the hearers, in demonstration of the Spirit and power, with fervor and zeal, and that he not divorce Christian duty from Christian faith. The preacher must, as Christ's ambassador, seek to build up the saints in the most holy faith and beseech the unconverted to be reconciled to God. Nothing is more necessary than that the Gospel of salvation by grace be proclaimed without any adulteration or compromise, in order that the hearers may learn to rely for salvation only on the grace of God in Christ, to the exclusion of their own works or character, ascribing all glory to God alone for their salvation. The preacher is to instruct his hearers in the whole counsel of God, exhort the congregation to more perfect obedience to Christ, and warn them of the sins and dangers that are around them and within them. A preacher fails to perform his task as a God-appointed watchman on Zion's walls when he neglects to warn the congregation of prevalent soul-destroying teachings by enemies of the Gospel.
  \item
    The session is to give diligence that no person enter the pulpit concerning whose soundness in doctrine and life, or knowledge of Scripture, there is reasonable doubt.
  \item
    No person shall be invited to preach in any of the churches under our care without the consent of the Session, unless sent by the Presbytery.
  \end{enumerate}
\item
  \textbf{The Sacraments}

  \begin{enumerate}
  \def\labelenumii{\alph{enumii}.}
  \tightlist
  \item
    The sacraments, baptism and the Lord's Supper, as visible signs and seals of the Word of the covenant, are important elements of public worship. They represent Christ and His benefits, confirm His people's participation in Him, visibly mark off from the world those who belong to His church, and solemnly bind them to covenant faith and loyalty.
  \item
    Because the sacraments are ordinances of Christ for the benefit of the visible church, they are to be administered only under the oversight of the government of the church. Moreover, in ordinary circumstances they are properly administered only in a gathering of the congregation for the public worship of God, baptism being a sacrament whereby the parties baptized are solemnly admitted into the visible church, and the Lord's Supper signifying and sealing the communion of believers with Christ and with each other as members of His mystical body.
  \item
    Although the efficacy of the sacraments does not depend upon the piety or intention of the person administering them, they are not to be administered by any private person, but only a minister of the Word. Because the sacraments were given from Christ through the apostles to the church,\footnote{Matthew 28:16-20; 1 Corinthians 11:23.} and ministers are ``stewards of the mysteries of God''\footnote{1 Corinthians 4:1.} in Christ's church, no other is permitted to take this honor to himself.\footnote{Hebrews 5:4.}
  \end{enumerate}
\end{enumerate}

\begin{enumerate}
\def\labelenumi{\arabic{enumi}.}
\setcounter{enumi}{4}
\tightlist
\item
  \textbf{Blessings}

  \begin{enumerate}
  \def\labelenumii{\alph{enumii}.}
  \tightlist
  \item
    The salutation and the benediction are blessings pronounced in God's name and in His own words. Accordingly, they are properly used only in a gathering of Christ's church and by a minister of the Word.
  \item
    A salutation is the greeting from God to His people who have gathered to worship Him. It is fittingly pronounced immediately before or after the call to worship. Words of salutation from Scripture may be used, such as the opening greeting from one of the New Testament epistles, ``Grace to you and peace from God our Father and the Lord Jesus Christ.''
  \item
    A benediction is the pronouncement of God's blessing upon His people at the conclusion of the worship service. Words of benediction taken from Scripture are to be used. The high priestly benediction, ``The LORD bless you, and keep you; the LORD make His face shine upon you and be gracious to you; the LORD lift up His countenance on you, and give you peace,''\footnote{Numbers 6:24-26.} or the Trinitarian apostolic benediction, ``The grace of the Lord Jesus Christ, and the love of God, and the fellowship of the Holy Spirit, be with you all,''\footnote{2 Corinthians 13:14.} are distinctly appropriate. If, however, the minister deems another benediction taken from Scripture more fitting for a particular occasion, he may use it.
  \end{enumerate}
\end{enumerate}

\hypertarget{elements-of-ordinary-public-worshipfrom-the-people-to-god}{%
\section*{55. Elements of Ordinary Public Worship---From the People to God}\label{elements-of-ordinary-public-worshipfrom-the-people-to-god}}
\addcontentsline{toc}{section}{55. Elements of Ordinary Public Worship---From the People to God}

\protect\hypertarget{chapter-slug-54-elements-of-ordinary-public-worship-from-the-people-to-god}{\href{}{}}

\begin{enumerate}
\def\labelenumi{\arabic{enumi}.}
\tightlist
\item
  \protect\hypertarget{55}{\href{}{}}\textbf{Public Prayer}

  \begin{enumerate}
  \def\labelenumii{\alph{enumii}.}
  \tightlist
  \item
    Prayer is an essential element of public worship. In order to be accepted by God, prayer is to be by faith, in the name of the Son of God, by the help of His Spirit, and according to God's will.
  \item
    In preparation for the service, the session should provide circumstances conducive to the people adopting an attitude appropriate to worship and prayer.
  \item
    He who leads in public prayer serves as the voice of the congregation. For this reason, he should pray in such a way, in clear words and in the plural, that the entire assembly of God's people can pray with him; and it is the duty of the members of the congregation, not only to hear his words, but also to pray them in their hearts. To these ends, he who leads should diligently prepare himself for public prayers, so that he may perform this duty with propriety and with profit to the worshipers.
  \item
    It is particularly appropriate that public prayer be led by a pastor of the congregation, because in it he both guides the people in their corporate prayer to God and teaches them how to pray biblically. Accordingly, every minister should, by a thorough acquaintance with the Holy Scriptures, by the study of the best writings on prayer, by meditation, and by a life of communion with God, endeavor to acquire both the spirit and the skill of prayer, as should ruling elders. When a guest minister is present, it is well that a pastor or ruling elder, as one who knows the congregation, lead in the prayers of intercession.
  \item
    Near the beginning of the service, there should be a brief prayer of approach to God in response to His call to enter His presence for worship. It may express humble adoration, confess unworthiness and inability to worship aright, seek His merciful acceptance through Jesus Christ, and invoke the gracious working of the Holy Spirit.
  \item
    During the service, there should be comprehensive prayer, which may be offered as more than one prayer throughout the worship service. Such prayer should include adoration of God's perfections, thanksgiving for all His mercies, confession of sin, supplication for forgiveness through the blood of the atonement and for renewal by the Holy Spirit, lamentation in times of distress or crisis, and intercession for the needs of God's people and others. It is fitting that the congregation intercede for the whole of mankind; for civil rulers; for the church universal; for Christian missions at home and abroad, Christian education, and other Christian activities; for our whole Church; for churches in ties of like faith with us; for the welfare of the local congregation itself, including its officers, its ministries, and its members, pleading for their growth in sanctification and remembering the daily needs and care of the people---the families, the singles, the rising generation, the elderly, the poor, the sick, the dying, the mourning, the erring, and unsaved loved ones; and for whatever else may seem particularly suitable.
  \item
    It is fitting that a prayer of confession of sin precede or follow any reading of the law of God to the congregation.
  \item
    It is appropriate that there be a brief offertory prayer either immediately preceding or immediately following the worship of God with offerings. Such prayer may thank God for His gifts, devote the offering and the worshipers to His service, and invoke His blessing on its use and on those who give.
  \item
    It is fitting to pray at the time of the reading and preaching of the Word. Such prayer may petition for the Holy Spirit to grant illumination and to apply the Word preached to the minds, hearts, and lives of the people and give thanks for the Word received. It is valuable that such prayer should be by the one preaching the Word.
  \item
    While public prayer must always be offered with deep humility and holy reverence and be free from vain repetition or display of words, it can be fitting at times for the entire congregation to pray vocally in unison. The form of prayer that our Lord Jesus taught His disciples, commonly called ``the Lord's Prayer,'' is particularly appropriate for this use by the congregation. Great care should be taken, however, to guard against allowing this practice to become a mere formula or ritual.
  \item
    It is fitting to encourage the congregation to join vocally in a corporate ``amen'' at the conclusion of a prayer.
  \item
    Our prayers should be addressed to God the Father and concluded in the name of the Son, according to Scripture.\footnote{Luke 11:2; John 14:13-14.}
  \end{enumerate}
\end{enumerate}

\begin{enumerate}
\def\labelenumi{\arabic{enumi}.}
\setcounter{enumi}{1}
\tightlist
\item
  \textbf{Congregational Singing}

  \begin{enumerate}
  \def\labelenumii{\alph{enumii}.}
  \tightlist
  \item
    Congregational singing is a duty and privilege and should not be neglected in worship. Let every member of the church take part in this act of worship. God's people should sing, not merely with the lips, but with understanding and with grace in their hearts, making melody to the Lord.
  \item
    As public worship is for the praise and glory of God and the building up of the saints, not for the entertainment of the congregation nor the praise of man, the character of the songs used therein is to befit the nature of God and the purpose of worship.
  \item
    Congregations do well to sing the metrical versions or other musical settings of the Psalms frequently in public worship. Congregations also do well to sing hymns of praise that respond to the full scope of divine revelation. It is recommended that new songs and Psalm settings be sung along with the familiar hymns of the church.
  \item
    In the choice of song for public worship, great care must be taken that all the materials of song are fully in accord with the Scriptures. The words are to be suitable for the worship of God and the tunes are to be appropriate to the meaning of the words and to the occasion of public worship. Care should be taken to the end that the songs chosen will express those specific truths and sentiments which are appropriate at the time of their use in the worship service.
  \item
    Musical gifts are properly used in public worship to assist the congregation in its worship of God. They may not be used for the praise or applause of men.
  \item
    No person may take a prominent part in the leadership of the service unless he is a professing Christian who is a member of Christ's church.
  \end{enumerate}
\item
  \textbf{Public Confession of Faith}

  \begin{enumerate}
  \def\labelenumii{\alph{enumii}.}
  \tightlist
  \item
    Individual believers are to publicly profess their faith in Christ before God and His people in order formally to pledge their commitment to serve Christ and to be welcomed into all the privileges of full communion with God's people.
  \item
    It is also fitting that the congregation as one body confess its common faith, using creeds that are true to the Word of God, such as the Apostles' Creed or the Nicene Creed.
  \end{enumerate}
\item
  \textbf{The Bringing of Offerings}

  \begin{enumerate}
  \def\labelenumii{\alph{enumii}.}
  \tightlist
  \item
    The bringing of offerings in the public assembly of God's people on the Lord's Day is a solemn act of worship to almighty God. The people of God are to set aside to Him the firstfruits of their labors; in so doing, they should present themselves with thanksgiving as a living sacrifice to God. All should participate in this act of worship when God gives opportunity for it. Parents are to instruct and encourage their children by precept and example to give of their substance regularly, purposefully, generously, and joyfully to the Lord through His church.
  \item
    It is the duty of the pastor, since he is to proclaim to the people the whole counsel of God, to cultivate biblical stewardship and the grace of liberal giving in the members of the church. The pastor himself, as well as the officers of the church, should be exemplary in their own worship through tithes and offerings. The Pastor should remind the church of the admonition in Scripture that everyone is to give as the Lord has prospered him, of the assurance of Scripture that God loves a cheerful giver, and of the blessed example of the Lord Jesus Christ, who, though He was rich, became poor, in order that poor sinners through His poverty might become rich.
  \item
    The session shall take care that the tithes and offerings of the congregation are used only for biblical purposes, such as the maintenance of public worship, the preaching of the Gospel throughout the world, the ministry of mercy in Christ's name, and other Christian objects.
  \item
    It is desirable that Christian love be demonstrated by offerings for the use of the deacons in the ministry of mercy on behalf of the church. It is appropriate that a special offering be received for this purpose following the Lord's Supper.
  \end{enumerate}
\end{enumerate}

\hypertarget{the-administration-of-the-sacramentsgeneral-provisions}{%
\section*{56. The Administration of the Sacraments---General Provisions}\label{the-administration-of-the-sacramentsgeneral-provisions}}
\addcontentsline{toc}{section}{56. The Administration of the Sacraments---General Provisions}

\protect\hypertarget{chapter-slug-56-the-administration-of-the-sacraments-general-provisions}{\href{}{}}

\begin{enumerate}
\def\labelenumi{\arabic{enumi}.}
\tightlist
\item
  \protect\hypertarget{56}{\href{}{}}In order that the sacraments may be observed with discernment and profit, it is imperative that adequate preparation be made prior to their administration. Before observing the sacraments, God's people ought to meditate on the teaching of the Word of God concerning them, particularly as summarized in the Confession of Faith and Catechisms. It is also advisable that from time to time the preaching include suitable instruction on the sacraments. Moreover, when the sacraments are being celebrated, the minister shall always accompany them by the preaching of the Word, and he shall take especial care in that preaching to proclaim Christ and His benefits, so that God's people can understand what the sacrament means. In connection with the administration of the sacraments, he shall set forth a summary of the teaching of the Word of God as to their institution, meaning, and nature.
\item
  Concerning paedo-baptism, the baptism of infants is not to be unduly delayed, but is to be administered as soon as practicable.
\item
  Concerning credo-baptism, baptism is to be administered as soon as practicable after public profession of faith in Christ.
\item
  The Lord's Supper is to be celebrated frequently, but the frequency may be determined by each session as it may judge most conducive to edification.
\item
  In the administration of the sacraments, the minister shall follow the directions prescribed in this chapter, but, except in the case of the words of the baptismal formula, he is not required to use the exact language of the indented forms (below), which are suggested as appropriate. He may employ these or similar forms, using his own liberty and godly wisdom, as he deems best for the edification of the people.
\end{enumerate}

\hypertarget{the-baptism-of-infants}{%
\section*{57. The Baptism of Infants}\label{the-baptism-of-infants}}
\addcontentsline{toc}{section}{57. The Baptism of Infants}

\protect\hypertarget{chapter-slug-57-the-baptism-of-infants}{\href{}{}}

\begin{enumerate}
\def\labelenumi{\arabic{enumi}.}
\item
  \protect\hypertarget{57}{\href{}{}}\textbf{Prerequisites}

  For a child to be presented for baptism, at least one parent must be a communing member of the Church, in good standing, normally of the local congregation. In order to present a child for baptism, parents shall make prior arrangements with the session. Before presentation for baptism, the session shall ensure that instruction has been given to the parent(s) as to the nature, privileges, and responsibilities of baptism for the parents and the child. Only parents who are communing members of a church may be permitted to take parental vows. If only one of two parents is a communing member and taking vows, it shall be the duty of the minister to inform the congregation of the situation, including that the one parent is not a member of this congregation and is not taking the vows. In extraordinary circumstances, at the parents' request, the session may permit the baptism of a child of parents who are communing members of another church which is approved by the session, on behalf of and with the written permission of the governing body of that church. In such a case, the session shall inform the other governing body, in writing, when the baptism has been administered.
\item
  \textbf{The Administration of Baptism to Infants}

  \begin{enumerate}
  \def\labelenumii{\alph{enumii}.}
  \item
    \textbf{The Institution of the Sacrament}

    The minister ought to read the words of the institution of the sacrament of baptism from a passage such as Matthew 28:18-20.
  \item
    \textbf{The Meaning and Nature of the Sacrament}

    The minister shall first summarize before the congregation the teaching of the Word of God and the Confession and Catechisms of this church as to the meaning and nature of the sacrament of baptism. In doing so, he may use these or like words:

    \begin{quote}
    The Lord Jesus Christ instituted baptism as a covenant sign and seal for His church. He uses it not only for the solemn admission of the person who is baptized into the visible church, but also to depict and to confirm His ingrafting of that person into Himself and His including that person in the covenant of grace.

    The Lord uses baptism to portray to us that we and our children are conceived and born in sin and need to be cleansed.

    He uses it to witness and seal to us the remission of sins and the bestowal of all the gifts of salvation through union with Christ. Baptism with water signifies and seals cleansing from sin by the blood and the Spirit of Christ, together with our death unto sin and our resurrection unto newness of life by virtue of the death and resurrection of Christ. The time of the outward application of the sign does not necessarily coincide with the inward work of the Holy Spirit which the sign represents and seals to us. Because these gifts of salvation are the gracious provision of the triune God, who is pleased to claim us as His very own, we are baptized in the name of the Father and of the Son and of the Holy Spirit.

    In our baptism, the Lord puts His name on us, claims us as His own, and summons us to assume the obligations of the covenant. He calls us to believe in Jesus Christ as our Savior, to renounce the devil, the world, and the flesh, and to walk humbly with our God in devotion to His commandments.
    \end{quote}
  \item
    \textbf{Exhortation to the Members of the Congregation to Improve Their Baptism}

    Then the minister may exhort the congregation in these or like words:

    \begin{quote}
    As solemn vows are about to be made before you, and baptism is now to be administered, you who are baptized will do well to take this occasion to reflect on your own baptism. Christ has put His name and claim on you. He calls you to be repentant for your sins against your covenant God, to confess your faith before men, and to live in newness of life to God, who sealed His covenant with you by the blood of His own Son.
    \end{quote}
  \item
    \textbf{The Ground of Baptizing Infants}

    The minister shall then give instruction as to the ground of the baptism of infants. He may use these or like words:

    \begin{quote}
    Although our young children do not yet understand these things, they are nevertheless to be baptized. For God commands that all who are under His covenant of grace be given the sign of the covenant.

    God made the promise of the covenant to believers and to their offspring. In the Old Testament, He declared to Abraham: ``I will establish My covenant between Me and you and your descendants after you throughout their generations for an everlasting covenant, to be a God unto you, and to your descendants after you.''\footnote{Genesis 17:7.} For this reason, in the Old Testament, God commanded that covenant infants be given the sign of circumcision.

    The covenant is the same in essence in both the Old and the New Testaments. Indeed, the grace of God for the consolation of believers is even more fully manifested in the New Testament. He declares that ``For the promise is for you and your children.''\footnote{Acts 2:39.} He promises, ``Believe in the Lord Jesus Christ, and you will be saved, you and your household.''\footnote{Acts 16:31.} He affirms that if even one parent is a believer, the children are ``holy.''\footnote{1 Corinthians 7:14.} Moreover, our Savior admitted little children into His presence, embracing and blessing them, and saying, ``for the kingdom of God belongs to such as these.''\footnote{Mark 10:14.}

    And so, in the New Testament no less than in the Old, the children of believers have an interest in the covenant and a right to the covenant sign and to the outward privileges of the covenant people, the church. In the New Testament, baptism has replaced circumcision as the covenant sign.\footnote{Colossians 2:11--12.} Therefore, by the covenant sign of baptism the children of believers are to be distinguished from the world and solemnly admitted into the visible church.
    \end{quote}
  \item
    \textbf{The Covenant Commitments of the Parents and the Congregation}
    The minister shall then require the parents to vow publicly their duty as Christian parents to present their children for baptism and to nurture them in the Christian faith, by answering these questions in the affirmative:

    \begin{quote}
    \begin{enumerate}
    \def\labelenumiii{(\arabic{enumiii})}
    \tightlist
    \item
      Do you reaffirm your own faith in Jesus Christ as Savior and Lord?
    \item
      Do you acknowledge your child's need of the cleansing blood of Jesus Christ, and the renewing grace of the Holy Spirit?
    \item
      Do you claim God's covenant promises in {[}his/her{]} behalf and do you look in faith to the Lord Jesus Christ for {[}his/her{]} salvation, as you do for your own?
    \item
      Do you now unreservedly dedicate your child to God, and promise, in humble reliance upon God's grace, that you will seek to set before {[}him/her{]} a godly example, that you will pray with and for {[}him/her{]}, that you will teach {[}him/her{]} the doctrines of our holy religion, and that you will strive, by all the tools which God has given us, to bring {[}him/her{]} up in the nurture and admonition of the Lord?
    \end{enumerate}
    \end{quote}

    The minister shall then require the congregation\footnote{If the session has permitted the parents' request for the baptism of an infant to take place in another church, the congregation present at the baptism should not take this vow. At a subsequent date, with explanation (especially in exclusively credo-baptistic churches), the vow should be administered to the home congregation.} to vow publicly their duty to assist the parent(s) in the Christian nurture of the child:

    \begin{quote}
    Do you as a congregation undertake the responsibility of assisting the parents in the Christian nurture of this child? {[}If so, say, ``We do.''{]}
    \end{quote}
  \item
    \textbf{Prayer}
  \end{enumerate}

  The minister shall then pray for the presence and blessing of the triune God, that the grace signified and sealed by baptism may be abundantly realized.

  \begin{enumerate}
  \def\labelenumii{\alph{enumii}.}
  \setcounter{enumii}{6}
  \tightlist
  \item
    \textbf{The Baptism}
  \end{enumerate}

  Then, calling the child by name, the minister shall say, as he baptizes him with water, without adding any other ceremony:

  \begin{quote}
  {[}Name of Child{]}, I baptize you in the name of the Father and of the Son and of the Holy Spirit.
  \end{quote}

  \begin{enumerate}
  \def\labelenumii{\alph{enumii}.}
  \setcounter{enumii}{7}
  \tightlist
  \item
    \textbf{Charge to the Parents}
  \end{enumerate}

  It is then fitting that the minister give a charge to the parents in the following or like words:

  \begin{quote}
  Beloved in Christ Jesus, we give thanks to God for this child that He has given you, and for your expressed desire for {[}him/her{]} to know the Lord and to follow Him all {[}his/her{]} days. Along with the great blessing of the gift of this child have come responsibilities that you have just acknowledged and to which you have solemnly committed yourselves, and I charge you to continue steadfastly in the commitments that you have made today before God and these witnesses, humbly relying upon the grace of God in the diligent use of the means of grace---especially the Word of God, the sacraments, and prayer.
  \end{quote}

  \begin{enumerate}
  \def\labelenumii{\roman{enumii}.}
  \tightlist
  \item
    \textbf{Prayer}
    The whole service of baptism shall then be concluded with prayer. It is well in such prayer to thank the Lord for His covenant of grace, rejoice that this child has been included, and to ask the Lord to graciously enable him to be a covenant keeper, daily dying to sin and walking in newness of life in Christ.
  \end{enumerate}
\end{enumerate}

\hypertarget{a-note-on-infant-dedication}{%
\section*{58. A Note on Infant Dedication}\label{a-note-on-infant-dedication}}
\addcontentsline{toc}{section}{58. A Note on Infant Dedication}

\protect\hypertarget{chapter-slug-58-a-note-on-infant-dedication}{\href{}{}}

\protect\hypertarget{58}{\href{}{}}While infant dedication is not a sacrament, those who hold credo-baptist views may yet wish to present their child before the congregation, thanking God for his safe delivery, asking God's blessing upon him, and testifying to God's promises concerning our children. A suggested ceremony for infant dedication can be found in BCO \protect\hyperlink{79}{79}.

\hypertarget{the-baptism-of-those-professing-faith}{%
\section*{59. The Baptism of Those Professing Faith}\label{the-baptism-of-those-professing-faith}}
\addcontentsline{toc}{section}{59. The Baptism of Those Professing Faith}

\protect\hypertarget{chapter-slug-59-the-baptism-of-those-professing-faith}{\href{}{}}

\begin{enumerate}
\def\labelenumi{\arabic{enumi}.}
\item
  \protect\hypertarget{59}{\href{}{}}\textbf{Prerequisites}

  He who seeks to be baptized shall make a public profession of his faith before the congregation prior to the baptism. He shall previously have received instruction in the Christian faith in accordance with the confessional standards of this Church, including instruction as to the meaning of baptism, and have also made before the session of the church a credible profession of faith in Christ according to the provisions of BCO \protect\hyperlink{61.3}{61.3}.
\item
  \textbf{The Administration of Baptism to Those Professing Faith}

  \begin{enumerate}
  \def\labelenumii{\alph{enumii}.}
  \item
    \textbf{The Institution of the Sacrament}

    The minister ought to read the words of the institution of the sacrament of baptism from a passage such as Matthew 28:18-20.
  \item
    \textbf{The Covenant Commitment of the One Receiving Baptism and the Congregation}

    At the time of the service at which the person is to be baptized, he shall first profess his faith publicly before the assembled congregation. The minister may address him in these or like words:

    \begin{quote}
    Beloved in the Lord Jesus Christ, we thank our God for the grace that was given you, in that our Savior has sought and found you and through faith you have become a partaker of the covenant of grace. We rejoice that in His grace He has brought you to this congregation and given you the desire to profess your faith before us and to unite with us. We ask that you testify before us to the faith that you profess by giving assent to the following questions.
    \end{quote}

    To this end, the minister shall require the person to profess publicly his Christian faith by answering these questions in the affirmative:

    \begin{quote}
    \begin{enumerate}
    \def\labelenumiii{(\arabic{enumiii})}
    \tightlist
    \item
      Do you believe in God the Father Almighty, Maker of heaven and earth; and in Jesus Christ His only Son our Lord; and in the Holy Spirit, the Lord and Giver of life?
    \item
      Do you acknowledge yourself to be a sinner in the sight of God, justly deserving His wrath, and without hope apart from His sovereign mercy?
    \item
      Do you believe the Bible, consisting of the Old and New Testaments, to be the infallible Word of God, and its doctrine of salvation to be the perfect and only true doctrine of salvation?
    \item
      Do you believe in the Lord Jesus Christ as the Son of God, and Savior~of sinners, and do you receive and rest upon Him alone for salvation as~He is offered in the Gospel?
    \item
      Do you now resolve and promise, in humble reliance upon the grace of the Holy Spirit, that you will endeavor to live as a faithful follower of Christ?
    \item
      Do you promise to support the Church in its worship and work to the best of your ability, to submit yourself to its government and discipline, and to strive for its purity and peace?
    \end{enumerate}
    \end{quote}

    The minister shall then require the congregation to vow publicly their duty to welcome into Christian fellowship those being baptized:

    \begin{quote}
    Do you, the members of this congregation, welcome this disciple {[}\emph{or} these disciples{]} and promise by your fellowship to strengthen {[}his/her/their{]} tie{[}s{]} with the household of God? {[}If so, say, ``We do.''{]}
    \end{quote}

    If the session deems it appropriate, it may also ask him to bear brief testimony to his faith in his own words.

    After answers to the above questions in the affirmative, the minister shall proceed to the baptism.
  \item
    \textbf{The Meaning and Nature of the Sacrament}

    The minister shall then summarize before the congregation the teaching of the Word of God and the Confession and Catechisms of this church as to the meaning and nature of the sacrament of baptism. He may use these or like words:

    \begin{quote}
    The Lord Jesus Christ instituted baptism as a covenant sign and seal for His church. He uses it not only for the solemn admission of the person who is baptized into the visible church, but also to depict and to confirm His ingrafting of that person into Himself and His including that person in the covenant of grace.

    The Lord uses baptism to portray to us that we and our children are conceived and born in sin and need to be cleansed.

    He uses it to witness and seal to us the remission of sins and the bestowal of all the gifts of salvation through union with Christ. Baptism with water signifies and seals cleansing from sin by the blood and the Spirit of Christ, together with our death unto sin and our resurrection unto newness of life by virtue of the death and resurrection of Christ. Because these gifts of salvation are the gracious provision of the triune God, who is pleased to claim us as His very own, we are baptized in the name of the Father and of the Son and of the Holy Spirit.

    In our baptism, the Lord puts His name on us, claims us as His own, and summons us to assume the obligations of the covenant. He calls us to believe in Jesus Christ as our Savior, to renounce the devil, the world, and the flesh, and to walk humbly with our God in devotion to His commandments.
    \end{quote}
  \item
    \textbf{Prayer}

    Thereupon the minister shall pray for the presence and blessing of the triune God, that the grace signified and sealed by baptism may be abundantly realized.
  \item
    \textbf{The Baptism}

    Then, calling the person by name, he shall baptize him with water, without any other ceremony, saying:

    \begin{quote}
    {[}Name of Person{]}, I baptize you in the name of the Father and of the Son and of the Holy Spirit.
    \end{quote}
  \item
    \textbf{Welcome and Charge}

    It is then fitting that the minister address the baptized person in the following or like words:

    \begin{quote}
    Beloved, in the name of the Lord Jesus Christ I welcome you to all the privileges of full communion with God's people, and in particular to participation in the sacrament of the Holy Supper.

    I charge you to continue steadfastly in the confession that you have made, humbly relying upon the grace of God in the diligent use of the means of grace---especially the Word of God, the sacraments, and prayer.

    Rest assured that if you confess Christ before men, He will confess you before His Father who is in heaven.

    May the God of all grace, who called you unto His eternal glory in Christ, after you have suffered a little while, perfect, establish, and strengthen you. To Him be the glory and dominion for ever and ever. Amen.
    \end{quote}
  \item
    \textbf{Prayer}

    The whole service of baptism shall be concluded with prayer. It is well in such a prayer to thank the Lord for His covenant of grace, rejoice that this brother has been included, and to ask the Lord to graciously enable him to be a covenant keeper, daily dying to sin and walking in newness of life in Christ.
  \end{enumerate}
\end{enumerate}

\hypertarget{the-lords-supper}{%
\section*{60. The Lord's Supper}\label{the-lords-supper}}
\addcontentsline{toc}{section}{60. The Lord's Supper}

\protect\hypertarget{chapter-slug-60-the-lords-supper}{\href{}{}}

\begin{enumerate}
\def\labelenumi{\arabic{enumi}.}
\item
  \protect\hypertarget{60}{\href{}{}}\textbf{General Provisions of the Lord's Supper}

  The Lord's Supper is to be celebrated frequently, but the frequency may be determined by each session as it judges most conducive to edification. Where the Lord's Supper is celebrated less frequently, public notice should be given to the congregation, at least the Sabbath before the administration of this ordinance, and either then, or on some day of the week, the people may be instructed in its nature, and urged to make due preparation for it, that all may come in a suitable manner to this holy feast. In accord with the historical practice of the church, the use of a loaf and common cup is commended as a faithful and rich observation of Christ's commands for His Table. The senior pastor shall administer the Lord's Supper. If he is unable, an assistant or associate pastor may administer it.
\item
  \textbf{The Institution of the Sacrament}

  The minister shall read the words of the institution and instruction of the Lord's Supper as found in 1 Corinthians 11:23-29 or one of the Gospel accounts (Matthew 26:26-29, Mark 14:22-25, or Luke 22:14-20). In addition, he may read words of instruction from passages such as John 6 and 1 Corinthians 10.
\item
  \textbf{The Meaning and Nature of the Sacrament}

  The minister shall then summarize before the congregation the teaching of the Word of God as to the meaning and nature of the sacrament in the following or like words:

  \begin{quote}
  Our Lord Jesus Christ instituted the Lord's Supper as an ordinance to be observed by His church until He comes again. It is not a re-sacrificing of Christ, but is a remembrance of the once-for-all sacrifice of Himself in His death for our sins. Nor is it a mere memorial to Christ's sacrifice. It is a means of grace by which God feeds us with the crucified, resurrected, exalted Christ. He does so by His Holy Spirit and through faith. Thus He strengthens us in our warfare against sin and in our endeavors to serve Him in holiness. The sacrament further signifies and seals the forgiveness of our sin and our nourishment and growth in Christ. The bread and wine represent the crucified body and the shed blood of the Savior, which He gave for His people. In this sacrament, God confirms that He is faithful and true to fulfill the promises of His covenant, and He calls us to deeper gratitude for our salvation, to renewed consecration, and to more faithful obedience. The Supper is also a bond and pledge of the communion that believers have with Him and with each other as members of His body. As Scripture says, ``Since there is one bread, we who are many are one body; for we all partake of the one bread.''\footnote{1 Corinthians 10:17.} The Supper anticipates the consummation of the ages, when Christ returns to gather all His redeemed people at the glorious wedding feast of the Lamb. As we come to the Lord's Table, we humbly resolve to deny ourselves, to crucify the sin that is within us, to resist the devil, and to follow Christ as becomes those who bear His name.
  \end{quote}
\end{enumerate}

\begin{enumerate}
\def\labelenumi{\arabic{enumi}.}
\setcounter{enumi}{3}
\item
  \textbf{Invitation and Fencing the Table}

  The minister shall then declare who may come to, and who are excluded from, the Lord's Table according to the Word of God. He may use the following or like words:

  \begin{quote}
  We have heard, my brothers, how our Lord administered His Supper among His disciples, and in this He shows us that strangers, that is, those not of the company of the faithful should not be admitted. Following this rule, therefore, in the name and by the authority of our Lord Jesus Christ, I excommunicate all idolaters, blasphemers, despisers of God, heretics, and all who form separate parties to break the unity of the church, all perjurers, all those who rebel against their father and mother and against their superiors, all fomenters of sedition or mutiny, quarrelers, fighters, adulterers, fornicators, sexual deviants, thieves, lovers of money, plunderers, drunkards, gluttons, and all those who lead a scandalous life; declaring to those that they are to abstain from this holy table lest they pollute and contaminate this sacred food, which our Lord Jesus Christ gives only to His servants and faithful ones.

  Therefore, according to the exhortation of the Apostle Paul, let each one test and examine his own conscience, to know whether he truly repents of his faults and is sorry for them, desiring from now on to live in holiness and in conformity with God; and above all, whether he trusts in the mercy of God and seeks his salvation wholly from Jesus Christ; and whether renouncing all hostility and malice, he has the good intention and the courage to live in harmony and brotherly love with his neighbors.

  If we have such a testimony in our hearts before God, let us not doubt in the least that He acknowledges us to be His children and that the Lord Jesus is speaking to us, bringing us to His table and offering us this Holy Sacrament, which He delivered to His disciples.

  And since we are conscious of much frailty and misery in ourselves, as well as not having a perfect faith, but that we are prone rather to unbelief and distrust, so that we are not entirely dedicated to serving God and with such a zeal as we ought, but we have instead a battle daily against the lusts of our flesh; nevertheless, since our Lord has granted us this grace of having His Gospel engraved on our heart, so that we might resist all unbelief, and He has given us the desire and longing to renounce our own desires to pursue His righteousness and holy commandments; let us all be assured that the sins and imperfections that are in us will not prevent Him from receiving us, nor from making us worthy to share in this spiritual table. For we do not come insisting that we are perfect or righteous in ourselves, but rather, seeking our life in Jesus Christ, we confess that we are dead. Let us understand, therefore, that this Sacrament is a medicine for poor, spiritually sick people and that the only worthiness that our Lord requires of us is to know ourselves well enough to be displeased with our sins and to find all our pleasure, joy, and contentment in Him alone.\footnote{From John Calvin's Lord's Supper liturgy, 1542, 1566, modernized. See Mark Earngey and Jonathan Gibson, \emph{Reformation Worship} (New Growth Press, 2018), 326--27.}
  \end{quote}
\end{enumerate}

\begin{enumerate}
\def\labelenumi{\arabic{enumi}.}
\setcounter{enumi}{4}
\item
  \textbf{Exhortation}

  If desired, the minister may exhort the people of God, in the following or other words, to embrace in the sign the thing that is signified:

  \begin{quote}
  So let us first believe in these promises, which Jesus Christ, who is the infallible truth, spoke with His mouth, namely, that He truly wishes to make us partakers of His body and blood; that we might possess Him fully, so that He might live in us and we in Him. And since we see only bread and wine, yet we do not doubt that He accomplishes spiritually in our souls all that He demonstrates to us outwardly through these visible signs, namely, that He is the heavenly bread that feeds and nourishes us for eternal life. So let us be grateful for the infinite goodness of our Savior, who spreads out all His riches and goods on this table to distribute them to us. For by giving Himself to us, He testifies to us that all He has is ours. Therefore, let us receive this Sacrament as a seal that the power of His death and passion is imputed to us for righteousness, just as though we had suffered it ourselves. Let us therefore not be so wicked as to pull back from where Jesus Christ so gently invites us through His Word. But considering the worth of this precious gift which He has given us, let us present ourselves to Him with ardent zeal, so that He would make us able to receive it.\footnote{Ibid., 327--28.}

  Beloved congregation, lift up your hearts from these visible elements even to heaven itself, where Jesus Christ is seated at the right hand of the Father, from where we look for Him to return and perfect our redemption. All the promises of God are yes and amen in Him. Every spiritual blessing is found in Him. With joyful hearts, in Christian love, partake of His Table, giving thanks for the great love that He has shown to us.
  \end{quote}
\end{enumerate}

\begin{enumerate}
\def\labelenumi{\arabic{enumi}.}
\setcounter{enumi}{5}
\item
  \textbf{Prayer}

  The distribution of the elements shall be preceded by prayer. It is well in such prayer to praise God for His mighty power and grace in bringing salvation; confess our unworthiness to come to the Table because of our own utter lack of righteousness; reaffirm our trust in God's grace and in Christ's righteousness and mediation; plead for the Lord to grant the gracious, effectual working of His Spirit in us; thank God for the elements, request Him to use them for their intended purpose; and ask Him to grant that by faith His people may feed upon Jesus Christ, crucified and raised for them, so that, being strengthened by grace, they might live in Him and for Him.
\item
  \textbf{Partaking of the Elements}

  Normally,\footnote{This portion of the Lord's Supper liturgy follows the historical Reformed liturgies of the early Reformation. With the addition of the word ``normally,'' we do not intend to allow significant departure from this process of partaking of the elements. Intinction---the taking of the elements simultaneously by communicants---or the use of many individual cups does not \emph{significantly} depart from the outlined process. The significance of both the bread and the cup should be described prior to eating and drinking.} the elements of the Lord's Supper should be taken in the following manner.

  After prayer and thanksgiving, the minister shall take the bread, saying in the following or like words:

  \begin{quote}
  Our Lord Jesus Christ, the same night in which He was betrayed, took bread, blessed it, broke it, and gave it to His disciples, as I, ministering in His name, give this bread to you.
  \end{quote}

  The minister shall then break the bread and give it to the people; it shall be distributed to the people by church officers, preferably the ruling elders, but in any case male officers. The bread may be eaten either upon reception of it, or in unison when all have been served, as the session may judge most conducive to edification. The minister may continue, before the bread is eaten, saying:

  \begin{quote}
  Our Lord Jesus said, ``Take, eat, this is My body, which is for you; do this in remembrance of Me.''
  \end{quote}

  Having given the bread, the minister shall take the cup and give it to the people, saying in the following or like words:

  \begin{quote}
  In the same manner, our Savior also took the cup, and having given thanks as has been done in His name, He gave it to His disciples, as I ministering in His name give this cup to you.
  \end{quote}

  The minister shall then give the cup, as in the distribution of the bread; it shall be distributed to the people by church officers, preferably the ruling elders, but in any case male officers. The minister may continue, before the cup is drunk, saying:

  \begin{quote}
  Our Lord Jesus said, ``This cup is the new covenant in My blood, which is shed for many for the remission of sins; drink of it, all of you.''
  \end{quote}
\item
  \textbf{Response of Thanksgiving and Commitment}

  When all have partaken, prayer or a congregational song should be offered. It is well in such prayer to give thanks for the sacrifice of Jesus Christ, through whom we have the forgiveness of sins; recommit God's people to Christ and to each other; present them as a living sacrifice to God; and plead that the Holy Spirit will make the sacrament effectual to the edifying and strengthening of God's people.

  It is well that the congregation respond by singing a psalm or hymn that focuses on the benefits of Christ's death and the triumph of the Gospel, bringing forth gratitude and joy and renewed commitment of the believer to His Lord, and that an offering be taken for the relief of the poor or for some other sacred purpose.
\item
  \textbf{Blessing}

  The following benediction is particularly appropriate when the Lord's Supper has been celebrated:

  \begin{quote}
  ``Now the God of peace, who brought up from the dead the great Shepherd of the sheep through the blood of the eternal covenant, even Jesus our Lord, equip you in every good thing to do His will, working in us that which is pleasing in His sight, through Jesus Christ, to whom be the glory forever and ever. Amen.''\footnote{Hebrews 13:20-21.}
  \end{quote}
\end{enumerate}

\emph{The above exhortations and explanations are in accord with the historic Reformed liturgies.}

\hypertarget{public-reception-of-church-membersgeneral-provisions}{%
\section*{61. Public Reception of Church Members---General Provisions}\label{public-reception-of-church-membersgeneral-provisions}}
\addcontentsline{toc}{section}{61. Public Reception of Church Members---General Provisions}

\protect\hypertarget{chapter-slug-61-public-reception-of-church-members-general-provisions}{\href{}{}}

\begin{enumerate}
\def\labelenumi{\arabic{enumi}.}
\tightlist
\item
  \protect\hypertarget{61}{\href{}{}}Only those may be admitted to full communion in the church who have been baptized and have made public profession of faith in Jesus Christ.
\item
  In order to aid those who contemplate making public profession or reaffirmation of faith in Christ to understand the implication of this significant act and to perform it meaningfully, the pastor or someone approved by the session shall conduct classes in Christian doctrine and life, both for the children of the church and for any others who may manifest an interest in the way of salvation.
\item
  \protect\hypertarget{61.3}{\href{}{}}In order for the session to assure itself so far as possible that the candidate makes a credible profession, it shall examine him to ascertain that he possesses the knowledge requisite for saving faith in the Lord Jesus Christ, relies on the merits of Christ alone, and is determined by the grace of God to lead a Christian life.
\item
  In the public reception of church members, the minister shall follow the directions prescribed in this chapter, but he is not required to use the exact language of the forms (below), which are suggested as appropriate. He may employ these or similar forms, using his own liberty and godly wisdom as he deems best for the edification of the people.
\item
  Whether baptized or unbaptized, children of communing members are non-communing members of the particular church and come to the Lord's Table, also, by a credible profession of faith. Normally such professions should be actively sought in the regular course of the elder board's shepherding work. Because God has limited the ability of infants and very young children to articulate a credible profession of faith, demonstrate the fruit of repentance, and discern the Lord's body, they do not yet qualify for admission to communicant membership and the Lord's Table.\footnote{Calvin's process for allowing covenant children to come to the Lord's Supper is described by Scott Manetsch as follows: ``By the time that students completed their studies at the schola privata at age eleven or twelve, they would have worked through Calvin's \emph{Catechism} six or seven times and most would have mastered its doctrinal contents. It is important to note, however, that admission to the Lord's Table was not tied directly to a child's level of schooling or age. The \emph{Ecclesiastical Ordinances} made clear that boys and girls were welcomed to the Lord's Supper only after they had reached the age of discretion (around ten years of age) and were able satisfactorily to articulate the basic doctrines of the reformed religion and confess it as their own. Toward that end, four times a year on the Sunday before Geneva's quarterly communion service, young people who were prepared to confess their Christian faith stood in front of the worshiping congregation and, in response to the ministers' questions, recited the shortened form of the \emph{Catechism} that served as their formal profession of faith. A week later, they were invited for the first time to feed upon the body and blood of Jesus Christ offered in the sacred meal of the Lord's Table.'' \emph{Calvin's Company of Pastors: Pastoral Care and the Emerging Reformed Church, 1536-1609} (Oxford University Press, 2013), 270-71. Additionally: ``Another primary source, though, describes the age of discretion and Geneva's children coming to the Lord's table as between ages eight and ten: `Throughout all this, somebody else reads from the pulpit in the vernacular, with head uncovered, the Gospel of Saint John, from the beginning of the thirteenth chapter, until everyone has taken their pieces, both men and women, each one at their different tables, along with the boys and girls of around eight to ten years of age.'\,'' Ibid., 275, quoting Antoince Cathelan, \emph{Passevent Parisien}, 74.} For further explanation, see BCO \protect\hyperlink{1.3}{1.3}.
\end{enumerate}

\begin{enumerate}
\def\labelenumi{\arabic{enumi}.}
\setcounter{enumi}{5}
\tightlist
\item
  If they have been baptized, non-communing members of the congregation may be received into communicant membership by profession of faith.
\end{enumerate}

\hypertarget{reception-into-full-communion-of-non-communing-members-by-profession-of-faith}{%
\section*{62. Reception into Full Communion of Non-communing Members by Profession of Faith}\label{reception-into-full-communion-of-non-communing-members-by-profession-of-faith}}
\addcontentsline{toc}{section}{62. Reception into Full Communion of Non-communing Members by Profession of Faith}

\protect\hypertarget{chapter-slug-62-reception-into-full-communion-of-non-communing-members-by-profession-of-faith}{\href{}{}}

\begin{enumerate}
\def\labelenumi{\arabic{enumi}.}
\item
  \protect\hypertarget{62}{\href{}{}}When a non-communing member who was baptized as an infant is received into full communion, that reception is effective at the time of his public profession of faith. On the occasion of that person's public reception, it is highly advisable that the minister remind the people that he is already a member of the church, albeit a non-communing member, and has been receiving the blessings of Christ as a member of the church, and that those blessings have resulted in this day wherein, having given evidence of conscious saving faith in Christ, he is now about to confess that faith and become a communing member of the congregation. The minister may then address him in these or like words:

  \begin{quote}
  Beloved in the Lord Jesus Christ, we thank our God for the grace that was given you, in that you have accepted God's covenant promise that was signified and sealed unto you in your infancy by holy baptism. We ask you now to profess your faith publicly.
  \end{quote}
\item
  The minister shall then require the person to profess publicly his Christian faith by giving assent to these questions:

  \begin{quote}
  \begin{enumerate}
  \def\labelenumii{(\arabic{enumii})}
  \tightlist
  \item
    Do you believe in God the Father Almighty, Maker of heaven and earth; and in Jesus Christ His only Son our Lord; and in the Holy Spirit, the Lord and Giver of life?
  \item
    Do you acknowledge yourself to be a sinner in the sight of God, justly deserving His wrath, and without hope apart from His sovereign mercy?
  \item
    Do you believe the Bible, consisting of the Old and New Testaments, to be the infallible Word of God, and its doctrine of salvation to be the perfect and only true doctrine of salvation?
  \item
    Do you believe in the Lord Jesus Christ as the Son of God, and Savior~of sinners, and do you receive and rest upon Him alone for salvation as~He is offered in the Gospel?
    (5)\protect\hypertarget{62.2.5}{\href{}{}} Do you now resolve and promise, in humble reliance upon the grace of the Holy Spirit, that you will endeavor to live as a faithful follower of Christ?
  \item
    Do you promise to support the Church in its worship and work to the best of your ability, to submit yourself to its government and discipline, and to strive for its purity and peace?
  \end{enumerate}
  \end{quote}

  If the session deems it appropriate, it may also ask him to bear brief testimony to his faith in his own words.
\item
  It is appropriate that the minister exhort the congregation in these or like words:

  \begin{quote}
  From the time {[}Name{]} was baptized, the whole congregation has been obligated to love and receive {[}him/her{]} as a member of the church. As {[}he/she{]} is received into full communion, the congregation is reminded of these obligations. For in Christ we are members of one another. Christ claims this {[}brother/sister{]} as His own and calls you to receive {[}him/her{]} in love and commitment. Therefore, you ought to commit yourself before God to assist {[}Name{]} in {[}his/her{]} Christian nurture by godly example, prayer, and encouragement in our most precious faith and in the fellowship of believers.
  \end{quote}
\item
  When anyone has publicly professed his faith in this way, it is fitting that the minister address him in the following or similar words:

  \begin{quote}
  Beloved, in the name of the Lord Jesus Christ I welcome you to all the privileges of full communion with God's people, and in particular to participation in the sacrament of the Holy Supper.

  I charge you to continue steadfastly in the confession that you have made, humbly relying upon the grace of God in the diligent use of the means of grace---especially the Word of God, the sacraments, and prayer.

  Rest assured that if you confess Christ before men, He will confess you before His Father who is in heaven.

  May the God of all grace, who called you unto His eternal glory in Christ, after you have suffered a little while, perfect, establish, and strengthen you. To Him be the glory and dominion for ever and ever. Amen.
  \end{quote}

  This part of the service shall be concluded with prayer.
\end{enumerate}

\emph{For children of paedo-baptistic parents making a profession of faith in a credo-baptistic church, please see BCO \protect\hyperlink{66.3}{66.3}, \protect\hyperlink{66.4}{66.4}.}

\hypertarget{reception-by-letter-of-transfer-from-another-church-within-evangel-presbytery}{%
\section*{63. Reception by Letter of Transfer from Another Church within Evangel Presbytery}\label{reception-by-letter-of-transfer-from-another-church-within-evangel-presbytery}}
\addcontentsline{toc}{section}{63. Reception by Letter of Transfer from Another Church within Evangel Presbytery}

\protect\hypertarget{chapter-slug-63-reception-by-letter-of-transfer-from-another-church-within-evangel-presbytery}{\href{}{}}

\protect\hypertarget{63}{\href{}{}}When a person is received into membership on letter of transfer from another Evangel Presbytery congregation, that reception is effective at the time of the action of the session to receive him. Nevertheless, a session may deem it appropriate to welcome that person publicly into the congregation and allow him to give public expression to his faith. If this is done, it shall be made clear to the congregation that the person has already been received by action of the session. Nevertheless, the minister may address him in appropriate words similar to those found below in BCO \protect\hyperlink{64.4}{64.4}.

\hypertarget{reception-by-letter-of-transfer-from-another-church-of-like-faith-and-practice}{%
\section*{64. Reception by Letter of Transfer from Another Church of Like Faith and Practice}\label{reception-by-letter-of-transfer-from-another-church-of-like-faith-and-practice}}
\addcontentsline{toc}{section}{64. Reception by Letter of Transfer from Another Church of Like Faith and Practice}

\protect\hypertarget{chapter-slug-64-reception-by-letter-of-transfer-from-another-church-of-like-faith-and-practice}{\href{}{}}

\begin{enumerate}
\def\labelenumi{\arabic{enumi}.}
\item
  \protect\hypertarget{64}{\href{}{}}When a person is received into membership on letter of transfer from another church of like faith and practice approved by the session, that reception is effective at the time of his public profession of faith. On the occasion of that person's public reception, the minister shall address him in these or like words:

  \begin{quote}
  Beloved in the Lord Jesus Christ, we thank our God for the grace that was given you, in that you have accepted God's promise of salvation and publicly confessed your faith in the Savior, Jesus Christ. We praise Him that He brought you into communicant membership in a church of like faith and practice with this congregation. We rejoice that God, in His gracious providence, has now brought you here and given you a desire to unite with us, and that your former church has committed you to our fellowship and oversight. We ask that you testify before us to the faith that you profess by giving assent to the following questions.
  \end{quote}
\item
  The minister shall then require the person to profess publicly his Christian faith by giving assent to these questions:

  \begin{quote}
  \begin{enumerate}
  \def\labelenumii{(\arabic{enumii})}
  \tightlist
  \item
    Do you believe in God the Father Almighty, Maker of heaven and earth; and in Jesus Christ His only Son our Lord; and in the Holy Spirit, the Lord and Giver of life?
  \item
    Do you acknowledge yourself to be a sinner in the sight of God, justly deserving His wrath, and without hope apart from His sovereign mercy?
  \item
    Do you believe the Bible, consisting of the Old and New Testaments, to be the infallible Word of God, and its doctrine of salvation to be the perfect and only true doctrine of salvation?
  \item
    Do you believe in the Lord Jesus Christ as the Son of God, and Savior~of sinners, and do you receive and rest upon Him alone for salvation as~He is offered in the Gospel?
  \item
    Do you now resolve and promise, in humble reliance upon the grace of the Holy Spirit, that you will endeavor to live as a faithful follower of Christ?
  \item
    Do you promise to support the Church in its worship and work to the best of your ability, to submit yourself to its government and discipline, and to strive for its purity and peace?
  \end{enumerate}
  \end{quote}

  If the pastor deems it appropriate, he may also ask him to bear brief testimony to his faith in his own words.
\item
  The minister may exhort the congregation in these or like words:

  \begin{quote}
  As {[}Name{]} is received into full communion in the church, the whole congregation is obligated to receive {[}him/her{]}, for in Christ we are members of one another. Christ claims this {[}brother/sister{]} as His own and calls you to serve {[}him/her{]} in love. Therefore, you ought to commit yourself before God to assist {[}Name{]} in {[}his/her{]} Christian nurture by godly example, prayer, and encouragement in our most precious faith and in the fellowship of believers.
  \end{quote}
\item
  \protect\hypertarget{64.4}{\href{}{}}When anyone has publicly professed his faith in this way, it is fitting that the minister address him in the following or like words:

  \begin{quote}
  Beloved, in the name of the Lord Jesus Christ I welcome you to all the privileges of full communion with this congregation of God's people.

  I charge you to continue steadfastly in the confession that you have made, humbly relying upon the grace of God in the diligent use of the means of grace---especially the Word of God, the sacraments, and prayer.

  Rest assured that if you confess Christ before men, He will confess you before His Father who is in heaven.

  May the God of all grace, who called you unto His eternal glory in Christ, after you have suffered a little while, perfect, establish, and strengthen you. To Him be the glory and dominion for ever and ever. Amen.
  \end{quote}

  This part of the service shall be concluded with prayer.
\end{enumerate}

\hypertarget{reception-by-reaffirmation-of-faith}{%
\section*{65. Reception by Reaffirmation of Faith}\label{reception-by-reaffirmation-of-faith}}
\addcontentsline{toc}{section}{65. Reception by Reaffirmation of Faith}

\protect\hypertarget{chapter-slug-65-reception-by-reaffirmation-of-faith}{\href{}{}}

\begin{enumerate}
\def\labelenumi{\arabic{enumi}.}
\item
  \protect\hypertarget{65}{\href{}{}}When a person is received into membership by reaffirmation of faith, that reception is effective at the time of his public profession of faith. On the occasion of that person's public reception, the minister shall address him in these or like words:

  \begin{quote}
  Beloved in the Lord Jesus Christ, we thank our God for the grace that was given you, in that you have accepted God's promise of salvation and publicly confessed your faith in the Savior, Jesus Christ. We rejoice that God, in His gracious providence, has brought you into this congregation and given you a desire to reaffirm the faith that you have previously professed, and to unite with us. We ask that you testify before us to the faith that you profess by giving assent to the following questions.
  \end{quote}
\item
  The minister shall then require the person to profess publicly his Christian faith by giving assent to these questions:

  \begin{quote}
  \begin{enumerate}
  \def\labelenumii{(\arabic{enumii})}
  \tightlist
  \item
    Do you believe in God the Father Almighty, Maker of heaven and earth; and in Jesus Christ His only Son our Lord; and in the Holy Spirit, the Lord and Giver of life?
  \item
    Do you acknowledge yourself to be a sinner in the sight of God, justly deserving His wrath, and without hope apart from His sovereign mercy?
  \item
    Do you believe the Bible, consisting of the Old and New Testaments, to be the infallible Word of God, and its doctrine of salvation to be the perfect and only true doctrine of salvation?
  \item
    Do you believe in the Lord Jesus Christ as the Son of God, and Savior~of sinners, and do you receive and rest upon Him alone for salvation as~He is offered in the Gospel?
  \item
    Do you now resolve and promise, in humble reliance upon the grace of the Holy Spirit, that you will endeavor to live as a faithful follower of Christ?
  \item
    Do you promise to support the Church in its worship and work to the best of your ability, to submit yourself to its government and discipline, and to strive for its purity and peace?
  \end{enumerate}
  \end{quote}

  If the pastor deems it appropriate, he may also ask him to bear brief testimony to his faith in his own words.
\item
  It is appropriate that the minister exhort the congregation in these or like words:

  \begin{quote}
  As {[}Name{]} is received into full communion in the church, the whole congregation is obligated to receive (him/her), for in Christ we are members of one another. Christ claims this (brother/sister) as His own and calls you to serve (him/her) in love. Therefore, you ought to commit yourself before God to assist {[}Name{]} in (his/her) Christian nurture by godly example, prayer, and encouragement in our most precious faith and in the fellowship of believers.
  \end{quote}
\item
  When anyone has publicly professed his faith in this way, it is fitting that the minister address him in the following or like words:

  \begin{quote}
  Beloved, in the name of the Lord Jesus Christ I welcome you to all the privileges of full communion with this congregation of God's people.

  I charge you to continue steadfastly in the confession that you have made, humbly relying upon the grace of God in the diligent use of the means of grace---especially the Word of God, the sacraments, and prayer.

  Rest assured that if you confess Christ before men, He will confess you before His Father who is in heaven.

  May the God of all grace, who called you unto His eternal glory in Christ, after you have suffered a little while, perfect, establish, and strengthen you. To Him be the glory and dominion for ever and ever. Amen.
  \end{quote}

  This part of the service shall be concluded with prayer.
\end{enumerate}

\hypertarget{reception-of-new-members-by-public-profession-of-faith}{%
\section*{66. Reception of New Members by Public Profession of Faith}\label{reception-of-new-members-by-public-profession-of-faith}}
\addcontentsline{toc}{section}{66. Reception of New Members by Public Profession of Faith}

\protect\hypertarget{chapter-slug-66-reception-of-new-members-by-public-profession-of-faith}{\href{}{}}

\begin{enumerate}
\def\labelenumi{\arabic{enumi}.}
\item
  When an unbaptized person is received into membership by profession of faith, that reception is effective at the time of his public profession of faith and baptism. He shall be received in accord with BCO \protect\hyperlink{59}{59}. When a previously baptized person who is not a member of the congregation and has not previously made a confession of his faith is received into membership by profession of faith, that reception is effective at the time of his public profession of faith. On the occasion of that person's public reception, the minister shall address him in these or like words:

  \begin{quote}
  Beloved in the Lord Jesus Christ, we thank our God for the grace that was given you, in that our Savior has sought and found you and through faith you have become a partaker of the covenant of grace. We rejoice that in His grace He has brought you to this congregation and given you the desire to profess your faith before us and to unite with us. We ask that you testify before us to the faith that you profess by giving assent to the following questions.
  \end{quote}
\item
  The minister shall then require the person to profess publicly his Christian faith by giving assent to these questions:

  \begin{quote}
  \begin{enumerate}
  \def\labelenumii{(\arabic{enumii})}
  \tightlist
  \item
    Do you believe in God the Father Almighty, Maker of heaven and earth; and in Jesus Christ His only Son our Lord; and in the Holy Spirit, the Lord and Giver of life?
  \item
    Do you acknowledge yourself to be a sinner in the sight of God, justly deserving His wrath, and without hope apart from His sovereign mercy?
  \item
    Do you believe the Bible, consisting of the Old and New Testaments, to be the infallible Word of God, and its doctrine of salvation to be the perfect and only true doctrine of salvation?
  \item
    Do you believe in the Lord Jesus Christ as the Son of God, and Savior~of sinners, and do you receive and rest upon Him alone for salvation as~He is offered in the Gospel?
  \item
    Do you now resolve and promise, in humble reliance upon the grace of the Holy Spirit, that you will endeavor to live as a faithful follower of Christ?
  \item
    Do you promise to support the Church in its worship and work to the best of your ability, to submit yourself to its government and discipline, and to strive for its purity and peace?
  \end{enumerate}
  \end{quote}

  If the pastor deems it appropriate, he may also ask him to bear brief testimony to his faith in his own words.
\item
  \protect\hypertarget{66.3}{\href{}{}}It is appropriate that the minister exhort the congregation in these or like words:

  \begin{quote}
  As {[}Name{]} is received into full communion in the church, the whole congregation is obligated to receive (him/her), for in Christ we are members of one another. Christ claims this (brother/sister) as His own and calls you to serve (him/her) in love. Therefore, you ought to commit yourself before God to assist {[}Name{]} in (his/her) Christian nurture by godly example, prayer, and encouragement in our most precious faith and in the fellowship of believers.
  \end{quote}
\item
  \protect\hypertarget{66.4}{\href{}{}}When anyone has publicly professed his faith in this way, it is fitting that the minister address him in the following or like words:

  \begin{quote}
  Beloved, in the name of the Lord Jesus Christ I welcome you to all the privileges of full communion with God's people, and in particular to participation in the sacrament of the Holy Supper.

  I charge you to continue steadfastly in the confession that you have made, humbly relying upon the grace of God in the diligent use of the means of grace---especially the Word of God, the sacraments, and prayer.

  Rest assured that if you confess Christ before men, He will confess you before His Father who is in heaven.

  May the God of all grace, who called you unto His eternal glory in Christ, after you have suffered a little while, perfect, establish, and strengthen you. To Him be the glory and dominion for ever and ever. Amen.
  \end{quote}

  This part of the service shall be concluded with prayer.
\end{enumerate}

\hypertarget{other-occasions-of-public-worship}{%
\section*{67. Other Occasions of Public Worship}\label{other-occasions-of-public-worship}}
\addcontentsline{toc}{section}{67. Other Occasions of Public Worship}

\protect\hypertarget{chapter-slug-67-other-occasions-of-public-worship}{\href{}{}}

\protect\hypertarget{67}{\href{}{}}Under the Gospel, we are commanded to keep no other particular day holy, except the Lord's Day. Nevertheless, God's people may observe other occasions of worship as the Lord allows. Such observance is both consonant with Scripture and pastorally appropriate.

\hypertarget{prayer-meetings}{%
\section*{68. Prayer Meetings}\label{prayer-meetings}}
\addcontentsline{toc}{section}{68. Prayer Meetings}

\protect\hypertarget{chapter-slug-68-prayer-meetings}{\href{}{}}

\protect\hypertarget{68}{\href{}{}}Prayer meetings for the purpose of corporate prayer should be held under the direction of the Session. They may be conducted by the Pastor or member of the session. They may also be conducted by other members of the church when so authorized by the session. The exercises appropriate for the prayer meeting are prayer, reading of Scriptures, instruction and exhortation by men. All of the people of the church, young and old, male and female, should be encouraged to pray aloud in public and so have an active part in the prayer meeting.

\hypertarget{prayer-and-fasting}{%
\section*{69. Prayer and Fasting}\label{prayer-and-fasting}}
\addcontentsline{toc}{section}{69. Prayer and Fasting}

\protect\hypertarget{chapter-slug-69-prayer-and-fasting}{\href{}{}}

\begin{enumerate}
\def\labelenumi{\arabic{enumi}.}
\tightlist
\item
  \protect\hypertarget{69}{\href{}{}}When great and notable calamities come upon or threaten the church, community, or nation, when judgment is deserved because of sin, when the people seek some special blessing from the Lord, or when a pastor is to be ordained or installed, it is fitting that the people of God engage in times of solemn prayer and fasting.
\item
  Prayer and fasting may be observed by private individuals and families at their discretion or by the Church at the discretion of the appropriate judicatory. If the civil authority calls for a time of prayer and fasting that the judicatories of the Church deem to be in harmony with the Scriptures, they should consider issuing such a call to their members.
\item
  Public notice is to be given before the time of prayer and fasting comes, to enable persons to order their temporal affairs so that they can participate.
\item
  It is especially appropriate on days of prayer and fasting called by the Church that the people of God gather for a time of prayer, the singing of psalms and hymns, and the reading and preaching of the Word of God. Let them lament their distress or unworthiness before the Lord, confess their sins, humbly implore the Lord for deliverance from the judgment present or imminent or for the blessing sought, and commit themselves anew to the faithful service of the Lord their God. It is fitting on such days that God's people abstain from food and from such activities as may distract from their solemn engagement in prayer.
\end{enumerate}

\hypertarget{thanksgiving}{%
\section*{70. Thanksgiving}\label{thanksgiving}}
\addcontentsline{toc}{section}{70. Thanksgiving}

\protect\hypertarget{chapter-slug-70-thanksgiving}{\href{}{}}

\begin{enumerate}
\def\labelenumi{\arabic{enumi}.}
\tightlist
\item
  \protect\hypertarget{70}{\href{}{}}1. When God's blessings on the church, community, or nation are particularly evident, it is fitting that the people of God engage in special times of thanksgiving.
\item
  Special times of thanksgiving may be observed by private individuals and families at their discretion or by the Church as called by the appropriate judicatory. If the civil authority calls for a time of thanksgiving that the judicatories of the Church deem to be in harmony with the Scriptures, they should consider issuing such a call to their members.
\item
  Public notice is to be given before the day of thanksgiving comes, to enable persons to order their temporal affairs so that they can participate.
\item
  It is especially appropriate on special days of thanksgiving called by the Church that the people of God gather for prayer, testimony to God's blessings, joyful singing of psalms and hymns, and the reading and preaching of the Word of God. Let them give thanks to God for His goodness to His people and especially for the greatness of His mercies to them in Christ. And let them commit themselves anew to the faithful service of the Lord their God in gratitude for His blessings. It is fitting on such days that God's people spend the day in expressions of Christian love and charity toward one another, rejoicing more and more in the Lord, as becomes those who make the joy of the Lord their strength. Also, they may feast together before the Lord with joy and thanksgiving.
\end{enumerate}

\hypertarget{home-fellowship-groups}{%
\section*{71. Home Fellowship Groups}\label{home-fellowship-groups}}
\addcontentsline{toc}{section}{71. Home Fellowship Groups}

\protect\hypertarget{chapter-slug-71-home-fellowship-groups}{\href{}{}}

\begin{enumerate}
\def\labelenumi{\arabic{enumi}.}
\tightlist
\item
  \protect\hypertarget{71}{\href{}{}}It is fitting for church members to gather in each other's homes on the Lord's Day and other days, eating together, discussing and applying the preaching of God's Word, studying the Scriptures, singing praises, confessing sins, praying for one another and bearing one another's burdens.
\item
  Such a gathering is no substitute for the public worship of the local church on the Lord's Day.
\item
  Officers of the church may lead these groups or leadership may be delegated to other faithful men. Pastors and elders should maintain frequent communication with group leaders, both to ensure that those leaders are leading a holy life, and to be made aware of pastoral needs of those involved in the group.
\end{enumerate}

\hypertarget{the-solemnization-of-marriage}{%
\section*{72. The Solemnization of Marriage}\label{the-solemnization-of-marriage}}
\addcontentsline{toc}{section}{72. The Solemnization of Marriage}

\protect\hypertarget{chapter-slug-72-the-solemnization-of-marriage}{\href{}{}}

\begin{enumerate}
\def\labelenumi{\arabic{enumi}.}
\tightlist
\item
  \protect\hypertarget{72}{\href{}{}}Marriage is a divine institution though not a sacrament, nor peculiar to the Church of Christ. It is proper that every commonwealth, for the good of society, make laws to regulate marriage, which all citizens are bound to obey insofar as they do not transgress the laws of God.\footnote{Acts 5:29.}
\end{enumerate}

\begin{enumerate}
\def\labelenumi{\arabic{enumi}.}
\setcounter{enumi}{1}
\tightlist
\item
  Christians should marry in the Lord; therefore it is appropriate that their marriage be solemnized by a lawful minister, that special instruction be given them, and suitable prayers offered, when they enter into this relation.
\item
  \protect\hypertarget{72.3}{\href{}{}}During the solemnization of the marriage, the three Scriptural purposes of marriage as described by the Westminster Confession of Faith shall be stated: ``Marriage was ordained for the mutual help of husband and wife, for the increase of mankind with a legitimate issue, and of the Church with an holy seed; and for preventing of uncleanness.'' Additionally, the man is required to vow to love and cherish the woman, and the woman is required to vow to love, cherish, and obey the man.
\item
  Marriage is to be a lifelong, monogamous union between one man and one woman,\footnote{For the intended meaning of the words ``man'' and ``woman,'' please refer to BCO \protect\hyperlink{28}{28}, particularly \protect\hyperlink{28.7}{section 7}.} in accordance with the Word of God.
\end{enumerate}

\begin{enumerate}
\def\labelenumi{\arabic{enumi}.}
\setcounter{enumi}{4}
\tightlist
\item
  The parties should be of such years of discretion as to be capable of making their own choice; and if they be under age, or live with their parents, the consent of the parents or others, under whose care they are, should be previously obtained, and well proven to the minister before he proceeds to solemnize the marriage.
\item
  Parents should neither compel their children to marry contrary to their will, nor deny their consent without just and significant reasons.
\item
  Marriage and the welfare of civil society cannot be separated. The happiness of families and the credit of Christianity are deeply interested in it. Therefore, the intent to marry should be sufficiently published for objections to the marriage to be registered in a timely way, and for them to be adjudicated.\footnote{This requirement is based upon the ancient practice of ``publishing the banns.'' In Calvin's Geneva, for example, it was required that the minister publish the banns---that is, announce to the congregation the intent of a couple to marry---for the three Sundays preceding the wedding. This announcement allowed those who objected to the wedding to explain their reasons to the minister. It also taught the congregation about the public and solemn nature of the wedding ceremony. For more on Calvin's practice, see John Witte Jr.~and Robert M. Kingdon, \emph{Courtship, Engagement, and Marriage}, vol.~1 of \emph{Sex, Marriage, and Family Life in John Calvin's Geneva} (Eerdmans, 2005).} Ministers should be careful that they obey the laws of the community to the extent that those laws do not transgress the laws of God as declared by the Constitution of Evangel Presbytery; and that they not destroy the peace and comfort of families, ministers should be assured that, with respect to the parties applying to them, no just objections lie against their marriage.
\end{enumerate}

\hypertarget{the-visitation-of-the-sick}{%
\section*{73. The Visitation of the Sick}\label{the-visitation-of-the-sick}}
\addcontentsline{toc}{section}{73. The Visitation of the Sick}

\protect\hypertarget{chapter-slug-73-the-visitation-of-the-sick}{\href{}{}}

\begin{enumerate}
\def\labelenumi{\arabic{enumi}.}
\tightlist
\item
  \protect\hypertarget{73}{\href{}{}}The power of the prayer of faith is great, and Christians therefore should pray for the sick at the throne of heavenly grace, and should also seek God's blessing upon all proper means which are being employed for their recovery. Moreover, when persons are sick, their minister, or some officer of the church, should be notified, that the minister, officers, and members may unite their prayers for the sick. It is the privilege and duty of the pastor to visit the sick and to minister to their physical, mental, and spiritual welfare.
\item
  In view of the varying circumstances of the sick, the minister should use discretion in the performance of this duty. In some circumstances, the minister will be wise to encourage women to visit other women who are sick or in need of care. For those unable to attend services for extended periods of time, the sacraments may be administered to them by a senior pastor or associate pastor only when a subset of the members of the congregation is gathered in their residence. A short exposition of the Word of God should always accompany the application of baptism or the celebration of the Lord's Supper. Those gathered, and in good standing, should all partake of the Lord's Supper together.
\end{enumerate}

\hypertarget{worship-in-the-home}{%
\section*{74. Worship in the Home}\label{worship-in-the-home}}
\addcontentsline{toc}{section}{74. Worship in the Home}

\protect\hypertarget{chapter-slug-74-worship-in-the-home}{\href{}{}}

\begin{enumerate}
\def\labelenumi{\arabic{enumi}.}
\tightlist
\item
  \protect\hypertarget{74}{\href{}{}}In addition to public worship, it is the duty of each person in private, and of every family in private, to worship God.
\item
  Private worship is commanded by our Lord. In this duty everyone, should spend some time in prayer, reading the Scriptures, holy meditation, and serious self-examination.
\item
  Family worship, which should be observed by every family, should consist in prayer, reading the Scriptures, and singing praises, even if briefly.
\item
  Parents should instruct their children in the Word of God, and in the principles of our holy religion. The reading of devotional literature should be encouraged and every proper opportunity for biblical instruction should be gladly embraced.
\item
  Parents should set an example of piety and consistent living before the family. The Lord's Day should be sanctified and left free for worship, rest, and works of mercy.
\item
  In the task of Christian education, parents should cooperate with the Church by setting their children an example in regular and punctual attendance at religious instruction provided by the church and services of worship, by assisting them in the preparation of their lessons, and by leading them in the consistent application of the teachings of the Gospel in their daily activities.
\end{enumerate}

\hypertarget{suggested-form-for-the-solemnization-of-marriage}{%
\section*{75. Suggested Form for the Solemnization of Marriage}\label{suggested-form-for-the-solemnization-of-marriage}}
\addcontentsline{toc}{section}{75. Suggested Form for the Solemnization of Marriage}

\protect\hypertarget{chapter-slug-75-suggested-form-for-the-solemnization-of-marriage}{\href{}{}}

\protect\hypertarget{75}{\href{}{}}\emph{The requirements of BCO \protect\hyperlink{72}{72}, particularly \protect\hyperlink{72.3}{section 3}, are required regardless of what liturgy is used for the marriage ceremony.}

\emph{The form that follows in this section is based upon the marriage solemnization found in the 1549 Book of Common Prayer}

Prelude

Seating of Mothers

Bridal Party's Entrance

Bride's Entrance

Words of Institution

\begin{center}
\emph{The Minister says,}

\end{center}

\begin{quote}
Dearly beloved, we are gathered here together in the sight of God, and in the presence of these witnesses, to join together this man and this woman in holy matrimony. Marriage is an honorable estate, instituted by God in the garden in the time of man's innocence, before the corruption of sin, and signifying unto us the mystical union between Christ and His Church. This holy estate Christ adorned and beautified with His presence and first miracle in Cana of Galilee, and is commended by St.~Paul to be honorable among all men.

Therefore, it is not to be entered into lightly or unadvisedly, nor is it to be undertaken loosely as satisfaction for carnal lusts, as if men were unrestrained animals with no understanding; but rather, it is to be sought reverently, discreetly, advisedly, soberly, and in the fear of God. In ordaining the gift of marriage, our Lord has commended to us three purposes for it. First, marriage is given for the mutual society, help, and comfort that man and wife should have from the other, both in prosperity and adversity. Second, marriage is given for the procreation of children, to be brought up in the fear and nurture of the Lord, and to the praise of God. Third, marriage is instituted as a remedy against sin, and to avoid fornication, so that those who are married might live in the purity of marriage, and keep themselves undefiled as members of Christ's body.

Into this holy estate this man and this woman come now to be joined.

If any man can show just cause why they may not lawfully be joined, let him speak now, or else hereafter forever hold his peace.
\end{quote}

\begin{center}
\emph{And also speaking to the persons to be married, he shall say,}

\end{center}

\begin{quote}
I require and charge you both, that if either of you know any reason why you may not lawfully be joined in matrimony, you tell it to me. For you can be certain that all who are joined together in any way besides what God's word allows are not joined by God, nor is their marriage lawful, and they will answer on the dreadful day of judgement, when the secrets of all hearts shall be revealed.
\end{quote}

\begin{center}
\emph{At this time, if anyone raises an objection why the couple may not be joined together in matrimony, and is willing to stand by his allegation while it is investigated, then the marriage must be deferred, until the truth is found.}

\end{center}

Declaration of Consent

\begin{center}
\emph{If no objection is raised, then the minister shall say unto the man,}

\end{center}

\begin{quote}
{[}Name of Groom{]}, will you have this woman to be your wedded wife, to live together after God's ordinance in the holy estate of matrimony? Will you love her, comfort her, honor, and keep her in sickness and in health? And, forsaking all others, keep yourself only unto her, so long as you both shall live?
\end{quote}

\begin{center}
\emph{The man shall answer,}

\end{center}

\begin{quote}
I will.
\end{quote}

\begin{center}
\emph{Then the minister shall say to the woman,}

\end{center}

\begin{quote}
{[}Name of Bride{]}, will you have this man to be your wedded husband, to live together after God's ordinance, in the holy estate of matrimony? Will you obey him, and serve him, love, honor, and keep him in sickness and in health? And, forsaking all others, keep yourself only unto him, so long as you both shall live?
\end{quote}

\begin{center}
\emph{The woman shall answer,}

\end{center}

\begin{quote}
I will.
\end{quote}

\begin{center}
\emph{Then shall the Minister say,}

\end{center}

\begin{quote}
Who gives this woman to be married to this man?
\end{quote}

\begin{center}
\emph{The father of the bride or a man standing in his place shall say,}

\end{center}

\begin{quote}
I do.
\end{quote}

\begin{center}
\emph{Then the minister, receiving the woman at her father's or friend's hands, will cause the man to take the woman by the right hand.}

\end{center}

Invocation

\begin{quote}
Let us pray. Almighty and ever blessed God, whose presence is the happiness of every condition and whose favor makes holy every relation; We ask You to be present and to look with favor upon these Your servants, as together they make their covenant before You. As You have brought them together by Your providence, sanctify them by Your Spirit, giving them a new frame of heart fit for their new estate. Pour out Your grace upon them so that they may enjoy the comforts, bear up under the cares, endure the trials, and perform the duties of life together under Your heavenly guidance and protection. May they be truly joined in the honorable estate of marriage, and may they know the salvation of the only true God. We ask these things through our Lord Jesus Christ. Amen.
\end{quote}

\begin{center}
\emph{The Minister, Bride and Groom, Maid of Honor and Best Man, ascend to platform.}

\end{center}

Song

Scripture Reading (for example, Genesis 2:18-25, Ephesians 5:22-33)

Sermon

Exchange of Vows
::: \{.center data-latex=``\,``\}
\emph{The man and the woman turn and face each other and join hands. The man will say, repeating after the minister,}
:::

\begin{quote}
I {[}Name of Groom{]} take you {[}Name of Bride{]} to be my lawful wedded wife, to have and to hold from this day forward, for better, for worse, for richer, for poorer, in sickness, and in health, to love and to cherish, till death us do part: according to God's holy ordinance; and to this vow I pledge you my faithfulness.
\end{quote}

\begin{center}
\emph{The woman will say, repeating after the minister,}

\end{center}

\begin{quote}
I {[}Name of Bride{]} take you {[}Name of Groom{]} to be my lawful wedded husband, to have and to hold from this day forward, for better, for worse, for richer, for poorer, in sickness, and in health, to love, cherish, and to obey, till death us do part: according to God's holy ordinance; and to this vow I give you my faithfulness.
\end{quote}

Ring Ceremony

\begin{center}
\emph{The minister will say to the man,}

\end{center}

\begin{quote}
{[}Name of Groom{]}, what keepsake do you give {[}Name of Bride{]} of your promise to fulfill these vows you have made?
\end{quote}

\begin{center}
\emph{The best man gives the ring to the groom; the groom says, ``This ring.'' The groom takes the bride's left hand the puts the ring on her fourth finger. The man says, repeating after the minister,}

\end{center}

\begin{quote}
As a pledge and symbol of the vows we have made, with this ring I thee wed, with my body I thee honor: and all my worldly goods I give to thee. In the name of the Father, and of the Son, and of the Holy Spirit. Amen.
\end{quote}

\begin{center}
\emph{The minister will say to the woman,}

\end{center}

\begin{quote}
{[}Name of Bride{]}, what keepsake do you give {[}Name of Groom{]} of your promise to fulfill these vows you have made?
\end{quote}

\begin{center}
\emph{The maid of honor gives the ring to the bride; the bride says, ``This ring.'' The bride takes the groom's left hand the puts the ring on his fourth finger. The woman says, repeating after the minister,}

\end{center}

\begin{quote}
As a pledge and symbol of the vows we have made, with this ring I am wedded to thee, in the name of the Father, and of the Son, and of the Holy Spirit. Amen.
\end{quote}

Prayer

\begin{quote}
Let us pray.

O Eternal God, Father of our Lord Jesus Christ, Creator and Preserver of all mankind, Giver of all spiritual grace, and the Author of everlasting life: send Your blessing upon these Your servants, this man and this woman, whom we bless in Your name; that as Isaac and Rebecca lived faithfully together in the bonds of matrimony, so may this man and wife surely perform and keep the vow and covenant made between them, of which these rings are a token and pledge. And may they ever remain in perfect love and peace together, and live according to Your laws, through Jesus Christ our Lord. Amen.
\end{quote}

Pronouncement
::: \{.center data-latex=``\,``\}
\emph{Then the minister shall speak to the people,}
:::

\begin{quote}
Forasmuch as {[}Name of Groom{]} and {[}Name of Bride{]} have consented together to holy wedlock, and have witnessed their vow here before God and this company; and have given and pledged their faithfulness to each other by the joining of hands and exchanging of rings: By the authority committed unto me as a Minister of the Church of Jesus Christ, I pronounce that this man and this woman be man and wife, according to the ordinance of God. In the name of the Father, and of the Son, and of the Holy Spirit. Amen.
\end{quote}

\begin{center}
\emph{Then the minister shall join their right hands together, and say,}

\end{center}

\begin{quote}
Those whom God has joined together, let no man separate.\footnote{Matthew 19:6.}

{[}Name of Groom{]}, you may kiss your bride.
\end{quote}

Presentation of the Couple
::: \{.center data-latex=``\,``\}
\emph{The couple turns and faces the congregation. The minister shall say,}
:::

\begin{quote}
I now have the honor of presenting to you: Mr.~and Mrs.~\_\_\_\_\_\_\_\_\_\_\_\_\_\_\_\_\_\_\_\_\_\_\_.
\end{quote}

Recessional

\hypertarget{suggested-form-for-a-funeral}{%
\section*{76. Suggested Form for a Funeral}\label{suggested-form-for-a-funeral}}
\addcontentsline{toc}{section}{76. Suggested Form for a Funeral}

\protect\hypertarget{chapter-slug-76-suggested-form-for-a-funeral}{\href{}{}}

\protect\hypertarget{76}{\href{}{}}\emph{Because the funeral is a service of worship of the Lord Jesus Christ, eulogy of the deceased~shall be given prior to the call to worship at the beginning of worship. Similarly, if the family and friends wish to share remembrances, these shall precede the worship service.}

\emph{The following form is from The Book of Church Order of the Presbyterian Church in the United States (1933), modernized.}

Scripture Readings

\begin{quote}
``Jesus said to her, `I am the resurrection and the life; he who believes in Me will live even if he dies, and everyone who lives and believes in Me will never die. Do you believe this?'\,''\footnote{John 11:25-26.}

``\,`Man, who is born of woman, is short-lived and full of turmoil. Like a flower he comes forth and withers. He also flees like a shadow and does not remain.'\,''\footnote{Job 14:1-2.}

``For we are sojourners before You, and tenants, as all our fathers were; our days on the earth are like a shadow, and there is no hope.''\footnote{1 Chronicles 29:15.}

``For we have brought nothing into the world, so we cannot take anything out of it either.''\footnote{1 Timothy 6:7.}

``\,`The LORD gave and the LORD has taken away. Blessed be the name of the LORD.'\,''\footnote{Job 1:21b.}
\end{quote}

Prayer of Invocation

\begin{quote}
Let us pray.

O God, who is our God, and our fathers' God; You whose compassions fail not, but who is the same yesterday, today, and forever, grant us now Your presence, we beseech You, that our souls may be strengthened, and that we faint not under Your afflicting providence, but that through Your condescension may find all grace to help in this our time of need, which we ask in the name of Jesus Christ, our Lord and Saviour, to whom, with You and the Holy Spirit, we will ascribe all honor, majesty and might, world without end. Amen.
\end{quote}

Song

Scripture Readings (for example, Psalm 39:4-13; Psalm 90:1-12; 1 Corinthians 15:20-58. Instead of the foregoing passage from 1 Corinthians 15, one or more of the following may be substituted as the occasion may require: Ecclesiastes 12; Psalm 27; Revelation 22:1-5.)

Preaching of God's Word

Prayer

\begin{quote}
Let us pray.

Almighty and most merciful God, our heavenly Father, the consolation of the sorrowful and the support of the stricken, who does not willingly afflict the children of men, look in pity, we ask You, on all upon whom You have laid Your afflicting hand, and, in the multitude of Your tender mercies, be pleased to uphold and comfort them in the day of their trial and distress. Grant us all grace that we may lay to heart the lesson of this solemn providence, and work while the day lasts, knowing that the night comes, when no man can work; and that we may set our affections on things that are in heaven, and not on things that are on the earth. Enable us to live by faith on the Son of God, that when Christ, who is our life, shall appear, we also may appear with Him in glory.

O Lord Jesus Christ, Son of God, Lamb of God, which takes away the sin of the world, to whom shall we go but to You? You have the words of eternal life. You who were a Man of Sorrows and acquainted with grief, have pity upon those who cry to You. When our eyes grow dim in the shadows of death, and we pass through the deep waters, by Your agony and bloody sweat, and by Your death on Calvary, we ask You to remember us. O You who saved us, do not forsake us in the trying hour; You who vanquished death, give us the victory, and bring us to Your own everlasting rest in the assembly of Your saints on high.

O God, the Holy Spirit, Author of light and life and truth, inspire our souls with hope through the Gospel of our Lord Jesus Christ, imparting the benefits of His atonement, and the power of His all-sufficient grace. Release us from our sins; fill us with the fruits of Your own indwelling, and form us anew in the image of God. Help us now, O blessed Comforter; heal our wounded spirits and do not despise our broken and contrite hearts.

O God the Father, God the Son, and God the Holy Spirit, Triune Jehovah, have mercy upon us, Your servants, as we wait before You: and hear our prayer. Be pleased graciously to attend to our humble requests, and to do for us all that we need, glorifying Yourself by us both in this present world, and in that which is to come: all of which we ask through Jesus Christ our Lord. Amen.
\end{quote}

Lord's Prayer

\begin{quote}
Our Father who art in heaven, Hallowed be Thy name. Thy kingdom come. Thy will be done on earth as it is in heaven. Give us this day our daily bread. And forgive us our debts, as we forgive our debtors. And lead us not into temptation, but deliver us from evil. For Thine is the kingdom and the power and the glory forever. Amen.
\end{quote}

Benediction

\begin{quote}
``The grace of the Lord Jesus Christ, and the love of God, and the fellowship of the Holy Spirit, be with you all.''\footnote{2 Corinthians 13:14.} Amen.
\end{quote}

\hypertarget{suggested-form-for-a-childs-funeral}{%
\section*{77. Suggested Form for a Child's Funeral}\label{suggested-form-for-a-childs-funeral}}
\addcontentsline{toc}{section}{77. Suggested Form for a Child's Funeral}

\protect\hypertarget{chapter-slug-77-suggested-form-for-a-childs-funeral}{\href{}{}}

\protect\hypertarget{77}{\href{}{}}\emph{Because the funeral is a service of worship of the Lord Jesus Christ, eulogy of the deceased~shall be given prior to the call to worship at the beginning of worship. Similarly, if the family and friends wish to share remembrances, these shall precede the worship service.}

\emph{The following form is from The Book of Church Order of the Presbyterian Church in the United States (1933), modernized.}

Scripture Readings

\begin{quote}
May the LORD answer you in the day of trouble!\\
May the name of the God of Jacob set you securely on high!\\
May He send you help from the sanctuary\\
And support you from Zion!\footnote{Psalm 20:1-2.}

``\,`Man, who is born of woman, is short-lived and full of turmoil. Like a flower he comes forth and withers. He also flees like a shadow and does not remain.'\,''\footnote{Job 14:1-2.}

``\,`The LORD gave and the LORD has taken away. Blessed be the name of the LORD.'\,''\footnote{Job 1:21b.}

``\,`Come to Me, all who are weary and heavy-laden, and I will give you rest.'\,''\footnote{Matthew 11:28.}
\end{quote}

Prayer

\begin{quote}
Father of mercies, God of all grace, ever comforting us by the tender assurance of Your love for all those whom you chasten, be near to us now in this hour of sorrow, as we come to cast our care upon You, and seek the strength and consolation You only can impart.

As a father pities his children, so do You pity those who sit before You, smitten and afflicted. As one whom his mother comforts, do You comfort them, and so sanctify to them this sorrow that theirs may be everlasting consolation.

O You who leads Joseph like a flock, who knows Your own sheep by name as they follow You, carrying the lambs in Your arms and folding them in Your bosom, it is not Your will that one of these little ones should perish. When You do send your messenger, like a gentle shepherd, to lead them into the heavenly pastures, may bereaved parents hear the voice which says, ``Let the children alone, and do not hinder them from coming to Me.''\footnote{Matthew 19:14.} Fill them with resignation to Your will; give them the consolations of Your Spirit, and grant that, through Your grace, this chastening may be for their profit, that, being made partakers of Your holiness, they may be prepared for everlasting blessedness in that world where, after the separations and sorrows of this life, they may be forever with one another and with the Lord, through the merits and mediation of Jesus Christ, Your Son, our Savior. Amen.
\end{quote}

Song

Scripture Readings (for example, Psalm 23; 2 Samuel 12:16-23; Isaiah 51:12, 66:13; John 13:7; Hebrews 12:5-7, 11-12; Romans 8:15-18; John 14:1-2; Isaiah 40:1, 11; Mark 10:13-16; Matthew 18:10-14; Rev.~7:15-17)

Preaching of God's Word

Prayer

\begin{quote}
O God, our heavenly Father, who through the blood of Your Son has provided redemption for all Your own, we would render to You most hearty thanks, in this our time of grief, for the hope we have that the soul of this dear child whose loss we mourn is at rest in You. Not a sparrow falls to the ground without our Father, and those who are of more value than many sparrows cannot die until You, who regards with tender compassion the weakest of Your creatures, call them to Yourself. We would not sorrow as those who have no hope, but bow in humble submission to Your sovereign decree, and by divine grace would say, Your will be done.

Grant to Your servants, we implore You, the consolations of Your Spirit, giving us beauty for ashes, the oil of joy for mourning, and the garment of praise for the spirit of heaviness. May this chastisement which now seems so grievous yield us the peaceable fruits of righteousness, by drawing us into closer fellowship with You, that we may not set our affections on the things of this world; but upon that blessed home above, where all who have departed in Christ await us beyond the reach of sorrow.

Shine upon our darkness, O Lord; pardon all our sins; build us up and strengthen us in our most holy faith; and at last give us the victory over death, bringing us in holiness and joy to Your own eternal rest.

Hear now our prayer, O God, and be pleased mercifully to bestow the blessings which we need, for the love of Jesus Christ, Your well-beloved Son, to whom, with You and the Holy Spirit, be all dominion, glory and praise, world without end. Amen.
\end{quote}

Lord's Prayer

\begin{quote}
Our Father who art in heaven, Hallowed be Thy name. Thy kingdom come. Thy will be done on earth as it is in heaven. Give us this day our daily bread. And forgive us our debts, as we forgive our debtors. And lead us not into temptation, but deliver us from evil. For Thine is the kingdom and the power and the glory forever. Amen.
\end{quote}

Benediction

\begin{quote}
``The grace of the Lord Jesus Christ, and the love of God, and the fellowship of the Holy Spirit, be with you all.''\footnote{2 Corinthians 13:14.} Amen.
\end{quote}

\hypertarget{suggested-form-for-a-graveside-committal-service}{%
\section*{78. Suggested Form for a Graveside Committal Service}\label{suggested-form-for-a-graveside-committal-service}}
\addcontentsline{toc}{section}{78. Suggested Form for a Graveside Committal Service}

\protect\hypertarget{chapter-slug-78-suggested-form-for-a-graveside-committal-service}{\href{}{}}

\protect\hypertarget{78}{\href{}{}}\emph{The following form is from the 1549 Book of Common Prayer, modernized.}

\begin{center}
\emph{At the grave, while the corpse is made ready to be laid into the earth, the minister shall say,}

\end{center}

\begin{quote}
Man that is born of a woman has but a short time to live and is full of misery. He comes up and is cut down, like a flower; he flees like a shadow and never continues in one place.

In the midst of life, we are in death: and of whom may we seek for relief, but of You, O Lord, who for our sins are justly displeased?

Yet, O Lord God most holy, O Lord most mighty, O holy and most merciful Savior, deliver us not into the bitter pains of eternal death. You know, O Lord, the secrets of our hearts: shut not your merciful ears to our prayers; but spare us, Lord most holy, O God most mighty, O holy and merciful Savior, O most worthy Judge eternal, suffer us not at our last hour for any pains of death to fall from You.
\end{quote}

\begin{center}
\emph{Then shall follow this lesson, taken from the fifteenth chapter of the First Epistle to the Corinthians:}

\end{center}

\begin{quote}
But now Christ has been raised from the dead, the first fruits of those who are asleep. For since by a man \emph{came} death, by a man also came the resurrection of the dead. For as in Adam all die, so also in Christ all will be made alive. But each in his own order: Christ the first fruits, after that those who are Christ's at His coming, then \emph{comes} the end, when He hands over the kingdom to the God and Father, when He has abolished all rule and all authority and power. For He must reign until He has put all His enemies under His feet. The last enemy that will be abolished is death. For HE HAS PUT ALL THINGS IN SUBJECTION UNDER HIS FEET. But when He says, ``All things are put in subjection,'' it is evident that He is excepted who put all things in subjection to Him. When all things are subjected to Him, then the Son Himself also will be subjected to the One who subjected all things to Him, so that God may be all in all.

Otherwise, what will those do who are baptized for the dead? If the dead are not raised at all, why then are they baptized for them? Why are we also in danger every hour? I affirm, brethren, by the boasting in you which I have in Christ Jesus our Lord, I die daily. If from human motives I fought with wild beasts at Ephesus, what does it profit me? If the dead are not raised, LET US EAT AND DRINK, FOR TOMORROW WE DIE. Do not be deceived: ``Bad company corrupts good morals.'' Become sober-minded as you ought, and stop sinning; for some have no knowledge of God. I speak \emph{this} to your shame.

But someone will say, ``How are the dead raised? And with what kind of body do they come?'' You fool! That which you sow does not come to life unless it dies; and that which you sow, you do not sow the body which is to be, but a bare grain, perhaps of wheat or of something else. But God gives it a body just as He wished, and to each of the seeds a body of its own. All flesh is not the same flesh, but there is one \emph{flesh} of men, and another flesh of beasts, and another flesh of birds, and another of fish. There are also heavenly bodies and earthly bodies, but the glory of the heavenly is one, and the \emph{glory} of the earthly is another. There is one glory of the sun, and another glory of the moon, and another glory of the stars; for star differs from star in glory.

So also is the resurrection of the dead. It is sown a perishable \emph{body}, it is raised an imperishable \emph{body}; it is sown in dishonor, it is raised in glory; it is sown in weakness, it is raised in power; it is sown a natural body, it is raised a spiritual body. If there is a natural body, there is also a spiritual \emph{body}. So also it is written, ``The first MAN, Adam, BECAME A LIVING SOUL.'' The last Adam \emph{became} a life-giving spirit. However, the spiritual is not first, but the natural; then the spiritual. The first man is from the earth, earthy; the second man is from heaven. As is the earthy, so also are those who are earthy; and as is the heavenly, so also are those who are heavenly. Just as we have borne the image of the earthy, we will also bear the image of the heavenly.

Now I say this, brethren, that flesh and blood cannot inherit the kingdom of God; nor does the perishable inherit the imperishable. Behold, I tell you a mystery; we will not all sleep, but we will all be changed, in a moment, in the twinkling of an eye, at the last trumpet; for the trumpet will sound, and the dead will be raised imperishable, and we will be changed. For this perishable must put on the imperishable, and this mortal must put on immortality. But when this perishable will have put on the imperishable, and this mortal will have put on immortality, then will come about the saying that is written, ``DEATH IS SWALLOWED UP in victory. ``O DEATH, WHERE IS YOUR VICTORY? O DEATH, WHERE IS YOUR STING?'' The sting of death is sin, and the power of sin is the law; but thanks be to God, who gives us the victory through our Lord Jesus Christ.
\end{quote}

\begin{center}
\emph{Then shall follow this lesson, taken from the fourteenth chapter of the Gospel of John:}

\end{center}

\begin{quote}
``Do not let your heart be troubled; believe in God, believe also in Me. In My Father's house are many dwelling places; if it were not so, I would have told you; for I go to prepare a place for you. If I go and prepare a place for you, I will come again and receive you to Myself, that where I am, \emph{there} you may be also. And you know the way where I am going.'' Thomas said to Him, ``Lord, we do not know where You are going, how do we know the way?'' Jesus said to him, ``I am the way, and the truth, and the life; no one comes to the Father but through Me.''
\end{quote}

\begin{center}
\emph{Then the minister shall say,}

\end{center}

\begin{quote}
Forasmuch as it hath pleased Almighty God in His infinite wisdom and mercy to take out of this world the soul of our {[}brother/sister{]} departed, we now commit {[}his/her{]} body to the ground, earth to earth, ashes to ashes, and dust to dust.
\end{quote}

\begin{center}
\emph{Then shall follow this lesson, taken from the fourth chapter of the Apostle Paul's First Epistle to the Thessalonians,}

\end{center}

\begin{quote}
But we do not want you to be uninformed, brethren, about those who are asleep, so that you will not grieve as do the rest who have no hope. For if we believe that Jesus died and rose again, even so God will bring with Him those who have fallen asleep in Jesus. For this we say to you by the word of the Lord, that we who are alive and remain until the coming of the Lord, will not precede those who have fallen asleep. For the Lord Himself will descend from heaven with a shout, with the voice of \emph{the} archangel and with the trumpet of God, and the dead in Christ will rise first. Then we who are alive and remain will be caught up together with them in the clouds to meet the Lord in the air, and so we shall always be with the Lord. Therefore comfort one another with these words.
\end{quote}

\begin{center}
\emph{Then shall be said,}

\end{center}

\begin{quote}
I heard a voice from heaven, saying, ``Write, `Blessed are the dead who die in the Lord from now on!'\,'' ``Yes,'' says the Spirit, ``so that they may rest from their labors,\ldots{}''
\end{quote}

\begin{center}
\emph{Then the minister shall say,}

\end{center}

\begin{quote}
Let us pray.

Almighty God, our heavenly Father, who, in thy perfect wisdom and mercy has ended for your servants departed, the voyage of this troubled life, grant we ask you, that we who are still to continue our course amidst earthly dangers, temptations, and troubles may evermore be protected by Your mercy, and finally come to the haven of eternal salvation, through Jesus Christ our Lord. Amen.
\end{quote}

\begin{center}
\emph{Then the minister shall pronounce a benediction,}

\end{center}

\begin{quote}
The LORD bless you, and keep you;\\
The LORD make His face shine on you,\\
And be gracious to you;\\
The LORD lift up His countenance on you,\\
And give you peace.

Amen.
\end{quote}

\hypertarget{suggested-form-for-infant-dedication}{%
\section*{79. Suggested Form for Infant Dedication}\label{suggested-form-for-infant-dedication}}
\addcontentsline{toc}{section}{79. Suggested Form for Infant Dedication}

\protect\hypertarget{chapter-slug-79-suggested-form-for-infant-dedication}{\href{}{}}

\protect\hypertarget{79}{\href{}{}}\emph{While infant dedication is not a sacrament, those who hold credo-baptist views may yet wish to present their child before the congregation, thanking God for his safe delivery, asking God's blessing upon him, and testifying to God's promises concerning our children. The following is a sample form which may be used where parents would like to publicly dedicate their children to the Lord.}

\textbf{Explanation of Infant Dedication}

\begin{quote}
It has always been a custom among God's people to mark the blessing of the Lord in the birth of a child. Today, this family {[}these families{]} has come to give thanks to God for bringing this child safely into the world, to continue the duty and privilege given our father Adam in naming this child, and to testify to God's promises concerning our children. We have therefore set aside this time to seek God's blessing on this child, and for these parents to proclaim their commitment to raise him within the household of God, the Church, and to tenderly care for, disciple, and teach him the truths of Scripture. Alongside them, our church family pledges to be responsible for this child, to share in their duty to care for him, and to assist them in their teaching and discipleship.

This dedication is an act in which the parents confess that this child is a joyful trust and stewardship from God. They also rejoice with the psalmist, who says,

\begin{quote}
``Behold, children are a gift of the LORD,\\
The fruit of the womb is a reward.\\
Like arrows in the hand of a warrior,\\
So are the children of one's youth.\\
How blessed is the man whose quiver is full of them;\\
They will not be ashamed\\
When they speak with their enemies in the gate.''\footnote{Psalm 127:3-5.}
\end{quote}

These parents also come to claim God's many promises that He will be a God to us and to our children, as the Scriptures testify:

\begin{quote}
``For the promise is for you and your children and for all who are far off, as many as the Lord our God will call to Himself.''\footnote{Acts 2:39.}
\end{quote}
\end{quote}

\textbf{Naming of the Child and Vows of the Parents}

\begin{quote}
\textbf{Pastor}: Father, name your child.\\
\textbf{Father}: {[}Gives full name{]}.\\
\textbf{Pastor}: As you come to dedicate your {[}son/daughter{]} {[}Name{]} to the Lord, I now ask you to reaffirm your faith by answering the following questions.\\
Pastor: Do you confess your faith in God the Father, in Jesus Christ the only begotten Son of God, and in the Holy Spirit, the Lord and Giver of Life?\\
\textbf{Parent(s)}: We do.\\
\textbf{Pastor}: Do you acknowledge that this child is conceived and born in sin, that no one can enter the kingdom of God unless he be born of the Spirit, and that it is only by faith in Christ that this child may be given the blessing of eternal life?\\
\textbf{Parent(s)}: We do.\\
\textbf{Pastor}: Do you promise to teach your child the commandments of God, to live a godly and sober life before {[}him/her{]}, to renounce the devil and all his works, and to show by your care and discipline the love, mercy, discipline, and righteousness of our Lord?\\
\textbf{Parent(s)}: We do.\\
\textbf{Pastor}: Do you acknowledge and claim for yourselves the promise of God that He will be a God to you and your children, such that it is your hope that your child may in God's good time profess faith in our Lord Jesus Christ?\\
\textbf{Parent(s)}: We do.
\end{quote}

\textbf{Charge to the Parents}

\begin{quote}
Beloved, I charge you to teach your child the Word of God, to instruct {[}him/her{]} in the principles of our holy religion as contained in the Scriptures of the Old and New Testaments, to pray with {[}him/her{]} and for {[}him/her{]}, to set an example of piety and godliness before {[}him/her{]}, and to endeavor by every method which God has given us to bring up your child in the nurture and admonition of the Lord.
\end{quote}

\textbf{Vows of the Congregation}

\begin{quote}
\textbf{Pastor}: Will the members of {[}Name of Church{]} please rise?\\
\textbf{Pastor}: Do you, the congregation of {[}Name of Church{]}, promise to tell this child of the Gospel of our Lord, to help {[}him/her{]} learn all of Christ's commandments, and to assist {[}him/her{]} in living a godly and Christian life in the household of God?\\
\textbf{Congregation}: We do.\\
\textbf{Pastor}: Do you promise to help these parents raise this child in the nurture and admonition of the Lord, so that in due time {[}he/she{]} may confess faith in our Lord and Savior, Jesus Christ?\\
\textbf{Congregation}: We do.\\
{[}Please be seated.{]}
\end{quote}

\textbf{Prayer of Blessing}

\begin{quote}
Let us pray.

Almighty and everlasting God, who of Your infinite mercy and goodness has promised that You will not be only our God, but also the God and Father of our children: we thank you for the safe delivery of this child through the pains and difficulties of labor. And we come to you now, humbly asking that you take this child under your fatherly care and protection, and guide {[}him/her{]} into faith. Allow {[}him/her{]} to grow in wisdom and in stature, and in favor with God and man. Abundantly enrich (him/her) with your heavenly grace: bring {[}him/her{]} safely through the perils of childhood, deliver {[}him/her{]} from the temptations of youth, lead {[}him/her{]} to a sincere confession of faith in Jesus Christ, and cause {[}him/her{]} to persevere in that confession to {[}his/her{]} life's end.

Father, please guide these parents with your counsel as they train and teach their child; and help them to lead their household into an ever-increasing knowledge of Christ. We commend to Your care the children and families of this congregation. Help us in our homes to honor You, and to lovingly serve one another. To Your name be all blessing and glory through Jesus Christ our Lord. Amen.
\end{quote}

\hypertarget{the-creeds}{%
\section*{80. The Creeds}\label{the-creeds}}
\addcontentsline{toc}{section}{80. The Creeds}

\protect\hypertarget{chapter-slug-80-the-creeds}{\href{}{}}
\protect\hypertarget{80}{\href{}{}}

\hypertarget{the-apostles-creed}{%
\subsection*{\texorpdfstring{1. \textbf{The Apostles' Creed}}{1. The Apostles' Creed}}\label{the-apostles-creed}}
\addcontentsline{toc}{subsection}{1. \textbf{The Apostles' Creed}}

\begin{quote}
I believe in God, the Father Almighty,\\
Maker of heaven and earth,\\
and in Jesus Christ, His only Son, our Lord:

Who was conceived by the Holy Ghost,\footnote{Or, ``Holy Spirit.''}\\
born of the virgin Mary,\\
suffered under Pontius Pilate,\\
was crucified, dead, and buried;

He descended into hell;

The third day He arose again from the dead;

He ascended into heaven,\\
and sitteth\footnote{Or, ``sits.''} on the right hand of God the Father Almighty;\\
from thence\footnote{Or, ``there.''} He shall come to judge the quick\footnote{Or, ``living.''} and the dead.

I believe in the Holy Ghost,\footnote{Or, ``Holy Spirit.''}\\
the holy catholic\footnote{Or, ``universal.''} church,\\
the communion of saints,\\
the forgiveness of sins,\\
the resurrection of the body,\\
and the life everlasting.

Amen.
\end{quote}

\hypertarget{the-nicene-creed}{%
\subsection*{\texorpdfstring{2. \textbf{The Nicene Creed}}{2. The Nicene Creed}}\label{the-nicene-creed}}
\addcontentsline{toc}{subsection}{2. \textbf{The Nicene Creed}}

\begin{quote}
We believe in one God, the Father Almighty,\\
Maker of heaven and earth,\\
and of all things visible and invisible.

And in one Lord Jesus Christ, the only-begotten Son of God,\\
begotten of the Father before all worlds;\\
God of God, Light of Light,\\
very God of very God;\\
begotten, not made,\\
being of one substance with the Father,\\
by whom all things were made.\\
Who, for us men and for our salvation,\\
came down from heaven,\\
and was incarnate by the Holy Spirit of the virgin Mary,\\
and was made man;\\
and was crucified also for us under Pontius Pilate;\\
He suffered and was buried;\\
and the third day He rose again, according to the Scriptures;\\
and ascended into heaven, and sits on the right hand of the Father;\\
and He shall come again, with glory, to judge the quick\footnote{Or, ``living.''} and the dead;\\
whose kingdom shall have no end.

And we believe in the Holy Spirit,\\
the Lord and Giver of Life;\\
who proceeds from the Father and the Son;\\
who with the Father and the Son together is worshiped and glorified;\\
who spoke by the prophets.

And we believe (in) one holy catholic\footnote{Or, ``universal.''} and apostolic Church.\\
We acknowledge one baptism for the remission of sins;\\
and we look for the resurrection of the dead,\\
and the life of the world to come.

Amen.
\end{quote}

\hypertarget{the-athanasian-creed}{%
\subsection*{\texorpdfstring{3. \textbf{The Athanasian Creed}}{3. The Athanasian Creed}}\label{the-athanasian-creed}}
\addcontentsline{toc}{subsection}{3. \textbf{The Athanasian Creed}}

\begin{quote}
{[}1{]} Whoever desires to be saved should above all hold to the catholic\footnote{Or, ``universal.''} faith.~{[}2{]} Anyone who does not keep it whole and unbroken will doubtless perish eternally.

{[}3{]} Now this is the catholic faith: that we worship one God in Trinity and~the Trinity in unity, {[}4{]} neither confounding their persons nor dividing the~essence. {[}5{]} For the person of the Father is a distinct person, the person of the Son is another, and that of the Holy Spirit still another. {[}6{]} But the divinity of the Father, Son, and Holy Spirit is one, the glory equal, the majesty coeternal. {[}7{]} Such as the Father is, such is the Son and such is the Holy Spirit. {[}8{]} The Father is uncreated, the Son is uncreated, the Holy Spirit is uncreated. {[}9{]} The Father is immeasurable, the Son is immeasurable, the Holy Spirit is immeasurable. {[}10{]} The Father is eternal, the Son is eternal, the Holy Spirit is eternal. {[}11{]} And yet there are not three eternal beings; there is but one eternal being. {[}12{]} So too there are not three uncreated or immeasurable beings; there is but one uncreated and immeasurable being. {[}13{]} Similarly the Father is almighty, the Son is almighty, the Holy Spirit is almighty. {[}14{]} Yet there are not three almighty beings; there is but one almighty being. {[}15{]} Thus, the Father is God, the Son is God, the Holy Spirit is God. {[}16{]} Yet there are not three gods; there is but one God. {[}17{]} Thus, the Father is Lord, the Son is Lord, the Holy Spirit is Lord. {[}18{]} Yet there are not three lords; there is but one Lord. {[}19{]} Just as Christian truth compels us to confess each person individually as both God and Lord, {[}20{]} so catholic religion forbids us to say that there are three gods or lords. {[}21{]} The Father was neither made nor created nor begotten from anyone. {[}22{]} The Son was neither made nor created; He was begotten from the Father alone. {[}23{]} The Holy Spirit was neither made nor created nor begotten; He proceeds from the Father and the Son. {[}24{]} Accordingly, there is one Father, not three fathers; there is one Son, not three sons; there is one Holy Spirit, not three holy spirits. {[}25{]} None in this Trinity is before or after, none is greater or smaller; {[}26{]} in their entirety the three persons are coeternal and coequal with each other. {[}27{]} So in everything, as was said earlier, the unity in Trinity, and the Trinity in unity, is to be worshiped. {[}28{]} Anyone then who desires to be saved should think thus about the Trinity.

{[}29{]} But it is necessary for eternal salvation that one also believe in the incarnation of our Lord Jesus Christ faithfully. {[}30{]} Now this is the true faith: that we believe and confess that our Lord Jesus Christ, God's Son, is both God and man, equally. {[}31{]} He is God from the essence of the Father, begotten before time; and He is man from the essence of His mother, born in time; {[}32{]} completely God, completely man, with a rational soul and human flesh; {[}33{]} equal to the Father as regards divinity, less than the Father as regards humanity. {[}34{]} Although He is God and man, yet Christ is not two, but one. {[}35{]} He is one, however, not by His divinity being turned into flesh, but by God's taking humanity to Himself. {[}36{]} He is one, certainly not by the blending of His essence, but by the unity of His person. {[}37{]} For just as one man is both rational soul and flesh, so too the one Christ is both God and man. {[}38{]} He suffered for our salvation; He descended to hell; He arose from the dead on the third day; {[}39{]} He ascended to heaven; He is seated at the Father's right hand; {[}40{]} from there He will come to judge the living and the dead. {[}41{]} At His coming all people will arise bodily {[}42{]} and give an accounting of their own deeds. {[}43{]} Those who have done good will enter eternal life, and those who have done evil will enter eternal fire.

{[}44{]} This is the catholic faith: that one cannot be saved without believing it firmly and faithfully.
\end{quote}

\hypertarget{the-chalcedonian-creed}{%
\subsection*{\texorpdfstring{4. \textbf{The Chalcedonian Creed}}{4. The Chalcedonian Creed}}\label{the-chalcedonian-creed}}
\addcontentsline{toc}{subsection}{4. \textbf{The Chalcedonian Creed}}

\begin{quote}
We, then, following the holy Fathers, all with one consent, teach men to confess one and the same Son, our Lord Jesus Christ, the same perfect in Godhead and also perfect in manhood; truly God and truly man, of a reasonable\footnote{Or, ``rational.''} soul and body; consubstantial with the Father according to the Godhead, and consubstantial with us according to the Manhood; in all things like unto us, without sin; begotten before all ages of the Father according to the Godhead, and in these latter days, for us and for our salvation, born of the Virgin Mary, the God-bearer,\footnote{τῆς θεοτόκου, sometimes rendered ``Mother of God.''} according to the Manhood; one and the same Christ, Son, Lord, Only-begotten, to be acknowledged in two natures, inconfusedly, unchangeably, indivisibly, inseparably; the distinction of natures being by no means taken away by the union, but rather the property of each nature being preserved, and concurring in one Person and one Subsistence, not parted or divided into two persons, but one and the same Son, and only begotten, God the Word, the Lord Jesus Christ, as the prophets from the beginning have declared concerning Him, and the Lord Jesus Christ Himself has taught us, and the Creed of the holy Fathers has handed down to us.
\end{quote}

\hypertarget{appendix-1-presbytery-forms}{%
\chapter*{Appendix 1: Presbytery Forms}\label{appendix-1-presbytery-forms}}
\addcontentsline{toc}{chapter}{Appendix 1: Presbytery Forms}

\hypertarget{form-of-ministerial-obligation}{%
\section*{Form of Ministerial Obligation}\label{form-of-ministerial-obligation}}
\addcontentsline{toc}{section}{Form of Ministerial Obligation}

\protect\hypertarget{chapter-form-of-ministerial-obligation}{\href{}{}}

\protect\hypertarget{form-of-ministerial-obligation}{\href{}{}}Believing the Scriptures of the Old and New Testaments, as originally given, to be the infallible Word of God, which is the only infallible rule of faith and practice; and

Sincerely receiving and adopting the Confession of Faith and the Catechisms of this Church, as containing the system of doctrine taught in the Holy Scriptures; and further promising that if at any time I find myself out of accord with any of the fundamentals of this system of doctrine, I will on my own initiative, make known to my Presbytery the change which has taken place in my views since the assumption of my ordination vows; and

Approving of the form of government and discipline of Evangel Presbytery as being in conformity with the general principles of biblical polity; and

Promising subjection to my brethren in the Lord; and

Having been induced, as far as I know my own heart, to seek the office of the holy ministry from love to God and a sincere desire to promote His glory in the Gospel of His Son; and

Promising to be zealous and faithful in maintaining the truths of the Gospel and the purity and peace of the Church, whatever persecution or opposition may arise unto me on that account; and

Engaging to be faithful and diligent in the exercise of all my duties as a Christian and a minister of the Gospel, whether personal or relational, private or public; and to endeavor by the grace of God to adorn the profession of the Gospel in my manner of life, and to walk with exemplary piety before the flock of which God shall make me overseer; therefore

I, \_\_\_\_\_\_\_\_\_\_\_\_\_\_\_\_\_\_\_\_\_\_\_\_\_, do sincerely receive and subscribe to the above obligation as a just and true exhibition of my faith and principles, and do resolve and promise to exercise my ministry in conformity therewith.

Signed:

Date:

Work:

Date ordained:

By whom:

On reverse side, feel free to indicate any approved differences you may have with the Westminster Standards, or differences you would now like to register.

\hypertarget{appendix-2-original-texts-of-liturgical-forms}{%
\chapter*{Appendix 2: Original Texts of Liturgical Forms}\label{appendix-2-original-texts-of-liturgical-forms}}
\addcontentsline{toc}{chapter}{Appendix 2: Original Texts of Liturgical Forms}

\hypertarget{calvins-service-of-the-lords-supper-1542-1566}{%
\section*{1. Calvin's Service of the Lord's Supper (1542, 1566)}\label{calvins-service-of-the-lords-supper-1542-1566}}
\addcontentsline{toc}{section}{1. Calvin's Service of the Lord's Supper (1542, 1566)}

\protect\hypertarget{chapter-slug-1-calvins-service-of-the-lords-supper-1542-1566}{\href{}{}}

\protect\hypertarget{appendix2.1}{\href{}{}}

\emph{We must note that the Sunday before the Supper is celebrated, we announce it to the people: first, in order that each may be ready and prepared to receive it worthily and with appropriate reverence; second, so that we would not present children unless they are well instructed and have made a profession of their faith in the church; third, so that if there are strangers who are still untaught and ignorant, they would come to be instructed privately. The day when we celebrate it, the minister touches upon it at the end of the sermon, or, if necessary, makes it the entire sermon, an exposition to the people of what our Lord intends to say and signify by this mystery and how we ought to receive it.}

\emph{Then, having offered the prayers and recited the Confession of Faith to testify in the name of the people that all wish to live and die in the Christian doctrine and religion, he says aloud:}

\begin{quote}
Let us listen to how Jesus Christ instituted his Holy Supper for us, as Saint Paul relates in chapter eleven of the First Epistle to the Corinthians:

\begin{quote}
I received from the Lord, he says, what I have delivered to you; that the Lord Jesus on the night when he was given up, took bread, and having given thanks, broke it, and said, ``Take, eat. This is my body, which is broken for you. Do this in remembrance of me.'' Likewise, after suppers, he took the cup, saying, ``This cup is the new testament in my blood. Do this, each and every time that you drink it, in memory of me.'' That is, whenever you eat this bread and drink of this cup, you announce the Lord's death until he comes. Therefore, whoever eats this bread or drinks of this cup unworthily will be guilty of the body and blood of the Lord. But let each one examine themselves, and so let them eat this bread and drink of this cup. For whoever eats or drinks it unworthily partakes of his or her condemnation, not discerning the body of Christ. {[}1 Cor. 11:23-29{]}
\end{quote}

We have heard, my brother, how our Lord administered his Supper among his disciples, and in this he shows us that strangers, that is, those not of the company of the faithful should not be admitted. Following this rule, therefore, in the name and by the authority of our Lord Jesus Christ, I excommunicate all idolaters, blasphemers, despisers of God, heretics, and all who form separate parties to break the unity of the church, all perjurers, all those who rebel against their father and mother and against their superiors, all fomenters of sedition or mutiny, quarrelers, fighters, adulterers, debauchees, thieves, hoarders of wealth, plunderers, drunkards, gluttons, and all those who lead a scandalous life; declaring to those that they are to abstain from this holy table lest they pollute and contaminate this sacred food, which our Lord Jesus Christ gives only to his servants and faithful ones.

Therefore, according to the exhortation of Saint Paul, let each one test and examine his conscience, to know whether he truly repents of his faults and is sorry for them desiring from now on to live in holiness and in conformity with God; and above all, whether he trusts in the mercy of God and seeks his salvation wholly from Jesus Christ; and whether renouncing all hostility and malice, he has the good intention and the courage to live in harmony and brotherly love with his neighbors.

If we have such a testimony in our hearts before God, let us not doubt in the least that he acknowledges us to be his children and that the Lord Jesus is speaking to us, bringing us to his table and offering us this Holy Sacrament, which he delivered to his disciples.

And since we are conscious of much frailty and misery in ourselves, as well as not having a perfect faith, but that we are prone rather to unbelief and distrust, so that we are not entirely dedicated to serving God and with such a zeal as we ought, but we have instead to battle daily against the lusts of our flesh; nevertheless, since our Lord has granted us this grace of having his gospel engraved on our heart, so that we might resist all unbelief, and he has given us the desire and longing to renounce our own desires to pursue his righteousness and holy commandments; let us all be assured that the vices and imperfections that are in us will not prevent him from receiving us, nor from making us worthy to share in this spiritual table. For we do not come insisting that we are perfect or righteous in ourselves, but rather, seeking our life in Jesus Christ, we confess that we are dead. Let us understand, therefore, that this Sacrament is a medicine for poor, spiritual sick people and that the only worthiness that our Lord requires of us is to know ourselves well enough to be displeased with our vices and to find all our pleasure, joy, and contentment in him alone.

So let us first believe in these promises, which Jesus Christ, who is the infallible truth, spoke with his mouth, namely, that he truly wishes to make us partakers of his body and blood; that we might possess him fully, so that he might live in us and we in him. And since we see only bread and wine, yet we do not doubt that he accomplishes spiritually in our souls all that he demonstrates to us outwardly through these visible signs, namely, that he is the heavenly bread that feeds and nourishes us for eternal life. So let us be grateful for the infinite goodness of our Savior, who spreads out all his riches and goods on this table to distribute them to us. For by giving himself to us, he testifies to us that all that he has is ours.

Therefore, let us receive this Sacrament as a seal that the power of his death and passion is imputed to us for righteousness, just as though we had suffered it ourselves. Let us therefore not be so wicked as to pull back from where Jesus Christ so gently invites us through his Word. But considering the worth of this precious gift which he has given us, let us present ourselves to him with ardent zeal, so that he would make us able to receive it.

For this purpose, let us lift up our hearts and our spirits to where Jesus Christ is the glory of his Father, and from where we await him in our redemption. And let us not waste time with these earthly and corruptible elements, which we see with our eyes and touch with our hands, seeking him as though he were enclosed inside the bread or the wine. So our souls will be inclined to be nourished and revived by his substance, when they are thus lifted above all earthly things to reach heaven and enter the kingdom of God where he dwells. Let us therefore be satisfied with having the bread and the wine as signs and proofs, seeking the truth spiritually, which is how the Word of God promises that we will find it.
\end{quote}

\emph{That done, the ministers distribute the bread and the cup to the people, having warned that they should approach with reverence and good order. Meanwhile, we sing some psalms or read a text from Scripture, which is appropriate for that which is signified by the Sacrament. At the end, we offer thanksgiving, as it has been said.}\footnote{From Mark Earngey and Jonathan Gibson, \emph{Reformation Worship} (New Growth Press, 2018), 324--29.}

\hypertarget{the-form-of-solemnization-of-matrimony-1549-book-of-common-prayer}{%
\section*{2. The Form of Solemnization of Matrimony (1549 Book of Common Prayer)}\label{the-form-of-solemnization-of-matrimony-1549-book-of-common-prayer}}
\addcontentsline{toc}{section}{2. The Form of Solemnization of Matrimony (1549 Book of Common Prayer)}

\protect\hypertarget{chapter-slug-2-the-form-of-solemnization-of-matrimony-1549-book-of-common-prayer}{\href{}{}}

\protect\hypertarget{appendix2.2}{\href{}{}}\emph{First, the banns must be asked three several Sundays or holy days in the service time, the people being present, after the accustomed manner. And if the persons that would be married dwell in divers parishes, the banns must be asked in both parishes, and the Curate of the one parish shall not solemnize matrimony betwixt them, without a certificate of the banns being thrice asked from the Curate of the other parish. At the day appointed for Solemnization of Matrimony, the persons to be married shall come into the body of the church, with their friends and neighbors. And there the priest shall thus say,}

\begin{quote}
Dearly beloved friends, we are gathered together here in the sight of God, and in the face of his congregation, to join together this man and this woman in holy matrimony, which is an honorable estate instituted of God in paradise, in the time of man's innocency, signifying unto us the mystical union that is betwixt Christ and his Church: which holy estate, Christ adorned and beautified with his presence, and first miracle that he wrought in Cana of Galilee, and is commended of Saint Paul to be honorable among all men; and therefore is not to be enterprised, nor taken in hand unadvisedly, lightly, or wantonly, to satisfy men's carnal lusts and appetites, like brute beasts that have no understanding: but reverently, discreetly, advisedly, soberly, and in the fear of God. Duly considering the causes for the which matrimony was ordained. One cause was the procreation of children, to be brought up in the fear and nurture of the Lord, and praise of God. Secondly it was ordained for a remedy against sin, and to avoid fornication, that such persons as be married, might live chastely in matrimony, and keep themselves undefiled members of Christ's body. Thirdly, for the mutual society, help, and comfort, that the one ought to have of the other, both in prosperity and adversity. Into the which holy estate these two persons present: come now to be joined. Therefore if any man can shew any just cause why they may not lawfully be joined so together: Let him now speak, or else hereafter for ever hold his peace.
\end{quote}

\emph{And also speaking to the persons that shall be married, he shall say,}

\begin{quote}
I require and charge you (as you will answer at the dreadful day of judgement, when the secrets of all hearts shall be disclosed) that if either of you do know any impediment, why ye may not be lawfully joined together in matrimony, that ye confess it. For be ye well assured, that so many as be coupled together otherwise then God's word doth allow: are not joined of God, neither is their matrimony lawful.
\end{quote}

\emph{At which day of marriage if any man do allege any impediment why they may not be coupled together in matrimony; And will be bound, and sureties with him, to the parties, or else put in a caution to the full value of such charges as the persons to be married do sustain to prove his allegation: then the Solemnization must be deferred, unto such time as the truth be tried. If no impediment be alleged, then shall the Curate say unto the man,}

\begin{quote}
N.Wilt thou have this woman to thy wedded wife, to live together after God's ordinance in the holy estate of matrimony? Wilt thou love her, comfort her, honor, and keep her in sickness and in health? And forsaking all other keep thee only to her, so long as you both shall live?
\end{quote}

\emph{The man shall answer,}

\begin{quote}
I will.
\end{quote}

\emph{Then shall the priest say to the woman,}

\begin{quote}
N.Wilt thou have this man to thy wedded husband, to live together after God's ordinance, in the holy estate of matrimony? Wilt thou obey him, and serve him, love, honor, and keep him in sickness and in health? And forsaking all other keep thee only to him, so long as you both shall live?
\end{quote}

\emph{The woman shall answer,}

\begin{quote}
I will.
\end{quote}

\emph{Then shall the Minister say,}

\begin{quote}
Who giveth this woman to be married to this man?
\end{quote}

\emph{And the minister receiving the woman at her father or friend's hands: shall cause the man to take the woman by the right hand, and so either to give their troth to other: The man first saying,}

\begin{quote}
I N. take thee N. to my wedded wife, to have and to hold from this day forward, for better, for worse, for richer, for poorer, in sickness, and in health, to love and to cherish, till death us depart: according to God's holy ordinance: And thereto I plight thee my troth.
\end{quote}

\emph{Then shall they loose their hands, and the woman taking again the man by the right hand shall say,}

\begin{quote}
I N. take thee N. to my wedded husband, to have and to hold from this day forward, for better, for worse, for richer, for poorer, in sickness, and in health, to love, cherish, and to obey, till death us depart: according to God's holy ordinance: And thereto I give thee my troth.
\end{quote}

\emph{Then shall they again loose their hands, and the man shall give unto the woman a ring, and other tokens of spousage, as gold or silver, laying the same upon the book: And the Priest taking the ring shall deliver it unto the man: to put it upon the fourth finger of the woman's left hand. And the man taught by the priest, shall say,}

\begin{quote}
With this ring I thee wed: This gold and silver I thee give: with my body I thee worship: and withal my worldly Goods I thee endow. In the name of the Father, and of the Son, and of the Holy Ghost. Amen.
\end{quote}

\emph{Then the man leaving the ring upon the fourth finger of the woman's left hand, the minister shall say,}

\begin{quote}
Let us pray.

O Eternal God creator and preserver of all mankind, giver of all spiritual grace, the author of everlasting life: Send thy blessing upon these thy servants, this man, and this woman, whom we bless in thy name, that as Isaac and Rebecca (after bracelets and jewels of gold given of the one to the other for tokens of their matrimony) lived faithfully together; So these persons may surely perform and keep the vow and covenant betwixt them made, whereof this ring given, and received, is a token and pledge. And may ever remain in perfect love and peace together; And live according to thy laws; through Jesus Christ our Lord. Amen.
\end{quote}

\emph{Then shall the priest join their right hands together, and say,}

\begin{quote}
Those whom God hath joined together: let no man put asunder.
\end{quote}

\emph{Then shall the minister speak unto the people,}

\begin{quote}
Forasmuch as N. and N. have consented together in holy wedlock, and have witnessed the same here before God and this company; And thereto have given and pledged their troth either to other, and have declared the same by giving and receiving gold and silver, and by joining of hands: I pronounce that they be man and wife together. In the name of the Father, of the Son, and of the Holy Ghost. Amen.
\end{quote}

\emph{And the minister shall add this blessing,}

\begin{quote}
God the Father bless you. God the Son keep you: God the Holy Ghost lighten your understanding: The Lord mercifully with his favor look upon you, and so fill you with all spiritual benediction, and grace, that you may have remission of your sins in this life, and in the world to come life everlasting. Amen.
\end{quote}

\emph{Then shall they go into the quire, and the ministers or clerks shall say or sing, this psalm following,}

\emph{Beati omnes. cxxviii.}

Blessed are all they that fear the Lord, and walk in his ways.\\
For thou shalt eat the labour of thy hands. O well is thee, and happy shalt thou be.\\
Thy wife shall be as the fruitful vine, upon the walls of thy house.\\
Thy children like the olive branches round about thy table.\\
Lo, thus shall the man be blessed, that feareth the Lord.\\
The Lord from out of Zion, shall so bless thee: that thou shalt see Jerusalem in prosperity, all thy life long.\\
Yea that thou shalt see thy children's children: and peace upon Israel.\\
Glory to the Father, etc.\\
As it was in the beginning, etc.

\emph{Or else this psalm following,}

\emph{Deus misereatur nostri. Psalm lxvii.}

God be merciful unto us, and bless us, and shew us the light of his countenance: and be merciful unto us.\\
That thy way may be known upon the earth, thy saving health among all nations.\\
Let the people praise thee (O God) yea let all people praise thee.\\
O let the nations rejoice and be glad, for thou shalt judge the folk righteously, and govern the nations upon the earth.\\
Let the people praise thee (O God) let all people praise thee. Then shall the earth bring forth her increase: and God, even our own God, shall give us his blessing.\\
God shall bless us, and all the ends of the world shall fear him.\\
Glory to the Father, etc.\\
As it was in the beginning, etc.

\emph{The psalm ended, and the man and woman kneeling afore the altar: the priest standing at the altar, and turning his face toward them, shall say,}

\begin{quote}
Lord have mercy upon us.\\
\emph{Answer}. Christ have mercy upon us.\\
\emph{Minister}. Lord have mercy upon us.\\
Our Father which art in heaven, etc.\\
And lead us not into temptation.\\
\emph{Answer}. But deliver us from evil. Amen.\\
\emph{Minister}. O Lord save thy servant, and thy handmaiden.\\
Answer. Which put their trust in thee.\\
\emph{Minister}. O Lord send them help from thy holy place.\\
\emph{Answer}. And evermore defend them.\\
\emph{Minister}. Be unto them a tower of strength.\\
\emph{Answer}. From the face of their enemy.\\
\emph{Minister}. O Lord hear my prayer.\\
\emph{Answer}. And let my cry come unto thee.\\
\emph{The Minster}. Let us pray.

O God of Abraham, God of Isaac, God of Jacob, bless these thy servants, and sow the seed of eternal life in their minds, that whatsoever in thy holy word they shall profitably learn: they may indeed fulfill the same. Look, O Lord, mercifully upon them from heaven, and bless them: And as thou didst send thy Angel Raphael to Tobit, and Sara, the daughter of Raguel, to their great comfort; so vouchsafe to send thy blessing upon these thy servants, that they obeying thy will, and alway being in safety under thy protection: may abide in thy love unto their lives' end: through Jesus Christ our Lord. Amen.
\end{quote}

\emph{This prayer following shall be omitted where the woman is past childbirth,}

\begin{quote}
O merciful Lord, and heavenly Father, by whose gracious gift mankind is increased: We beseech thee assist with thy blessing these two persons, that they may both be fruitful in procreation of children; and also live together so long in godly love and honesty, that they may see their children's children, unto the third and fourth generation, unto thy praise and honour: through Jesus Christ our Lord. Amen.

O God which by thy mighty power hast made all things of naught, which also after other things set in order didst appoint that out of man (created after thine own image and similitude) woman should take her beginning: and, knitting them together, didst teach, that it should never be lawful to put asunder those, whom thou by matrimony hadst made one: O God, which hast consecrated the state of matrimony to such an excellent mystery, that in it is signified and represented the spiritual marriage and unity betwixt Christ and his church: Look mercifully upon these thy servants, that both this man may love his wife, according to thy word (as Christ did love his spouse the church, who gave himself for it, loving and cherishing it even as his own flesh); And also that this woman may be loving and amiable to her husband as Rachel, wise as Rebecca, faithful and obedient as Sara; And in all quietness, sobriety, and peace, be a follower of holy and godly matrons. O Lord, bless them both, and grant them to inherit thy everlasting kingdom, through Jesus Christ our Lord. Amen.
\end{quote}

\emph{Then shall the priest bless the man and the woman, saying,}

\begin{quote}
Almighty God, which at the beginning did create our first parents Adam and Eve, and did sanctify and join them together in marriage: Pour upon you the riches of his grace, sanctify and bless you, that ye may please him both in body and soul; and live together in holy love unto your lives' end. Amen.
\end{quote}

\emph{Then shall be said after the gospel a sermon, wherein ordinarily (so oft as there is any marriage) the office of man and wife shall be declared according to Holy Scripture. Or if there be no sermon, the minister shall read this that followeth,}

\begin{quote}
All ye which be married, or which intend to take the holy estate of matrimony upon you: hear what Holy Scripture doth say, as touching the duty of husbands toward their wives, and wives toward their husbands.

Saint Paul (in his epistle to the Ephesians the fifth chapter) doth give this commandment to all married men:

Ye husbands love your wives, even as Christ loved the church, and hath given himself for it, to sanctify it, purging it in the fountain of water, through the word, that he might make it unto himself, a glorious congregation, not having spot or wrinkle, or any such thing; but that it should be holy and blameless. So men are bound to love their own wives as their own bodies: he that loveth his own wife, loveth himself. For never did any man hate his own flesh, but nourisheth and cherisheth it, even as the Lord doth the congregation, for we are members of his body, of his flesh, and of his bones. For this cause shall a man leave father and mother, and shall be joined unto his wife, and they two shall be one flesh. This mystery is great, but I speak of Christ and of the congregation. Nevertheless let every one of you so love his own wife, even as himself.

Likewise the same Saint Paul (writing to the Colossians) speaketh thus to all men that be married: Ye men, love your wives and be not bitter unto them. Coloss. iii.

Hear also what saint Peter the apostle of Christ, (which was himself a married man,) saith unto all men that are married. Ye husbands, dwell with your wives according to knowledge: Giving honor unto the wife, as unto the weaker vessel, and as heirs together of the grace of life, so that your prayers be not hindered. 1 Pet. iii.

Hitherto ye have heard the duty of the husband toward the wife.

Now likewise, ye wives, hear and learn your duty toward your husbands, even as it is plainly set forth in Holy Scripture.

Saint Paul (in the forenamed epistle to the Ephesians) teacheth you thus: Ye women submit yourselves unto your own husbands as unto the Lord: for the husband is the wife's head, even as Christ is the head of the church: And he also is the saviour of the whole body. Therefore as the Church, or congregation, is subject unto Christ: So likewise let the wives also be in subjection unto their own husbands in all things. Ephes. v. And again he saith: Let the wife reverence her husband. And (in his epistle to the Colossians) Saint Paul giveth you this short lesson. Ye wives, submit yourselves unto your own husbands, as it is convenient in the Lord. Coloss. iii.

Saint Peter also doth instruct you very godly, thus saying, Let wives be subject to their own husbands, so that if any obey not the word, they may be won without the word, by the conversation of the wives; While they behold your chaste conversation, coupled with fear, whose apparel let it not be outward, with braided hair, and trimming about with gold, either in putting on of gorgeous apparel: But let the hid man which is in the heart, be without all corruption, so that the spirit be mild and quiet, which is a precious thing in the sight of God. For after this manner (in the old time) did the holy women, which trusted in God, apparel themselves, being subject to their own husbands: as Sara obeyed Abraham calling him lord, whose daughters ye are made, doing well, and being not dismayed with any fear. 1 Pet. iii.
\end{quote}

\emph{The new married persons (the same day of their marriage) must receive the holy communion.}\footnote{Original text from the 1549 Book of Common Prayer, with modernized spelling.}

\protect\hypertarget{chapter-slug-3-the-order-for-the-burial-of-the-dead-1549-book-of-common-prayer}{\href{}{}}

\hypertarget{the-order-for-the-burial-of-the-dead-1549-book-of-common-prayer}{%
\section*{3. The Order for the Burial of the Dead (1549 Book of Common Prayer)}\label{the-order-for-the-burial-of-the-dead-1549-book-of-common-prayer}}
\addcontentsline{toc}{section}{3. The Order for the Burial of the Dead (1549 Book of Common Prayer)}

\protect\hypertarget{appendix2.3}{\href{}{}}\emph{The priest meeting the corpse at the church style, shalt say: Or else the priests and clerks shalt sing, and so go either into the church, or towards the grave,}

\begin{quote}
I am the resurrection and the life (saith the Lord): he that believeth in me, yea though he were dead, yet shall he live. And whosoever liveth and believeth in me: shall not die forever. \emph{John} xi.

I know that my redeemer liveth, and that I shall rise out of the earth in the last day, and shall be covered again with my skin and shall see God in my flesh: yea and I myself shall behold him, not with other but with these same eyes. Job xix.

We brought nothing into this world, neither may we carry anything out of this world. The Lord giveth, and the Lord taketh away. Even as it pleaseth the Lord, so cometh things to pass: blessed be the name of the Lord.
\end{quote}

\emph{When they come at the grave, while the corpse is made ready to be laid into the earth, the priest shall say, or else the priest and clerks shall sing,}

\begin{quote}
Man that is born of a woman hath but a short time to live, and is full of misery: he cometh up and is cut down like a flower; he flieth as it were a shadow, and never continueth in one stay. 1 \emph{Tim}. vi. \emph{Job} i.

In the midst of life we be in death, of whom may we seek for succor but of thee, O Lord, which for our sins justly art moved? Yet O Lord God most holy, O Lord most mighty, O holy and most merciful Savior, deliver us not into the bitter pains of eternal death. Thou knowest, Lord, the secrets of our hearts: shut not up thy merciful eyes to our prayers: But spare us, Lord most holy, O God most mighty, O holy and merciful Savior, thou most worthy judge eternal, suffer us not at our last hour for any pains of death to fall from thee.
\end{quote}

\emph{Then the priest casting earth upon the corpse, shall say,}

\begin{quote}
I commend thy soul to God the Father Almighty, and thy body to the ground, earth to earth, ashes to ashes, dust to dust, in sure and certain hope of resurrection to eternal life, through our Lord Jesus Christ, who shall change our vile body, that it may be like to his glorious body, according to the mighty working whereby he is able to subdue all things to himself.
\end{quote}

\emph{Then shall be said or sung,}

\begin{quote}
I heard a voice from heaven saying, unto me: Write, blessed are the dead which die in the Lord. Even so saith the spirit, that they rest from their labors.

Let us pray.

We commend into thy hands of mercy (most merciful Father) the soul of this our brother departed, \emph{N}. And his body we commit to the earth, beseeching thine infinite goodness, to give us grace to live in thy fear and love, and to die in thy favor: that when the judgement shall come which thou hast committed to thy well-beloved Son, both this our brother, and we, may be found acceptable in thy sight, and receive that blessing, which thy well-beloved Son shall then pronounce to all that love and fear thee, saying: Come ye blessed children of my Father: Receive the kingdom prepared for you before the beginning of the world. Grant this, merciful Father, for the honor of Jesus Christ our only Savior, Mediator, and Advocate. Amen.
\end{quote}

\emph{This prayer shalt also be added,}

\begin{quote}
Almighty God, we give thee hearty thanks for this thy servant, whom thou hast delivered from the miseries of this wretched world, from the body of death and all temptation. And, as we trust, hast brought his soul which he committed into thy holy hands, into sure consolation and rest: Grant, we beseech thee, that at the day of judgement his soul and all the souls of thy elect, departed out of this life, may with us and we with them, fully receive thy promises, and be made perfect altogether through the glorious resurrection of thy Son Jesus Christ our Lord.
\end{quote}

\emph{These psalms with other suffrages following are to be said in the church either before or after the burial of the corpse,}

\emph{Dilexi, quoniam}. Psalm cxvi.

I am well pleased that the Lord hath heard the voice of my prayer.\\
That he hath inclined his ear unto me, therefore will I call upon him as long as I live.\\
The snares of death compassed me round about, and the pains of hell gat hold upon me: I shall find trouble and heaviness, and I shall call upon the name of the Lord, (O Lord,) I beseech thee deliver my soul.\\
Gracious is the Lord, and righteous, yea, our God is merciful.\\
The lord preserveth the simple: I was in misery and he helped me.\\
Turn again then unto thy rest, O my soul, for the Lord hath rewarded thee.\\
And why? thou hast delivered my soul from death, mine eyes from tears, and my feet from falling.\\
I will walk before the Lord in the land of the living.\\
I believed, and therefore will I speak: but I was sore troubled. I said in my haste: all men are liars.\\
What reward shall I give unto the Lord, for all the benefits that he hath done unto me?\\
I will receive the cup of salvation, and call upon the name of the Lord.\\
I will pay my vows now in the presence of all his people: right dear in the sight of the Lord is the death of his saints.\\
Behold (O Lord) how that I am thy servant: I am thy servant, and the son of thy handmaid, thou hast broken my bonds in sunder.\\
I will offer to thee the sacrifice of thanksgiving, and will call upon the Name of the Lord.\\
I will pay my vows unto the Lord, in the sight of all his people, in the courts of the Lord's house, even in the midst of thee, O Jerusalem.\\
Glory to the Father, etc.\\
As it was in the beginning, etc.

\emph{Lauda, anima, mea}. Psalm cxlvi.

Praise the Lord, (O my soul), while I live will I praise the Lord: yea, as long as I have any being, I will sing praises unto my God.

\emph{Note that this psalm is to be said after the others that followeth,}

\begin{quote}
O put not your trust in princes, nor in any child of man, for there is no help in them.\\
For when the breath of man goeth forth, he shall turn again to his earth, and then all his thoughts perish.\\
Blessed is he that hath the God of Jacob for his help: and whose hope is in the Lord his God.\\
Which made heaven and earth, the sea, and all that therein is: which keepeth his promise forever.\\
Which helpeth them to right that suffer wrong, which feedeth the hungry.\\
The Lord looseth men out of prison, the Lord giveth sight to the blind.\\
The Lord helpeth them up that are fallen, the Lord careth for the righteous.\\
The Lord careth for the strangers, he defendeth the fatherless and widow: as for the way of the ungodly, he turneth it upside down.\\
The Lord thy God, O Zion, shall be king for evermore, and throughout all generations.\\
Glory to the Father, etc.\\
As it was in the beginning, etc.\\

\emph{Domine, probasti}. Psalm cxxxix.

O Lord, thou hast searched me out, and known me. Thou knowest my down-sitting, and mine up-rising: thou understandest my thoughts long before.\\
Thou art about my path, and about my bed, and spiest out all my ways.\\
For lo, there is not a word in my tongue, but thou (O Lord) knowest it altogether.\\
Thou hast fashioned me, behind and before, and laid thine hand upon me.\\
Such knowledge is to wonderful and excellent for me: I cannot attain unto it.\\
Whither shall I go then from thy spirit? or whither shall I go then from thy presence?\\
If I climb up into heaven, thou art there: If I go down to hell, thou art there also.\\
If I take the wings of the morning, and remain in the uttermost parts of the sea;\\
Even there also shall thy hand lead me, and thy right hand shall hold me.\\
If I say: peradventure the darkness shall cover me, then shall my night be turned to day.Yea the darkness is no darkness with thee: but the night is all clear as the day, the darkness and light to thee are both alike.\\
For my reins are thine, thou hast covered me in my mother's womb: I will give thanks unto thee, for I am fearfully and wondrously made: marvelous are thy works, and that my soul knoweth right well.\\
My bones are not hid from thee, though I be made secretly, and fashioned beneath in the earth.\\
Thine eyes did see my substance, yet being unperfect: and in thy book were all my members written.\\
Which day by day were fashioned, when as yet there was none of them.\\
How dear are thy counsels unto me, O God? O how great is the sum of them?\\
If I tell them, they are more in number then the sand; when I wake up, I am present with thee.\\
Wilt thou not slay the wicked, O God? depart from me, ye bloodthirsty men.\\
For they speak unrighteously against thee: and then enemies take thy name in vain.\\
Do not I hate them, O Lord, that hate thee: and am not I grieved with those that rise up against thee?\\
Yea I hate them right sore, even as though they were mine enemies.\\
Try me, O God, and seek the ground of mine heart: prove me and examine my thoughts.\\
Look well if there be any way of wickedness in me, and lead me in the way everlasting.\\
Glory to the Father, etc.\\
As it was in the beginning, etc.
\end{quote}

\emph{Then shall follow this lesson, taken out of the XV. Chapter to the Corinthians, the first Epistle,}

\begin{quote}
Christ is risen from the dead, and become the firstfruits of them that slept. For by a man came death, and by a man came the resurrection of the dead. For as by Adam all die: even so by Christ shall all be made alive, but every man in his own order. The first is Christ, then they that are Christ's, at his coming. Then cometh the end, when he hath delivered up the kingdom to God the Father, when he hath put down all rule and all authority and power. For he must reign till he have put all his enemies under his feet. The last enemy that shall be destroyed, is death. For he hath put all things under his feet. But when he sayeth all things are put under him, it is manifest that he is excepted, which did put all things under him. When all things are subdued unto him, then shall the son also himself be subject unto him that put all things under him, that God may be all in all. Else what do they, which are baptized over the dead, if the dead rise not at all? Why are they then baptized over them? yea, and why stand we alway then in jeopardy? By our rejoicing which I have in Christ Jesus our Lord, I die daily. That I have fought with beasts at Ephesus after the manner of men, what advantageth it me, if the dead rise not again? Let us eat and drink, for tomorrow we shall die. Be not ye deceived: evil words corrupt good manners. Awake truly out of sleep, and sin not. For some have not the knowledge of God. I speak this to your shame. But some man will say: how arise the dead? with what body shall they come? Thou fool, that which thou sowest, is not quickened, except it die. And what sowest thou? Thou sowest not that body that shall be; but bare corn as of wheat, or of some other: but God giveth it a body at his pleasure, to every seed his own body. All flesh is not one manner of flesh: but there is one manner of flesh of men, another manner of flesh of beasts, another of fishes, another of birds. There are also celestial bodies, and there are bodies terrestrial. But the glory of the celestial is one, and the glory of the terrestrial is another. There is one manner glory of the sun, and another glory of the moon, and another glory of the stars. For one star differeth from another in glory. So is the resurrection of the dead. It is sown in corruption, it riseth again in incorruption. It is sown in dishonor, it riseth again in honor. It is sown in weakness, it riseth again in power. It is sown a natural body, it riseth again a spiritual body. There is a natural body, and there is a spiritual body: as it is also written: The first man Adam was made a living soul, and the last Adam was made a quickening spirit. Howbeit, that is not first which is spiritual: but that which is natural, and then that which is spiritual. The first man is of the earth, earthy: The second man is the Lord from heaven (heavenly). As is the earthy, such are they that are earthy. And as is the heavenly, such are they that are heavenly. And as we have borne the image of the earthy, so shall we bear the image of the heavenly. This say I brethren, that flesh and blood cannot inherit the kingdom of God: Neither doth corruption inherit uncorruption. Behold, I shew you a mystery. We shall not all sleep: but we shall all be changed, and that in a moment, in the twinkling of an eye by the last trump. For the trump shall blow, and the dead shall rise incorruptible, and we shall be changed. For this corruptible must put on incorruption: and this mortal must put on immortality. When this corruptible hath put on incorruption, and this mortal hath put on immortality: then shall be brought to pass the saying that is written: Death is swallowed up in victory: Death where is thy sting? Hell where is thy victory? The sting of death is sin: and the strength of sin is the law. But thanks be unto God, which hath given us victory, through our Lord Jesus Christ. Therefore, my dear brethren, be ye steadfast and unmovable, always rich in the work of the Lord, forasmuch as ye know that your labor is not in vain, in the Lord.
\end{quote}

\emph{The lesson ended then shall the Priest say,}

\begin{quote}
Lord, have mercy upon us.\\
Christ, have mercy upon us.\\
Lord, have mercy upon us.\\
Our Father which art in heaven, etc.\\
And lead us not into temptation.\\
\emph{Answer}. But deliver us from evil. Amen.

\emph{Priest}. Enter not (O Lord) into judgement with thy servant.\\
\emph{Answer}. For in thy sight no living creature shall be justified.\\
\emph{Priest}. From the gates of hell.\\
\emph{Answer}. Deliver their souls, O Lord.\\
\emph{Priest}. I believe to see the goodness of the Lord.\\
\emph{Answer}. In the land of the living.\\
\emph{Priest}. O Lord, graciously hear my prayer.\\
\emph{Answer}. And let my cry come unto thee.

Let us pray.

O Lord, with whom do live the spirits of them that be dead: and in whom the souls of them that be elected, after they be delivered from the burden of the flesh, be in joy and felicity: Grant unto us thy servant, that the sins which he committed in this world be not imputed unto him, but that he, escaping the gates of hell and pains of eternal darkness: may ever dwell in the region of height, with Abraham, Isaac, and Jacob, in the place where is no weeping, sorrow, nor heaviness: and when that dreadful day of the general resurrection shall come, make him to rise also with the just and righteous, and receive this body again to glory, then made pure and incorruptible, set him on the right hand of thy Son Jesus Christ, among thy holy and elect, that then he may hear with them these most sweet and comfortable words: Come to me ye blessed of my father, possess the kingdom which hath been prepared for you from the beginning of the world: Grant this we beseech thee, O merciful Father: through Jesus Christ our mediator and redeemer. Amen.
\end{quote}

\emph{The Celebration of Holy Communion when there is a Burial of the Dead}

\begin{quote}
\emph{Quemadmodum}. Psalm xlii.

Like as the hart desireth the water-brooks, so longeth my soul after thee, O God.\\
My soul is athirst for God, yea, even for the living God: when shall I come to appear before the presence of God?\\
My tears have been my meat day and night, while they daily say unto me, Where is now thy God?\\
Now when I think thereupon, I pour out my heart by myself: for I went with the multitude, and brought them forth unto the house of God, in the voice of praise and thanksgiving, among such as keep holyday.\\
Why art thou so full of heaviness, (O my soul): and why art thou so unquiet within me?\\
Put thy trust in God, for I will yet give him thanks for the help of his countenance.\\
My God, my soul is vexed within me: therefore will I remember thee concerning the land of Jordan, and the little hill of Hermon.\\
One deep calleth another, because of the noise of thy water pipes, all thy waves and storms are gone over me.\\
The Lord hath granted his lovingkindness on the day time, and in the night season did I sing of him, and made my prayer unto the God of my life.\\
I will say unto the God of my strength, why haste thou forgotten me? why go I thus heavily, while the enemy oppresseth me?\\
My bones are smitten asunder, while mine enemies (that trouble me) cast me in the teeth, namely while they say daily unto me: where is now thy God?\\
Why art thou so vexed, (O my soul) and why art thou so disquieted within me?\\
O put thy trust in God, for I will yet thank him which is the help of my countenance, and my God.\\
Glory to the Father, etc.\\
As it was in the beginning, etc.
\end{quote}

\emph{Collette.}

\begin{quote}
O merciful God the father of our Lord Jesus Christ, who is the resurrection and the life: In whom whosoever believeth shall live though he die: And whosoever liveth, and believeth in him, shall not die eternally: who also hath taught us (by his holy Apostle Paul) not to be sorry as men without hope for them that sleep in him: We meekly beseech thee (O Father) to raise us from the death of sin, unto the life of righteousness, that when we shall depart this life, we may sleep in him (as our hope is this our brother doeth), and at the general resurrection in the last day, both we and this our brother departed, receiving again our bodies, and rising again in thy most gracious favor: may with all thine elect Saints, obtain eternal joy. Grant this, O Lord God, by the means of our advocate Jesus Christ: which with thee and the Holy Ghost, liveth and reigneth one God for ever. Amen.
\end{quote}

\emph{The Epistle}. 1 Thess. iv.

\begin{quote}
I would not brethren that ye should be ignorant concerning them which are fallen asleep, that ye sorrow not as other do, which have no hope. For if we believe that Jesus died, and rose again: even so them also which sleep by Jesus, will God bring again with him. For this say we unto you in the word of the Lord: that we which shall live, and shall remain in the coming of the Lord, shall not come ere they which sleep. For the Lord himself shall descend from heaven with a shout, and the voice of the Archangel, and trump of God. And the dead in Christ shall arise first: then we which shall live (even we which shall remain) shall be caught up with them also in the clouds, to meet the Lord in the air. And so shall we ever be with the Lord. Wherefore comfort yourselves one another with these words.
\end{quote}

\emph{The gospel}. John vi.

\begin{quote}
Jesus said to his disciples and to the Jews: All that the Father giveth me, shall come to me: and he that cometh to me, I cast not away. For I came down from heaven: not to do that I will, but that he will, which hath sent me. And this is the Father's will which hath sent me, that of all which he hath given me, I shall lose nothing: but raise them up again at the last day. And this is the will of him that sent me: that every one which seeth the son and believeth on him, have everlasting life: And I will raise him up at the last day.\footnote{Original text from the 1549 Book of Common Prayer, with modernized spelling.}
\end{quote}

\hypertarget{updates}{%
\chapter*{Updates}\label{updates}}
\addcontentsline{toc}{chapter}{Updates}

\emph{Significant changes to this Book of Church Order will be listed here. For a~detailed diff hosted at Github, \href{https://github.com/Evangel-Presbytery/evangel-bco}{click here}.}

\begin{itemize}
\tightlist
\item
  \textbf{Current version:} Migrated from Pressbooks to Bookdown hosted on Github pages.
\item
  \href{https://www.dropbox.com/sh/e5lszl09qec2wy0/AADMEzOS1C1Z7Ao4r4xyddHNa?dl=0}{Hot Fix December 10, 2021}

  \begin{itemize}
  \tightlist
  \item
    The second paragraph of BCO 25.1 was amended on June 6, 2019. However, the entire paragraph was accidentally omitted from the published version of the BCO. This hotfix fixes that mistake.
  \end{itemize}
\item
  \href{https://www.dropbox.com/sh/67fg23e5sksui6j/AAB6LRjZ2dld47gYz1G0jo-Fa?dl=0}{Amended October 8, 2021}
\item
  \href{https://www.dropbox.com/sh/z9buy77lg1ay8t0/AAAnPPQysyhwRPJ7lpLxl3Bma?dl=0}{Amended June 3, 2021}
\item
  \href{https://www.dropbox.com/sh/yhg7o7s6vlx0jha/AABM5so3BTahNcU7rUWsZHAGa?dl=0}{Amended on October 31, 2019}
\item
  \href{https://www.dropbox.com/sh/6e1mcd2n8vojvpv/AADWRZsciqNw1xteiuRA0lr3a?dl=0}{Amended on June 6, 2019}
\item
  \href{https://www.dropbox.com/sh/w46pitp7sevpepk/AAAH_HDpb8Qpck42Dkfw661Za?dl=0}{Original Version released February 28, 2019}
\end{itemize}

\end{document}
